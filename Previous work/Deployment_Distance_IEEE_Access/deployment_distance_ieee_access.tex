\documentclass{../../Papers/ACCESS_latex_template_20240429 (1)/ACCESS_latex_template_20240429/ieeeaccess}
\usepackage{cite}
\usepackage{amsmath,amssymb,amsfonts}
\usepackage{graphicx}
\usepackage{textcomp}
\usepackage{multirow}
\usepackage{booktabs}
\usepackage{array}
\usepackage{float}

\def\BibTeX{{\rm B\kern-.05em{\sc i\kern-.025em b}\kern-.08em
    T\kern-.1667em\lower.7ex\hbox{E}\kern-.125emX}}

\begin{document}
\history{Date of publication xxxx 00, 0000, date of current version xxxx 00, 0000.}
\doi{10.1109/ACCESS.2024.XXXXXXX}

\title{Optimal Horizontal Deployment Distance for BLE-Based Pedestrian Monitoring in Outdoor Linear Pathways}

\author{\uppercase{Mayank Parmar}\authorrefmark{1},
\uppercase{Paula Kelly}\authorrefmark{2}, and
\uppercase{Damon Berry}\authorrefmark{3}}

\address[1]{TU Dublin, Grangegorman, Dublin, Ireland (e-mail: mayank.parmar@tudublin.ie)}
\address[2]{TU Dublin, Grangegorman, Dublin, Ireland (e-mail: paula.kelly@tudublin.ie)}
\address[3]{TU Dublin, Grangegorman, Dublin, Ireland (e-mail: damon.berry@tudublin.ie)}

\markboth
{Parmar \headeretal: Optimal Horizontal Deployment Distance for BLE-Based Pedestrian Monitoring}
{Parmar \headeretal: Optimal Horizontal Deployment Distance for BLE-Based Pedestrian Monitoring}

\corresp{Corresponding author: Mayank Parmar (e-mail: mayank.parmar@tudublin.ie).}

\begin{abstract}
Bluetooth Low Energy (BLE) is emerging as a privacy-preserving alternative for pedestrian monitoring in urban environments, yet optimal deployment configurations remain under-explored. This study investigates the impact of horizontal deployment distance on the quality of BLE signal measurements for pedestrian monitoring systems in outdoor linear pathways. We systematically evaluated four deployment distances (3~m, 5~m, 7~m, and 9~m) from a linear pathway using stationary pedestrian measurements at five key points, examining both Line-of-Sight (LoS) and Non-Line-of-Sight (nLoS) orientations with three repetitions each. Statistical analysis using ANOVA and Tukey's Honestly Significant Difference (HSD) test revealed that deployment distances of 3~m and 5~m produce significantly higher quality signals compared to 7~m and 9~m (p $<$ 0.05). Small-scale fading analysis through Rician distribution fitting demonstrated that 3~m and 5~m deployments exhibit superior LoS dominance, with K-factors reaching 117.84 at 5~m, indicating strong direct signal components and minimal multipath scattering. In contrast, 7~m and 9~m deployments showed predominantly scattered signals with K-factors approaching 0 at most measurement points, indicating Rayleigh fading conditions. These findings provide empirical evidence for optimal BLE Observer placement, contributing essential deployment guidelines for privacy-preserving pedestrian monitoring systems in outdoor environments.
\end{abstract}

\begin{keywords}
Bluetooth Low Energy, deployment distance, pedestrian monitoring, privacy-preserving sensing, RSSI, signal propagation, outdoor environments, Rician fading
\end{keywords}

\titlepgskip=-21pt

\maketitle

\section{Introduction}
\label{sec:introduction}

\PARstart{U}{nderstanding} pedestrian movement dynamics and activities are key elements of urban and transportation planning. Several modalities have been employed to measure and understand pedestrian movement, including optical sensors such as RGB and thermal cameras, surveys and questionnaires, WiFi, and Global Positioning System (GPS) systems. Optical sensors are prevalent in this regard due to accelerated advances in the technology and computational power necessary for processing large and complex information that these sensors acquire, resulting in accurate measurement. From the use of basic video cameras in the early studies \cite{Carroll2002,Lin2001,Masoud2001,Xu2005} to the use of modern vision methods and a fusion of vision sensors such as Kinect \cite{microsoft_kinect}, stereo vision, and thermal cameras \cite{Seer2014,Husman2021,Kristoffersen2016,Kunchala2023}, optical sensors have been at the forefront of measurement and analysis of pedestrian movement. However, with the increased pressure from privacy regulations such as the General Data Protection Regulation (GDPR), data collection using optical sensors has come under strict scrutiny. Additionally, the mechanism of data acquisition from these sensors is \textit{opportunistic}, which means that the acquisition process is not concerned with the consent of data contributors. Such systems will not discern between users who have consented from those who have not. To circumvent the issue of consent, additional manual effort is required to remove data acquired from non-consenting data contributors. Moreover, the cost of implementation of a vision-based system, both monetary and computationally, is another drawback \cite{Basalamah2016}.

GPS is another widely accepted and utilised modality for understanding pedestrian dynamics \cite{Draghici2018,McArdle2014,Blanke2014,Spek2008}. GPS, being a technology inherently developed for the purposes of navigation, is suitable for measuring pedestrians. It is efficient in capturing finer granularity pedestrian movement as the modality captures regular and precise movement. It is, however, \textit{participatory} in nature, meaning that it requires consent of its data contributors for participation giving it an advantage over opportunistic based monitoring systems in terms of privacy. This aspect however, confines the usability of the technology for acquiring pedestrian data to only a particular mechanism. GPS, however has other privacy concerns due to its inherent capability of tracking pedestrian movement at a finer scale. Literature often resolves this issue through \textit{geo-fencing} approaches, where a logical boundary is set up in the system which enables the measurement process only when a participant is within the experimental zone/region. Since studies facilitate measurements through GPS on participants' local devices, the collected data is often regularly uploaded to a cloud server. This leaves scope for unauthorised access such as man-in-the-middle attacks and also the possibility of lifting or changing the geo-fence.

Surveys and questionnaires are another important modality in pedestrian measurements. This modality is fully participatory, necessitating both objective and subjective responses directly from data contributors. While the advantages of this modality are significant due to direct responses requiring no assertions, they are prone to survey fatigue, resulting in biases. Biases might also emerge from psychological factors, social desirability bias. Moreover, there is time and resource cost associated with recruiting and training surveyors and data contributors, and the process itself is time consuming.

The restriction on the usage of conventional sensors, primarily due to strict privacy regulations has pushed researchers to adapt and identify unconventional sensing technologies such as WiFi and Bluetooth Low Energy (BLE) for capturing the movement of pedestrians. WiFi, due to its availability on mobile phones, and readily available infrastructure due to the ubiquity of public WiFi hotspots provided by various city councils and agencies \cite{Omar2016}, has more purchase in pedestrian-related research over BLE. This is also evident from a comparison of the number of scientific publications over the past ten years between the two technologies, as presented in Fig.~\ref{fig:wifi_ble_comparison} \cite{Dimensions2024}. Both technologies operate in a similar fashion where signal strength is collected and patterns in the signals are used for asserting pedestrian dynamics. WiFi is also capable of capturing more complex forms of information through CSI which provides more agency over its assertion in comparison to BLE. However, processing CSI data requires more computational capabilities.

\begin{figure}[!t]
\centering
\includegraphics[width=3.5in]{figures/wifi vs ble.png}
\caption{Comparison of the number of pedestrian-related publications using BLE and WiFi \cite{Dimensions2024}. Data extracted from Dimensions, the world's largest linked research database.}
\label{fig:wifi_ble_comparison}
\end{figure}

However, the adoption of BLE for this type of research cannot be overlooked. BLE provides superior flexibility over WiFi. The use of WiFi through existing infrastructure, originally meant to provide internet connectivity to the users, puts additional pressure on the infrastructure, thereby, reducing its throughput. Setting up an exclusive WiFi infrastructure for pedestrian measurement is often costly compared to BLE. BLE offers a significant advantage in the form of inexpensive beacons for asset tracking. These beacons can be provided to data contributors both as an incentive and as a mechanism to capture their data without requiring the use of their personal device. BLE offers inherently privacy preserving features which will be presented in Section~\ref{sec:background}.

Whilst the use of BLE as a tool to understand pedestrian movement dynamics and activities has become more common in academic research, studies entailing the assessment of a proper deployment of a BLE monitoring system, referred in this paper as an Observer, are sparse. As will be presented in the next section, the influence of an Observer deployment on the quality of the acquired data is significant. Poor data quality in the acquisition stage also affects data analysis and subsequently, assertions from the acquired data. Therefore, in this study, we present an approach and its usefulness in identifying optimal deployment distance, particularly on a horizontal plane, for a BLE Observer.

\subsection{Research Question and Contributions}
This paper addresses the following research question: \textit{What is the optimal horizontal deployment distance for a BLE Observer to achieve reliable signal quality for pedestrian monitoring in outdoor linear pathways?}

The main contributions of this work are:
\begin{itemize}
\item Systematic evaluation of four horizontal deployment distances (3~m, 5~m, 7~m, and 9~m) using stationary pedestrian measurements across five key points and two orientations (LoS and nLoS).
\item Statistical validation using ANOVA and Tukey's HSD test demonstrating significant differences in signal quality across deployment distances.
\item Small-scale fading analysis using Rician distribution fitting to quantify multipath effects and LoS dominance at different deployment distances.
\item Empirical evidence establishing 3~m and 5~m as optimal deployment distances for outdoor BLE-based pedestrian monitoring systems.
\item Practical deployment guidelines for researchers and practitioners implementing BLE monitoring infrastructure.
\end{itemize}

The remainder of this paper is organised as follows: Section~\ref{sec:background} provides background on BLE technology and signal propagation. Section~\ref{sec:related_work} reviews related work on BLE deployment and pedestrian sensing. Section~\ref{sec:methodology} describes the experimental methodology. Section~\ref{sec:results} presents the results. Section~\ref{sec:discussion} discusses the findings, limitations, and practical guidelines. Section~\ref{sec:conclusion} concludes the paper.

