\chapter{Evaluation} \label{ch:res}

\vspace{1cm}
\noindent\enquote{\itshape Everything has its wonders, even darkness and silence, and I learn, whatever state I may be in, therein to be content.}\bigbreak
\hfill $\thicksim$ \textit{Helen Keller}
\vspace{1cm}

This chapter is aimed at presenting the evaluation of experiments introduced in Chapter \ref{ch:meth}. The chapter is divided into two sections, related to each evaluation case described in the Methodology chapter. The first set of experiments was conducted to ascertain the viability of the technology, \gls{ble}, and its features for measuring pedestrian movement dynamics. This set comprised experiments to evaluate the characteristics of the antenna on the Observer, understand the fading effects of the environment where experiments are conducted, identify optimal deployment distance for the Observer, and assess the effect of the advertisement interval or advertisement rate of the Broadcaster on the measurement of \gls{rssi} by the Observer. Through these experiments, ultimately, the technical viability of \gls{ble} and the selected hardware platform and the environment was assessed.
\\

The second set of experiments was aimed at evaluating the ability and the accuracy of the Observer in detecting scenarios of heterogeneous pedestrian activities and movement dynamics. The developed system was hence applied to detect pauses in the movement of pedestrians, identify the likelihood of interaction between two pedestrians, ascertain the direction of travel of the pedestrians, and analyse the behaviour of pedestrians through a long-term study on a university campus.
\\

All the outdoor experiments, barring the final one in this chapter, were conducted using a single Observer at the same location, with one or multiple pathways, each 24 $metres$ in length, with five equidistant key points on the pathways -- \textit{start}, \textit{approach}, \textit{centre}, \textit{depart}, and \textit{end}, six $metres$ from each other, and a fixed location for deployment of the Observer perpendicular to the \textit{centre} point from all pathways. The decision to conduct experiments in a single location was to create coherent results, obtained from the same building topology. The decision for a 24 $metres$ long pathway was also deliberate as the focus of this study was on micro-level pedestrian dynamics. Moreover, to obtain clear local measurements, that is, observe pedestrian dynamics more accurately, measurement of stronger signals were deemed necessary to facilitate the study.
\\

Finally, the date of experiments, span of experiment procedure, number and detail of volunteers involved, and general weather condition at the time of experiments is provided in the respective section of each experiment in this chapter. The orientation of the Observer was the same for all the experiments, as depicted in Figure \ref{fig:meth/rpi_orientation} in Section \ref{subsub:obsbro} of Chapter \ref{ch:meth}. The hourly weather information was acquired from the nearby Phoenix Park synoptic weather station by Met Éireann \citep{phoenix_park_hourly_data}. The weather is 4.6 $kms$ away from the experimental location.
\\

The following figure illustrates the progression of experiments in this chapter. This figure will be used repeatedly to indicate the progress through the different experimental results and associated analysis.
\begin{figure}[htbp]
	\centerline{\includegraphics[width=\textwidth]{Figures/Ch4/timeline/0.png}}

\end{figure}

\section{Evaluation of the Technology and its Features}\label{sec:res/tinker}

\subsection{Characteristics of the Antenna of the Observer} \label{subsec:res/antenna}


\begin{figure}[htbp]
	\centerline{\includegraphics[width=\textwidth]{Figures/Ch4/timeline/timeline_antenna.png}}

\end{figure}

This experiment was divided into three parts at two different locations, as mentioned in Chapter \ref{ch:meth} under the subsection \ref{subsec:meth/antenna}. The first part corresponded to the measurements collected in an anechoic chamber where the Observer was rotated, as shown in figure \ref{fig:meth/plane}, such that the signals were emerging from 0\textdegree{}, 45\textdegree{}, 90\textdegree{}, 135\textdegree{}, and 180\textdegree{}, for 3 $minutes$. This specific resolution was chosen to avoid collecting fewer samples when replicating the experimental setup outdoors. With finer angular resolution than the selected, the resulting \gls{rssi} samples will change rapidly as a pedestrian moves along the path. Whereas, valuable details between the chosen angles might be missed with a larger angular resolution than the selected one. The size limitations of the anechoic chamber facility in the university prevented the experiments from being performed at a distance greater than 3 $meters$ between the Observer and the Broadcaster.
\\

The second part was conducted in an outdoor environment presented in Figure \ref{fig:meth/satellite_exp}. The measurements were first taken radially around the deployed Observer, to replicate the measurements taken in the anechoic chamber, at distances of 3 $metres$, 5 $metres$, 7 $metres$, and 9 $metres$, for 3 $minutes$ each. In this part, a sub-experiment was also conducted to measure the signal strength, the \gls{rssi}, at five different locations on two linear pathways 3 $metres$ and 5 $metres$ away from the deployed Observer.
\\

Measurements for the experiment conducted in the anechoic chamber were taken on April 4, 2023, starting 14:32 Irish time. Since the measurement was conducted in a noiseless environment, there was no influence of weather conditions or any other external factor. The data collection process in anechoic chamber ended on the same day at 14:45 hours Irish time. The data for the second experiment that measured \gls{rssi} at selected angles around the Observer in outdoors were collected between 14:17 hours Irish time and 14:31 hours on February 10, 2023. The weather information during this experiments is described below:
\\
\\
\textbf{At 14:00 hours on February 10, 2023}
\begin{itemize}
    \item \textbf{Precipitation (Rain):} $0.0$ $mm$
    \item \textbf{Air Temperature:} $11.8$ \textdegree{}$C$
    \item \textbf{Wet Bulb Temperature:} $10.1$ \textdegree{}$C$
    \item \textbf{Dew Point Temperature:} $8.4$ \textdegree{}$C$
    \item \textbf{Vapour Pressure:} $11.0$ hPa
    \item \textbf{Relative Humidity:} $80$ \%
    \item \textbf{Mean Sea Level Pressure:} $1032.2$ hPa
\end{itemize}

Finally, data for the third, linear pathway, sub-experiment was collected between 11:43 hours and 13:17 hours Irish time on February 1, 2023. Since the experiments spanned over an hour, the weather information presented below is from 12:00 hours and 13:00 hours. The weather information during this experiments are summarised below:
\\
\\
\textbf{At 12:00 hours on February 1, 2023}
\begin{itemize}
    \item \textbf{Precipitation (Rain):} $0.0\,\text{mm}$
    \item \textbf{Air Temperature:} $9.4\,^\circ\mathrm{C}$
    \item \textbf{Wet Bulb Temperature:} $7.8\,^\circ\mathrm{C}$
    \item \textbf{Dew Point Temperature:} $6.0\,^\circ\mathrm{C}$
    \item \textbf{Vapour Pressure:} $9.3\,\text{hPa}$
    \item \textbf{Relative Humidity:} $79\,\%$
    \item \textbf{Mean Sea Level Pressure:} $1024.1\,\text{hPa}$
\end{itemize}

\textbf{At 13:00 hours on February 1, 2023}
\begin{itemize}
    \item \textbf{Precipitation (Rain):} $0.0\,\text{mm}$
    \item \textbf{Air Temperature:} $10.0\,^\circ\mathrm{C}$
    \item \textbf{Wet Bulb Temperature:} $8.0\,^\circ\mathrm{C}$
    \item \textbf{Dew Point Temperature:} $5.6\,^\circ\mathrm{C}$
    \item \textbf{Vapour Pressure:} $9.1\,\text{hPa}$
    \item \textbf{Relative Humidity:} $74\,\%$
    \item \textbf{Mean Sea Level Pressure:} $1023.7\,\text{hPa}$
\end{itemize}


First, each of the three 1-$minute$ segments of measurements were collated and their median was calculated. This collation and calculation was performed for data collected in the anechoic chamber and data collected around the Observer in the outdoor experimental space. The polar charts presented in Figure \ref{fig:res/polar_anechoic} highlights the median \gls{rssi} of the entire sample of data for each angle in the anechoic chamber. Similarly, Figure \ref{fig:res/polar_outdoor} presents the median \gls{rssi} of the entire sample of data for each angle in the outdoor environment, including the additional data captured at a distance of 5 $metres$, 7 $metres$, and 9 $metres$.
\\

\begin{figure}[htbp]
	\centerline{\includegraphics[width=\textwidth]{Figures/Ch4/antenna characteristics/polarchart anechoic.jpg}}
	\caption{Median \gls{rssi} in Anechoic Chamber for Selected Angles of Measurement}
	\label{fig:res/polar_anechoic}
\end{figure}
\begin{figure}[htbp]
	\centerline{\includegraphics[width=\textwidth]{Figures/Ch4/antenna characteristics/polarchart outdoor.jpg}}
	\caption{Median \gls{rssi} in Outdoor Environment for Selected Angles and Distances of Measurement}
	\label{fig:res/polar_outdoor}
\end{figure}

The most striking observation when comparing the median \gls{rssi} in the anechoic chamber against the median \gls{rssi} at 3 $metres$ deployment distance in the outdoor environment is a significantly improved signal strength in the outdoor environment. Despite being a noiseless environment, the measurements in the anechoic chamber are weaker. While this greater \gls{rssi} in the outdoor environment could be an indication of some propagation mechanism acting as a boost on the signals, assessment of \gls{rssi} information alone is not sufficient to assert anything about it. To better understand the data, standard deviation and variance were then calculated for both of these cases. The outcome is presented in Table \ref{tab:res/stdvar_ant}.
\\

\begin{table}[h!]
    \centering
    \arrayrulecolor{DarkOliveGreen} % Set the color of table rules    
    \begin{tabular}{>{\columncolor{MintCream}}c | >{\columncolor{MintCream}}c >{\columncolor{MintCream}}c | >{\columncolor{MintCream}}c >{\columncolor{MintCream}}c}
        \hline
         \rowcolor{MidnightGreen}
          & \multicolumn{2}{c|}{\textcolor{white}{\textbf{Anechoic}}}& \multicolumn{2}{c}{\textcolor{white}{\textbf{Outdoor}}}\\
           \rowcolor{MidnightGreen} 
          \multirow{-2}{*}{\textcolor{white}{\textbf{Angle}}}  & \textcolor{white}{\textbf{Standard Deviation}} & \textcolor{white}{\textbf{Variance}} & \textcolor{white}{\textbf{Standard Deviation}} & \textcolor{white}{\textbf{Variance}}\\
        \hline
        0\textdegree{} & 0.1554& 0.0241& 2.7190& 7.3931\\
        45\textdegree{} & 0.3552 & 0.1262 & 1.9116& 3.6544\\
        90\textdegree{} & 0.2465& 0.0608& 3.2216& 10.3788\\
        135\textdegree{} & 0.2913& 0.0848& 2.6016& 6.7685\\
        180\textdegree{} & 0.1595& 0.0255 & 1.0197& 1.0397\\
        \hline\hline
    \end{tabular}
    \caption{Comparison of Standard Deviation and Variance of RSSI Collected in an Anechoic Chamber and at 3 Metres Around the Observer in Outdoor Experimental Location}
    \label{tab:res/stdvar_ant}
\end{table}

Through the calculation of \gls{sd} and Variance, it was seen that the fluctuations in RSSI were significant in the outdoor measurements compared to those in the anechoic chamber measurements. However, this did not explain the overall increased strength of the signals in the outdoor measurements. It only signified that the spread of values of the collected measurements in the anechoic chamber was lower, indicating that the environment has no external electromagnetic and topological influences which resulted in fewer obstructions in the path of the signal. The increased signal strength in outdoor measurements could be due to several factors. Some of these factors might include constructive interference of reflected signals multipath components \citep{Molisch2012} and less absorption of radio waves outdoors than in an anechoic chamber, where, the walls and floor use material that absorbs them \citep{Krchnak2024}.
\\

The fluctuations were then plotted using the error bars, depicting the spread of the measurements. The error bars were obtained through \gls{mda} technique described in Section \ref{subsec:meth/analysis/mda} in Chapter \ref{ch:meth}. The positive and negative errors represented the upper and lower deviation of the \gls{rssi} from the median respectively, that is, the extent of variation above and below the median value. The median value is also plotted on the error bars, denoted by the marker `$\times$'. These calculations were made on the \gls{sma} filtered \gls{rssi}, described in Section \ref{subsec:meth/analysis/sma} of Chapter \ref{ch:meth}, which are seen on the y-axis. The resulting chart is presented in Figure \ref{fig:res/ant/median_fluctuation_anechoic_outdoor}. The measurements from the anechoic chamber are shown in green lines and error bars, whereas, red and blue lines and error bars depict the measurements from the outdoor experiment with the deployment distances of 3 $metres$ and 5 $metres$ respectively. The large error bars in both cases of the outdoor experiments in comparison to the anechoic chamber experiment are a clear indication of the noise produced due to the environmental artefacts. 
\\

\begin{figure}[htbp]
	\centerline{\includegraphics[width=\textwidth]{Figures/Ch4/antenna characteristics/median and fluctuation anechoic outdoor.jpg}}
	\caption{Span and Median of \gls{rssi} Emanating from Selected Angles in an Anechoic Chamber and Selected Angles Around 3-$metre$ and 5-$metre$ Radius Respectively in the Outdoor Environment}
	\label{fig:res/ant/median_fluctuation_anechoic_outdoor}
\end{figure}

% It is noteworthy that in Figure \ref{fig:res/ant/median_fluctuation_anechoic_outdoor}, the \gls{rss} values at 45\textdegree{} and 135\textdegree{} are higher than the \gls{rss} value at 90\textdegree{} in the anechoic chamber. Since the environment inside the anechoic chamber is immutable, the outcome strongly suggests that the antenna of the Observer (\gls{rpi}) is more receptive (or sensitive) to the signals arriving towards the sides of it than to the signals arriving at the frontal face of it. This produces a pattern in the signal that appears to form a \textit{double hump} shape. The same double hump pattern is also observed in the outdoor experiment in the case of the 3 $metres$ deployment distance. While the double hump pattern is not present in the case of 5 $metres$ deployment distance, the \gls{rssi} of the signals arriving from the 45\textdegree{} region is higher than the \gls{rssi} of the signals arriving directly at the frontal face of the Observer, that is from the 90\textdegree{} region. This is indicative of a greater sensitivity of the antenna at least towards the signals arriving from the 45\textdegree{} region, which persists in the outdoor environment as well as it is in the anechoic chamber.
% \\

% It is important to note that the fluctuations represented by the span of the error bars, in the Figure \ref{fig:res/ant/median_fluctuation_anechoic_outdoor} for the outdoor experiment are also large because the measurements are collected for long intervals of time at each of those locations. This allows more time for the reflected, refracted, and diffracted signals to be observed. Realistically, since in the real world the pedestrian will most likely be walking, there are unlikely to be as many measurements per location. Moreover, since the research is mostly concerned with linear pathways, the measurements from the anechoic chamber are not representative of those for the linear pathway. Therefore, the next step is to assess the measurements obtained from a linear pathway.
% \\

The third and final part of this experiment targeted \gls{rssi} measurement on linear pathways in front of the Observer, instead of measurements radially around the Observer. Figure \ref{fig:res/ant/median_fluctuation_outdoor_pathway} depicts the median and fluctuation of measurement at \textit{start}, \textit{approach}, \textit{centre}, \textit{depart}, and \textit{end} key points on the pathway (as depicted in figure \ref{fig:meth/layout}) at a distance of 3 $metres$ from the Observer coloured in green and the measurements collected radially around the Observer at a distance of 3 $metres$ and 5 $metres$ in red and blue respectively. It was already intuitive that the signal strength at the extreme ends of the pathway, \textit{start} and \textit{end}, could be low since those locations are the farthest from the Observer on both the linear pathways, which was also confirmed through the \gls{rssi} measurements, as seen in Figure \ref{fig:res/ant/median_fluctuation_outdoor_pathway}. The difference between the \gls{rssi} obtained at each location on 3-$metre$ deployment distance and the \gls{rssi} obtained radially around the Observer at 5-$metre$ distance is lesser than the difference between those values at 3-$metre$ pathway and 3-$metre$ radial measurements. While through this it can be inferred that the \gls{rssi} measured at each point on 3-$metre$ deployments distance has a closer resemblance to the \gls{rssi} measurement radial around the Observer at 5-$metres$, it adds no value to the experiment objective. These differences are however presented in Table \ref{tab:res/diff_5m_pathway_3m_ant}.
\\
\begin{figure}[htbp]
	\centerline{\includegraphics[width=\textwidth]{Figures/Ch4/antenna characteristics/median and fluctuation outdoor around vs pathway.jpg}}
	\caption{Span and Median of \gls{rssi} Emanating from Selected Key Points and Angled on the Pathway 3-$metre$ Away from the Observer and Around 3-$metre$ and 5-$metre$ Radius Respectively in the Outdoor Environment}
	\label{fig:res/ant/median_fluctuation_outdoor_pathway}
\end{figure}

\begin{table}[h!]
    \centering
    \arrayrulecolor{DarkOliveGreen} % Set the color of table rules    
    \resizebox{\textwidth}{!}{\begin{tabular}{>{\columncolor{MintCream}}l | >{\columncolor{MintCream}}c >{\columncolor{MintCream}}c}
        \hline
         \rowcolor{MidnightGreen}
          {\textcolor{white}{\textbf{\small{Location/Angle}}}}& \textcolor{white}{\textbf{\small{|\gls{rssi} at 3m, radially around $-$ \gls{rssi} at Pathway|}}}& \textcolor{white}{\textbf{|\small{\gls{rssi} at 5m, radially around $-$ \gls{rssi} at Pathway|}}}\\
        \hline
        Start / 0\textdegree{} & 31.4& 26.8\\
        \hline
        Approach / 45\textdegree{} & 32.3& 29.35\\
        \hline
        Centre / 90\textdegree{} & 14.2& 10.95\\
        \hline
        Depart / 135\textdegree{} & 23.92& 12.92\\
        \hline
        End / 180\textdegree{} & 14.2& 12.9\\         
        \hline\hline
    \end{tabular}}
    \caption{Comparison the Distances between the Median \gls{rssi} of Broadcaster Located 5 $metres$ Around the Observer and at Locations on the Pathway Against the \gls{rssi} from Broadcaster Located 3 $metres$ Around the Observer}
    \label{tab:res/diff_5m_pathway_3m_ant}
\end{table}

While there was no resemblance between the double hump pattern obtained in the anechoic chamber and at 3 $metres$ deployment around the Observer with the \gls{rss} pattern obtained at different locations on the pathway, it was noted that the volunteer pedestrian was instructed to stand still at each of those locations. To establish whether or not the double hump pattern appears in the outdoor pathway for this experimental configuration and topology, another condition was tested where the pedestrian was instructed to move along the pathway. Since, the investigation of the movement of the pedestrian will follow in Sections \ref{sec:res/deployment_distance} and \ref{subsec:res/occlusion}, the case of movement was not tested here. However, clear directional sensitivity of the Observer's antenna for \gls{ble} signals arriving from 45\textdegree{} and 135\textdegree{} was observed in both the anechoic chamber and at 3 $metres$ around the Observer. Therefore, for this particular experimental configuration, there is a likelihood of finding a peak in the collected \gls{rss} values of a Broadcaster carried by a pedestrian in motion at an angle of 45\textdegree{},  the \textit{approach} key point or \textit{approach} region, from the plane of the Observer. Also, the error bars in Figure \ref{fig:res/ant/median_fluctuation_outdoor_pathway} showed that the total span of these measurements in both of these cases have overlapping \gls{rss} values, suggesting that there could be a likelihood, however small, of another lesser peak in the plot at an angle of 135\textdegree{} or, closer to the \textit{depart} key point.
\\

\vspace{10pt}

    \begin{tcolorbox}[
        colframe=white, % Border color
        colback=Ivory, % Background color
        coltitle=white, % Title text color
        title=Section Summary, % Title text
        fonttitle=\bfseries, % Title font style
        sharp corners, % Sharp corners for the main box
        enhanced,
        attach boxed title to top left={yshift=-2mm, xshift=2mm}, % Positioning the title
        boxed title style={colback=DarkSlateGray, colframe=SlateBlue, rounded corners=west, boxrule=0pt, top=2mm, bottom=2mm, right=2mm}, % Title box style
        width=\linewidth, % Width of the box
        boxrule=0pt, % No border for the main box
        drop shadow, % Shadow effect
        rounded corners, % Rounded corners for the main box
    ]
        The findings from this experiment are useful. One may intuitively think that the peak must be observed when the Broadcaster is directly opposite the Observer. However, as seen in this case, there is a greater likelihood of finding a peak before the \textit{centre} region and a lesser likelihood of finding a second peak after that region when the pedestrian is walking from the \textit{start} region to the \textit{end} region. This demonstrates that any studies or use of this technology to detect the proximity of pedestrians without first investigating the characteristics of the receiver antenna may result in feigned proximity. This will be further elucidated in Section \ref{sec:disc/obj2} in Chapter \ref{ch:disc}. The shape of the pattern can also potentially aid in the identification of the direction of a pedestrian simply by observing the pattern of the \gls{rssi}  and comparing it to the antenna characteristics. 
\\
\\
The work presented in this section is part of the publication titled, "Indication of Pedestrian's Travel Direction Through Bluetooth Low Energy Signals Perceived by a Single Observer Device" \citep{Parmar2023}.
    \end{tcolorbox}

% \begin{mdframed}[roundcorner=7pt]
%     The findings from this experiment are useful. One may intuitively think that the peak must be observed when the Broadcaster is directly opposite the Observer. However, as seen in this case, there is more likelihood of finding a peak before the \textit{centre} region and a smaller likelihood of finding a second peak after that region when the pedestrian is walking from the \textit{start} region to the \textit{end} region. This demonstrates that any studies or use of this technology to detect the proximity of pedestrians without first investigating the characteristics of the receiver antenna may result in feigned proximity. The shape of the pattern can also potentially aid in the identification of the direction of a pedestrian simply by observing the pattern of the \gls{rssi}  and comparing it to the antenna characteristic. 
% \\
% \\
% The work presented in this section is part of the publication titled, "Indication of Pedestrian's Travel Direction Through Bluetooth Low Energy Signals Perceived by a Single Observer Device" \citep{Parmar2023}.
% \end{mdframed}

\subsection{\acrfull{ls} and \acrfull{ss} Fading} \label{sec:res/fading}

\begin{figure}[htbp]
	\centerline{\includegraphics[width=\textwidth]{Figures/Ch4/timeline/timeline_fading.png}}

\end{figure}

Fading underscores the influence of environmental factors in the attenuation of a \gls{ble} signal. To understand the effect of the surrounding environment on the location selected for all outdoor experiments, except the last experiment, in this research, both \gls{ls} and \gls{ss} fading components were investigated. The measurements used to evaluate both types of fading were acquired for other experiments -- \gls{ls} fading using data acquired from the self-body occlusion experiment described in Section \ref{subsec:res/occlusion} and \gls{ss} fading using data acquired from the optimal deployment distance experiment described in Section \ref{sec:res/deployment_distance}, both in this chapter. Therefore, the day, time, duration, and weather condition during the experiments will be presented in the respective sections.
\\

\subsubsection{LS Fading}
\begin{wrapfigure}{l}{0.45\textwidth}
\vspace{-6mm}
    \begin{tcolorbox}[
        colframe=Teal, % Border color
        colback=MintCream, % Background color
        coltitle=white, % Title text color
        title=\footnotesize{Quick Recap}, % Title text
        fonttitle=\bfseries, % Title font style
        sharp corners, % Sharp corners for the main box
        enhanced,
        attach boxed title to top left={yshift=-2mm, xshift=2mm}, % Positioning the title
        boxed title style={colback=MidnightGreen, colframe=Teal, rounded corners=west, boxrule=0pt, top=2mm, bottom=2mm, right=2mm}, % Title box style
        width=\linewidth, % Width of the box
        boxrule=0pt, % No border for the main box
        drop shadow, % Shadow effect
        rounded corners, % Rounded corners for the main box
    ]
        \scriptsize{\textit{\gls{ls} fading}: Also known as Shadowing, it is a type of wireless signal attenuation that occurs due to environmental factors over long distances and correspond to presence of large obstructions in the path of the signals. Described in Section \ref{sec:lit/ble/signal} in Chapter \ref{ch:lit}.}
    \end{tcolorbox}    
    \vspace{-8mm}
\end{wrapfigure}

Before calculating the \gls{ls} fading component, a reference \gls{rssi} was obtained for the pair of devices by averaging the measurements at all angles in the anechoic chamber, as described in Section \ref{subsec:meth/evalexp/fading} in Chapter \ref{ch:meth}. The obtained results are: 
\\\\
\tab$\gls{rssi}_{ref} = -57.439 \text{ } dB$\\
\tab$d_{ref} = 3 \text{ } metres$
\\


These reference values were then used to calculate \gls{ls} fading through curve fitting. The \gls{ls} residuals were calculated by assessing the difference between the mean \gls{rssi} obtained at five key points to understand the impact of either shadowing or reflection at each of the key points on the pathway.
\\

The calculation of \gls{ls} fading was performed by comparing the measured \gls{rssi} at various locations with the values predicted by a path loss model. Figure \ref{fig:res/fading/ls} provides a visual comparison between the measured mean \gls{rssi} and the fitted path loss model across five key points:  \textit{start}, \textit{approach}, \textit{centre}, \textit{depart}, and \textit{end}. The calculations are also presented in Table \ref{tab:ls_fading}. In the table, it can be observed that the mean \gls{rssi} varies significantly between different key points, with the strongest at \textit{centre} (-44.71 $dB$) and the weakest at \textit{end} (-68.13 $dB$). Residuals, the difference between mean of the measured \gls{rssi} and the predicted \gls{rssi} obtained from model fitting, determine if other factors apart from distance attenuate the signals. Table \ref{tab:ls_fading} also presents significant residuals, varying from -6.25 $dB$ at \textit{end} key point to 7.69 $dB$ at \textit{approach}. Negative residuals in the table signify that the signals are weaker than expected, potentially due to obstacles, shadowing, or interference, suggesting that the model overestimates the signal strength at that location. Negative residuals were observed at \textit{start}, \textit{depart}, and \textit{end} key points on the pathway, with substantially large negative value at \textit{end} key point. Whereas, the positive residual, signifying that the measure \gls{rssi} is higher than predicted \gls{rssi}, indicates the presence of constructive interference, reflections, or an unobstructed \gls{los}. Positive residual were observed at \textit{approach} and \textit{centre} key points. The Relative Error signifies the percentage deviation of the measured value over the predicted value.

\begin{figure}[htbp]
	\centerline{\includegraphics[width=\textwidth]{Figures/Ch4/evaluation/fading/large_scale_fading.png}}
	\caption{Comparison of Mean \gls{rssi} and Predicted \gls{rssi}}
	\label{fig:res/fading/ls}
\end{figure}


\begin{table}[htbp]
    \centering
     \resizebox{\textwidth}{!}{\begin{tabular}{>{\columncolor{MintCream}}c >{\columncolor{MintCream}}c >{\columncolor{MintCream}}c >{\columncolor{MintCream}}c >{\columncolor{MintCream}}c}
    \hline
        \rowcolor{MidnightGreen}
        \textcolor{white}{\textbf{Location}} & \textcolor{white}{\textbf{Mean RSSI (dBm)}} & \textcolor{white}{\textbf{Fitted RSSI (dBm)}} & \textcolor{white}{\textbf{\gls{ls} Residual (dBm)}} & \textcolor{white}{\textbf{Relative Error Percentage (\%)}}\\
    \hline
        start & -63.26 & -61.88 & -1.38 & -2.23\\
        approach & -47.54 & -55.23 & 7.69 & 13.92 \\
        centre & -44.71 & -46.50 & 1.79 & 3.85\\
        depart & -57.09 & -55.23 & -1.86 & -3.37\\
        end & -68.13 & -61.88 & -6.25 & -10.10\\
    \hline\hline
    \end{tabular}}
    \caption{Large-Scale Fading}
    \label{tab:ls_fading}
\end{table}
\FloatBarrier

\subsubsection{SS Fading}
\begin{wrapfigure}{r}{0.45\textwidth}
\vspace{-1mm}
    \begin{tcolorbox}[
        colframe=Teal, % Border color
        colback=MintCream, % Background color
        coltitle=white, % Title text color
        title=\footnotesize{Quick Recap}, % Title text
        fonttitle=\bfseries, % Title font style
        sharp corners, % Sharp corners for the main box
        enhanced,
        attach boxed title to top left={yshift=-2mm, xshift=2mm}, % Positioning the title
        boxed title style={colback=MidnightGreen, colframe=Teal, rounded corners=west, boxrule=0pt, top=2mm, bottom=2mm, right=2mm}, % Title box style
        width=\linewidth, % Width of the box
        boxrule=0pt, % No border for the main box
        drop shadow, % Shadow effect
        rounded corners, % Rounded corners for the main box
    ]
        \scriptsize{\textit{\gls{ss} fading}: A type of wireless signal attenuation that refers to rapid fluctuations in signal amplitude, phase, or frequency over short distances or time-periods, typically caused by the interference of multiple signal path, known as multipath propagation. Described in Section \ref{sec:lit/ble/signal} in Chapter \ref{ch:lit}.}
    \end{tcolorbox}    
    \vspace{0mm}
\end{wrapfigure}

\gls{ss} fading was analysed by fitting the Rician distribution, described in Section \ref{subsec:meth/analysis/rician} in Chapter \ref{ch:meth}, to the deviations of measured \gls{rssi} from the mean \gls{rssi} at each key point. Figure \ref{fig:res/fading/ss} depicts the \gls{pdf}s of fitted Rician distributions corresponding to the fluctuations caused by multipath propagation at each key point, and the values are summarised in Table \ref{tab:ss_fading}. The Shape parameter, $K$, which is used to indicate the dominance of \gls{los} component, is seen highest and substantially large at the \textit{start} key point, signifying the dominance of \gls{los} component. Key points \textit{depart} and \textit{end} corresponded to Rayleigh distribution, indicating the presence of only multipath components. Finally, \textit{approach} and \textit{centre}, with a Shape parameter close to zero, indicated the dominance of multipath components with the presence of some \gls{los} components.
\\

\begin{figure}[htbp]
	\centerline{\includegraphics[width=\textwidth]{Figures/Ch4/evaluation/fading/small_scale_fading.png}}
	\caption{Rician Distribution Fitting to Deviations at Each key point}
	\label{fig:res/fading/ss}
\end{figure}
\begin{table}[htbp]
    \centering
     \resizebox{\textwidth}{!}{\begin{tabular}{>{\columncolor{MintCream}}c >{\columncolor{MintCream}}c >{\columncolor{MintCream}}c >{\columncolor{MintCream}}c >{\columncolor{MintCream}}c}
    \hline
        \rowcolor{MidnightGreen}
        \textcolor{white}{\textbf{Location}} & \textcolor{white}{\textbf{Mean RSSI (dBm)}} & \textcolor{white}{\textbf{Fitted RSSI (dBm)}} & \textcolor{white}{\textbf{Shape Param ($K$)}} & \textcolor{white}{\textbf{Scale Param ($\sigma$)}} \\
    \hline
        start & -63.26 & -45.75 & 37.18 & 5.47 \\
        approach & -47.54 & -53.28 & 0.20 & 13.93 \\
        centre & -44.71 & -57.68 & 0.14 & 17.14 \\
        depart & -57.09 & -60.80 & 0.00 & 6.12 \\
        end & -68.13 & -63.22 & 0.00 & 5.04 \\
    \hline\hline
    \end{tabular}}
    \caption{Small-Scale Fading}
    \label{tab:ss_fading}
\end{table}


The scale parameter, indicating signal fluctuation, was found to be highest at the \textit{centre} key point and least at \textit{end} key point. The scale parameter at \textit{centre} and \textit{approach}, as seen in Table \ref{tab:ss_fading}, suggests high fluctuation at these key points, likely due to complex multipath components arising at these locations. 

\subsection{Optimal Deployment Distance of the Observer} \label{sec:res/deployment_distance}


\begin{figure}[htbp]
	\centerline{\includegraphics[width=\textwidth]{Figures/Ch4/timeline/timeline_deployment.png}}

\end{figure}

To assess the optimal deployment distance, 4 candidate distances, 3 $metres$, 5 $metres$, 7 $metres$, and 9 $metres$ were selected. Thus, four paths were marked at those distances from the deployed Observer, each with five key points viz. \textit{start}, \textit{appraoch}, \textit{centre}, \textit{depart}, and \textit{end}. Two cases, \gls{los} and \gls{nlos}, were examined in this experiment. The volunteer pedestrian was instructed to stay stationary at each of those points on each path, one after the other, and measurements were taken in three rounds, each for one minute. 
\\

The procedure for this experiment was the same as the one for identifying the antenna characteristics of the Observer, described in Section \ref{subsec:res/antenna} in this chapter. However, while the antenna characteristics experiment only used 3 metres and 5 metres deployment distances, this experiment extended the measurement to additionally assess 7 metres and 9 metres deployment distances, as well extended the orientation of the Broadcaster to \gls{nlos} case. The measurements date, time, duration, and weather conditions for \gls{los} case were the same as presented in Section \ref{subsec:res/antenna} in this chapter. However, measurements for \gls{nlos} cases were taken on February 10, 2023 between 12:05 and 14:05 hours Irish time. The weather at 12:00, 13:00, 14:00 hours on the day of this experiment is presented below:
\\
\\
\textbf{At 12:00 hours on February 10, 2023}
\begin{itemize}
    \item \textbf{Precipitation (Rain):} $0.0\,\text{mm}$
    \item \textbf{Air Temperature:} $10.7\,^\circ\mathrm{C}$
    \item \textbf{Wet Bulb Temperature:} $9.5\,^\circ\mathrm{C}$
    \item \textbf{Dew Point Temperature:} $8.2\,^\circ\mathrm{C}$
    \item \textbf{Vapour Pressure:} $10.9\,\text{hPa}$
    \item \textbf{Relative Humidity:} $84\,\%$
    \item \textbf{Mean Sea Level Pressure:} $1032.6\,\text{hPa}$
\end{itemize}


\vspace{1em}
\noindent\textbf{At 13:00 hours on February 10, 2023}
\begin{itemize}
    \item \textbf{Precipitation (Rain):} $0.0\,\text{mm}$
    \item \textbf{Air Temperature:} $11.2\,^\circ\mathrm{C}$
    \item \textbf{Wet Bulb Temperature:} $9.8\,^\circ\mathrm{C}$
    \item \textbf{Dew Point Temperature:} $8.4\,^\circ\mathrm{C}$
    \item \textbf{Vapour Pressure:} $11.0\,\text{hPa}$
    \item \textbf{Relative Humidity:} $82\,\%$
    \item \textbf{Mean Sea Level Pressure:} $1032.3\,\text{hPa}$
\end{itemize}


\vspace{1em}
\noindent\textbf{At 14:00 hours on February 10, 2023}
\begin{itemize}
    \item \textbf{Precipitation (Rain):} $0.0\,\text{mm}$
    \item \textbf{Air Temperature:} $11.8\,^\circ\mathrm{C}$
    \item \textbf{Wet Bulb Temperature:} $10.1\,^\circ\mathrm{C}$
    \item \textbf{Dew Point Temperature:} $8.4\,^\circ\mathrm{C}$
    \item \textbf{Vapour Pressure:} $11.0\,\text{hPa}$
    \item \textbf{Relative Humidity:} $80\,\%$
    \item \textbf{Mean Sea Level Pressure:} $1032.2\,\text{hPa}$
\end{itemize}


Before delving into the analysis for assessing optimal deployment distance, the restriction of candidate deployment distances to 9 $metres$ must be clarified. The selected linear pathway was 24 $metres$ in length, irrespective of the distance between the pathway and the Observer. Since the Observer was deployed halfway across the pathway, the farthest locations on each pathways were their respective \textit{start} and \textit{end} key points. The closest distance between the pathway and the Observer was the deployment distance itself, that is the \textit{centre} key point, whereas the farthest distance between the Observer and pathway would be at those farthest points, which can be calculated using Pythagoras theorem, an example of this was presented in Figure \ref{fig:meth/ant_layout} in Chapter \ref{ch:meth}. 
\\

Based on Pythagoras Theorem, the distance between the Observer and \textit{start} key point can be calculated using the formula expressed in Equation \ref{eq:pytha} as the shortest distances between these points on the pathway for a right-angled triangle.

\begin{equation}
    \label{eq:pytha}
    hypotenuse = \sqrt{base^2 + perpendicular^2}
    \myequations{Pythagoras' Theorem}
\end{equation}
\\\\
where:\\
\null \hspace{0.5cm}$\bullet \text{\textit{ hypotenuse is the distance between Observer and `start' key point, }}$\\
\null \hspace{0.5cm}$\bullet\text{\textit{ base is the distance between `start' point and `centre' point, and}}$\\
\null \hspace{0.5cm}$\bullet\text{\textit{ perpendicular is the distance between the Observer and `centre' point on the pathway.}}$
\\

For this experimental case, the value of \textit{base} was 12 irrespective of the deployment distance because the Observer was deployed at the centre of the pathway, whereas \textit{perpendicular} was the deployment distance under test. Using the Equation \ref{eq:pytha}, \textit{hypotenuses} for the pathway can be calculated which will be the farthest distance signals will travel to reach the Observer on the respective pathway. This is presented in Figure \ref{fig:res/depdist/distance}, where, aside from candidate distances for this experiment (3 $m$, 5 $m$, 7 $m$, and 9$m$), 11 $metres$ deployment distance is also selected for scrutiny. 
\\

\begin{figure}[htbp]
	\centerline{\includegraphics[width=\textwidth]{Figures/Ch4/evaluation/deployment distance/changing_angles.png}}
	\caption{Visualisation of Distances Between Observer and \textit{Start} Key-point on Different Pathways}
	\label{fig:res/depdist/distance}
\end{figure}

In the figure, it can be seen that the \textit{hypotenuse} increases with an increase in the deployment distance. However, upon closer inspection, it can be seen that the difference between the \textit{hypotenuse} and respective \textit{perpendicular} is decreasing. The reduction of the difference between them means that the difference in the longest and the shortest distance for the signal to travel is insignificant for the for the measured length of the respective pathway. That is, if the experimental pathway was 70 $metres$ long, then it would have been near the limit of the \gls{ble} range and significant differences would still persist. To visualise this, the ratio between these distances is plotted against respective deployment distances. To facilitate reasoning, more candidate deployment distances are selected. Figure \ref{fig:res/depdist/ratios} presents the ratios against deployment distances, with a second y-axis representing the \textit{hypotenuse} or, the distance between \textit{start} key-point and Observer, or, the longest distance for signal to travel on the pathway.
\\

\begin{figure}[htbp]
	\centerline{\includegraphics[width=\textwidth]{Figures/Ch4/evaluation/deployment distance/ratio_distances.png}}
	\caption{Ratio of Distance from \textit{Start} to Observer and from \textit{Centre} to Observer at Various Deployment Distances}
	\label{fig:res/depdist/ratios}
\end{figure}

As presented in the figure, the ratio approaches 1 as the deployment distance is increased, signifying that the span of signal travel distance will not vary significantly with increase in deployment distance. Such a situation will prevent any significant trends in the resulting \gls{rssi} for inferring pedestrian activities or movement through those measurements. That is, the difference between the shortest distance from Observer to the pathway, viz. the \textit{centre} key point, and the longest distance from Observer to the pathway, viz. \textit{start} and \textit{end} key points reduces as the deployment distance increases. For example, consider two deployment distances, 3-$metre$ and 15-$metre$. The 3 $metres$ deployment distance pathway will comprise \textit{start}, \textit{approach}, \textit{centre}, \textit{depart}, and \textit{end} key points at a distances of $\approx$12.37 $metres$, $\approx$6.70 $metres$, 3 $metres$, $\approx$6.70 $metres$, and $\approx$12.37 $metres$ respectively from the Observer. Whereas, on the 15-$metre$ deployment distance, the same key points will be at a distances of $\approx$19.21 $metres$, $\approx$16.12 $metres$, 15 $metres$, $\approx$16.12 $metres$, and $\approx$19.21 $metres$ respectively. The difference between the shortest distances from the Observer, that is the \textit{centre} key point, and key points \textit{start} or \textit{end} in the case of 3-$metre$ deployment distance is $\approx$9.37 $metres$, whereas, this difference between \textit{centre} key point and \textit{approach} or \textit{depart} key point is $\approx$3.70 $metres$. The same calculation in the case of 15-$metre$ deployment distance yields respective differences of $\approx$4.21 $metres$ and $\approx$1.12 $metres$ between \textit{centre} and \textit{start}/\textit{end} key points and \textit{centre} and \textit{approach}/\textit{depart} key points. This reduction in the distances of different locations on the pathway from the Observer at higher deployment distances would lead to insignificant pattern changes in the resulting \gls{rssi} over the chosen experimental area, which subsequently, will impact detection and identification of pedestrian activity and movement dynamics.
\\

Intuitively, the first approach was to compare the signal strength on each of these paths to assess if the deployment distance results in `\textit{useful}' patterns in the resulting \gls{rssi} from the acquired signals. `\textit{Useful}' pattern here implies that the trend in the \gls{rssi} contains distinguishable movement that would deviate from what could be caused by fluctuations or artefacts of wireless propagation mechanisms such as reflection. This was performed in phases. First, the \gls{rss} values obtained in each of the three rounds of measurement for each location on each path were used to evaluate their respective median signal strength. These medians of all the rounds of measurements at each location on every path were then averaged. This provided a single value representing all of the rounds at each location. The term \textit{averaged median} will be used to refer to this calculated value hereon. Finally, all the averaged medians belonging to a single pathway, or five averaged medians per pathway, were again averaged to calculate a single descriptive value representing the overall signal strength on a path. Through this approach, it was possible to assess not only the overall strength of the signal acquired from a path but, also provide insights into the localised signal strength acquired from a certain location on each of those pathways through the intermediate averaged mean parameter.
\\

Figures \ref{fig:res/depdist/start_los}, \ref{fig:res/depdist/approach_los}, \ref{fig:res/depdist/centre_los}, \ref{fig:res/depdist/depart_los}, and \ref{fig:res/depdist/end_los} depict the captured \gls{rss} values at \textit{start}, \textit{approach}, \textit{centre}, \textit{depart}, and \textit{end} locations respectively on each of the pathways with their own median values, and the mean of those median values in the case of \gls{los}. In these figures, and the subsequent figure depicting similar parameters in this format, the x-axis, \textit{Instance}, represents the sequence of the acquisition of advertisements which resets to zero at the start of each new deployment distance region.
\\

\begin{figure}[htbp]
	\centerline{\includegraphics[width=\textwidth]{Figures/Ch4/evaluation/deployment distance/los/los - start point all pathways.jpg}}
	\caption{Comparison of \gls{rssi} and the Median Values for Three rounds of 1-minute at \textit{Start} Location on All Pathways (\gls{los})}
	\label{fig:res/depdist/start_los}
\end{figure}
\begin{figure}[htbp]
	\centerline{\includegraphics[width=\textwidth]{Figures/Ch4/evaluation/deployment distance/los/los - approach point all pathways.jpg}}
	\caption{Comparison of \gls{rssi} and the Median Values for Three rounds of 1-minute at \textit{Approach} Location on All Pathways (\gls{los})}
	\label{fig:res/depdist/approach_los}
\end{figure}

\begin{figure}[htbp]
	\centerline{\includegraphics[width=\textwidth]{Figures/Ch4/evaluation/deployment distance/los/los - centre point all pathways.jpg}}
	\caption{Comparison of \gls{rssi} and the Median Values for Three rounds of 1-minute at \textit{Centre} Location on All Pathways (\gls{los})}
	\label{fig:res/depdist/centre_los}
\end{figure}

\begin{figure}[htbp]
	\centerline{\includegraphics[width=\textwidth]{Figures/Ch4/evaluation/deployment distance/los/los - depart point all pathways.jpg}}
	\caption{Comparison of \gls{rssi} and the Median Values for Three rounds of 1-minute at \textit{Depart} Location on All Pathways (\gls{los})}
	\label{fig:res/depdist/depart_los}
\end{figure}
\begin{figure}[htbp]
	\centerline{\includegraphics[width=\textwidth]{Figures/Ch4/evaluation/deployment distance/los/los - end point all pathways.jpg}}
	\caption{Comparison of \gls{rssi} and the Median Values for Three rounds of 1-minute at \textit{End} Location on All Pathways (\gls{los})}
	\label{fig:res/depdist/end_los}
\end{figure}
\FloatBarrier

Similarly, Figures \ref{fig:res/depdist/start_nlos}, \ref{fig:res/depdist/approach_nlos}, \ref{fig:res/depdist/centre_nlos}, \ref{fig:res/depdist/depart_nlos}, and \ref{fig:res/depdist/end_nlos} depicts the captured \gls{rss} values at \textit{start}, \textit{approach}, \textit{centre}, \textit{depart}, and \textit{end} locations respectively on each of the pathways with their own median values, and the mean of those median values in the case of \gls{nlos}.
\\

\begin{figure}[htbp]
	\centerline{\includegraphics[width=\textwidth]{Figures/Ch4/evaluation/deployment distance/nlos/nlos - start point all pathways.jpg}}
	\caption{Comparison of \gls{rssi} and the Median Values for Three rounds of 1-minute at \textit{Start} Location on All Pathways (\gls{nlos})}
	\label{fig:res/depdist/start_nlos}
\end{figure}
\begin{figure}[htbp]
	\centerline{\includegraphics[width=\textwidth]{Figures/Ch4/evaluation/deployment distance/nlos/nlos - approach point all pathways.jpg}}
	\caption{Comparison of \gls{rssi} and the Median Values for Three rounds of 1-minute at \textit{Approach} Location on All Pathways (\gls{nlos})}
	\label{fig:res/depdist/approach_nlos}
\end{figure}

\begin{figure}[htbp]
	\centerline{\includegraphics[width=\textwidth]{Figures/Ch4/evaluation/deployment distance/nlos/nlos - centre point all pathways.jpg}}
	\caption{Comparison of \gls{rssi} and the Median Values for Three rounds of 1-minute at \textit{Centre} Location on All Pathways (\gls{nlos})}
	\label{fig:res/depdist/centre_nlos}
\end{figure}

\begin{figure}[htbp]
	\centerline{\includegraphics[width=\textwidth]{Figures/Ch4/evaluation/deployment distance/nlos/nlos - depart point all pathways.jpg}}
	\caption{Comparison of \gls{rssi} and the Median Values for Three rounds of 1-minute at \textit{Depart} Location on All Pathways (\gls{nlos})}
	\label{fig:res/depdist/depart_nlos}
\end{figure}
\begin{figure}[htbp]
	\centerline{\includegraphics[width=\textwidth]{Figures/Ch4/evaluation/deployment distance/nlos/nlos - end point all pathways.jpg}}
	\caption{Comparison of \gls{rssi} and the Median Values for Three rounds of 1-minute at \textit{End} Location on All Pathways (\gls{nlos})}
	\label{fig:res/depdist/end_nlos}
\end{figure}
\FloatBarrier

To detect the presence of any statistically significant differences in the measurements, the \gls{anova} technique, followed by Tukey's \gls{hsd} post-hoc analysis was applied to the data. This technique was presented in subsection \ref{subsec:meth/analysis/anova} of Chapter \ref{ch:meth}. To do this, the data from all the rounds for each location at each pathway were combined and used. The \gls{los} and \gls{nlos} cases were separately analysed. While they could be combined to gain an overall understanding, they were analysed separately to see if any nuances could be discovered in the process. The result obtained from \gls{anova} for the \gls{los} case is presented in Figure \ref{fig:res/depdist/anova_los}. The upper whisker and the lower whisker in the box plot depict the minimum and maximum \gls{rssi} respectively whereas the red line depicts the median value. The upper bound of the box represents the 75th percentile and the bottom represents the 25th percentile values. If the notches, which are at the 5\% significance level, do not overlap with the notches of other groups, then the two medians are significantly different.
\\

\begin{figure}[htbp]
	\centerline{\includegraphics[width=\textwidth]{Figures/Ch4/evaluation/deployment distance/los/anova.jpg}}
	\caption{Results of \gls{anova} on \gls{los} Measurements}
	\label{fig:res/depdist/anova_los}
\end{figure}

\begin{wrapfigure}{l}{0.45\textwidth}
\vspace{-7mm}
    \begin{tcolorbox}[
        colframe=Teal, % Border color
        colback=MintCream, % Background color
        coltitle=white, % Title text color
        title=\footnotesize{Quick Recap}, % Title text
        fonttitle=\bfseries, % Title font style
        sharp corners, % Sharp corners for the main box
        enhanced,
        attach boxed title to top left={yshift=-2mm, xshift=2mm}, % Positioning the title
        boxed title style={colback=MidnightGreen, colframe=Teal, rounded corners=west, boxrule=0pt, top=2mm, bottom=2mm, right=2mm}, % Title box style
        width=\linewidth, % Width of the box
        boxrule=0pt, % No border for the main box
        drop shadow, % Shadow effect
        rounded corners, % Rounded corners for the main box
    ]
        \scriptsize{\textit{\acrfull{ci}}: \gls{ci} in \gls{anova} provides a range of values that likely contain population parameters such as a mean difference between the groups with a specified level of confidence, 95\% in this case.}
    \end{tcolorbox}    
    \vspace{-7mm}
\end{wrapfigure}

The results of \gls{anova} for the \gls{los} case were fed into Tukey's \gls{hsd} test. The outcome is presented in Table \ref{tab:res/depdist/tukeys_los}. The first two columns in the table represent a comparison of the two groups. The lower and upper bounds of \gls{ci} are in rows three and five, while the fourth column represents the difference in means of the two groups being compared. Finally, the sixth column represents the $p$-value. The \gls{ci} represents the range of values where the difference of means is likely to fall. If the values of the lower and upper bound are such that they contain $0$, that is if the intervals have opposite signs, it signifies that there is no statistically significant difference between the two groups. The $p$-value is the measure of the probability of obtaining extreme results on the assumption of the null hypothesis being true. In this case, the null hypothesis dictates that there is no significant difference between the two. The $p$-value is always compared against some threshold for significance, $\alpha$, which in this case is $0.05$, as mentioned in Section \ref{subsec:meth/analysis/anova} in Chapter \ref{ch:meth}. Therefore, if the $p$-value is greater than 0.05, then, the null hypothesis cannot be rejected. The outcome of Tukey's \gls{hsd} test for the \gls{los} case is also visualised in Figure \ref{fig:res/depdist/tukeys_los}. The error bars in the figure represent the \gls{ci} with a marker at the difference of means. The dotted horizontal line represents the zero difference. The error bars, representing the \gls{ci} passing over the zero difference line signifies that there is no statistically significant difference between the particular pair or group.
\\

\begin{table}[!h]
    \centering
    \arrayrulecolor{DarkOliveGreen} % Set the color of table rules    
    \begin{threeparttable}
    \resizebox{\textwidth}{!}{\begin{tabular}{>{\columncolor{MintCream}}c  >{\columncolor{MintCream}}c >{\columncolor{MintCream}}c >{\columncolor{MintCream}}c  >{\columncolor{MintCream}}c >{\columncolor{MintCream}}c}
        \hline \rowcolor{MidnightGreen}
         \textcolor{white}{\textbf{Group 1}} & \textcolor{white}{\textbf{Group 2}} & \textcolor{white}{\textbf{\gls{ci}$^1$ Lower Bound}} & \textcolor{white}{\textbf{Difference in Means}} & \textcolor{white}{\textbf{\gls{ci} Upper Bound}} & \textcolor{white}{\textbf{$p$-value}}\\
        \hline
        3m & 5m & -0.6065 & 0.6817 & 1.9700 & 0.5249 \\
        \hline
        3m & 7m & 2.6236 & 3.9119 & 5.2002 & 0 \\
        \hline
        3m & 9m & 0.3530 & 1.6413 & 2.9296 & 0.0059 \\
        \hline
        5m & 7m & 1.9419 & 3.2302 & 4.5185 & 0 \\
        \hline
        5m & 9m & -0.3288 & 0.9595 & 2.2478 & 0.2223 \\
        \hline
        7m & 9m & -3.5589 & -2.2706 & -0.9823 & 0 \\
        \hline\hline
    \end{tabular}}
    \caption{Outcome of Tukey's \gls{hsd} (\gls{los})}
    \label{tab:res/depdist/tukeys_los}
    \begin{tablenotes}
    \item[1] \footnotesize{\gls{ci} is a range of values between which the difference in means is likely to be found.}
    \end{tablenotes}
    \end{threeparttable}
    
\end{table} 

\begin{figure}[htbp]
	\centerline{\includegraphics[width=\textwidth]{Figures/Ch4/evaluation/deployment distance/los/tukeys_los.jpg}}
	\caption{Comparison of \gls{ci}, Median, and Zero Difference for Each Group of Deployment Distances (\gls{los})}
	\label{fig:res/depdist/tukeys_los}
\end{figure}

The following findings, as presented in Table \ref{tab:res/depdist/tukeys_los}, were obtained from this analysis:
\begin{enumerate}
    \item 3 $metres$ vs 5 $metres$
        \begin{itemize}
            \item Lower bound of \gls{ci}: -0.6065. Upper bound of \gls{ci}: 1.97. Since this range contained $0$, it indicated that there is no statistically significant difference in the means.
            \item The $p$-value is 0.5249, which was found to be greater than $\alpha$. Therefore, the null hypothesis was not rejected.
            \item \textbf{Verdict:} Based on the presence of 0 in the \gls{ci} and high $p$-value, the difference in the means was not a true difference but likely to be a random variation. \textit{No significant difference}.
        \end{itemize}

    \item 3 $metres$ vs 7 $metres$
        \begin{itemize}
            \item Lower bound of \gls{ci}: 2.6236. Upper bound of \gls{ci}: 5.2002. Since this range did not contain $0$, it indicated that there was a statistically significant difference in the means.
            \item The $p$-value was 0, which was less than $\alpha$. Therefore, the null hypothesis was rejected.
            \item \textbf{Verdict:} Significant difference. 3 $metres$ was better than 7 $metres$.
        \end{itemize}

    \item 3 $metres$ vs 9 $metres$
        \begin{itemize}
            \item Lower bound of \gls{ci}: 0.3530. Upper bound of \gls{ci}: 2.9296. Since this range did not contain $0$, it indicated that there was a statistically significant difference in the means.
            \item The $p$-value was 0.0059, which was less than $\alpha$. Therefore, the null hypothesis was rejected.
            \item \textbf{Verdict:} Significant difference. 3 $metres$ was better than 9 $metres$.
        \end{itemize}

    \item 5 $metres$ vs 7 $metres$
        \begin{itemize}
            \item Lower bound of \gls{ci}: 1.9419. Upper bound of \gls{ci}: 4.5185. Since this range did not contain $0$, it indicated that there was a statistically significant difference in the means.
            \item The $p$-value was 0, which was less than $\alpha$. Therefore, the null hypothesis was rejected.
            \item \textbf{Verdict:} Significant difference. 5 $metres$ was better than 7 $metres$.
        \end{itemize}

    \item 5 $metres$ vs 9 $metres$
        \begin{itemize}
            \item Lower bound of \gls{ci}: -0.3288. Upper bound of \gls{ci}: 2.2478. Since this range contained $0$, it indicated that there was no statistically significant difference in the means.
            \item The $p$-value was 0.2223, which was greater than $\alpha$. Therefore, the null hypothesis was not rejected.
            \item \textbf{Verdict:} There was no significant difference.
        \end{itemize}

    \item 7 $metres$ vs 9 $metres$
        \begin{itemize}
            \item Lower bound of \gls{ci}: -3.5589. Upper bound of \gls{ci}: -0.9823. Since this range did not contain $0$, it indicated that there was a statistically significant difference in the means.
            \item The $p$-value was 0, which was less than $\alpha$. Therefore, the null hypothesis was rejected.
            \item \textbf{Verdict:} Significant difference. 7 $metres$ was better than 9 $metres$.
        \end{itemize}
\end{enumerate}

Based on these findings, the deployment distance of 3 $metres$ consistently produced significantly higher mean results compared to 7 $metres$ and 9 $metres$. However, the difference between 3 $metres$ and 5 $metres$ was not statistically significant. The deployment distance of 5 $metres$ produced significantly higher results compared to 7 $metres$ but did not differ significantly from 9 $metres$. Finally, the results for 7 $metres$ were significantly higher than those for 9 $metres$.
\\

Given these results, the most effective deployment distances appeared to be 3 $metres$ and 5 $metres$, with no significant difference between them, suggesting that either could be optimal depending on the context or specific goals of the deployment. It is noteworthy that the distance in this regard is horizontal distance, and does not take into account the height at which it is deployed.
\\

Results of \gls{anova} for the \gls{nlos} case are presented in Figure \ref{fig:res/depdist/anova_nlos}. Here, some outlier values can also be seen as red-coloured $+$ symbols. The \gls{anova} result fed into Tukey's \gls{hsd} produced the output presented in Table \ref{tab:res/depdist/tukeys_nlos} and in Figure \ref{fig:res/depdist/tukeys_nlos}. The summary derived from these results is as follows:

\begin{enumerate}
    \item 3 $metres$ vs 5 $metres$
        \begin{itemize}
            \item Lower bound of \gls{ci}: -1.5304. Upper bound of \gls{ci}: 1.3804. Since this range contained $0$, it indicated that there was no statistically significant difference in the means.
            \item The $p$-value was 0.9992, which was greater than $\alpha$. Therefore, the null hypothesis was not rejected.
            \item \textbf{Verdict:} \textit{No significant difference}.
        \end{itemize}

    \item 3 $metres$ vs 7 $metres$
        \begin{itemize}
            \item Lower bound of \gls{ci}: 0.1637. Upper bound of \gls{ci}: 3.0746. Since this range did not contain $0$, it indicated that there was a statistically significant difference in the means.
            \item The $p$-value was 0.022, which was less than $\alpha$. Therefore, the null hypothesis was rejected.
            \item \textbf{Verdict:} \textit{Significant difference}. 3 $metres$ was better than 7 $metres$.
        \end{itemize}

    \item 3 $metres$ vs 9 $metres$
        \begin{itemize}
            \item Lower bound of \gls{ci}: -0.5514. Upper bound of \gls{ci}: 2.3594. The range contained $0$, indicating no statistically significant difference in the means.
            \item The $p$-value was 0.3810, which was greater than $\alpha$. Therefore, the null hypothesis was not rejected.
            \item \textbf{Verdict:} \textit{No significant difference}. 
        \end{itemize}

    \item 5 $metres$ vs 7 $metres$
        \begin{itemize}
            \item Lower bound of \gls{ci}: 0.2387. Upper bound of \gls{ci}: 3.1495. Since this range did not contain $0$, it indicated that there was a statistically significant difference in the means.
            \item The $p$-value was 0.0148, which was less than $\alpha$. Therefore, the null hypothesis was rejected.
            \item \textbf{Verdict:} \textit{Significant difference}. 5 $metres$ was better than 7 $metres$.
        \end{itemize}

    \item 5 $metres$ vs 9 $metres$
        \begin{itemize}
            \item Lower bound of \gls{ci}: -0.4764. Upper bound of \gls{ci}: 2.4344. Since this range contained $0$, it indicated that there was no statistically significant difference in the means.
            \item The $p$-value was 0.3090, which was greater than $\alpha$. Therefore, the null hypothesis was not rejected.
            \item \textbf{Verdict:} \textit{No significant difference}.
        \end{itemize}

    \item 7 $metres$ vs 9 $metres$
        \begin{itemize}
            \item Lower bound of \gls{ci}: -2.1705. Upper bound of \gls{ci}: 0.7403. The range contained $0$, indicating no statistically significant difference in the means.
            \item The $p$-value was 0.5870, which was greater than $\alpha$. Therefore, the null hypothesis was not rejected.
            \item \textbf{Verdict:} \textit{No significant difference}.
        \end{itemize}
\end{enumerate}



\begin{figure}[htbp]
	\centerline{\includegraphics[width=\textwidth]{Figures/Ch4/evaluation/deployment distance/nlos/anova.jpg}}
	\caption{Results of \gls{anova} on \gls{nlos} Measurements}
	\label{fig:res/depdist/anova_nlos}
\end{figure}



    \begin{table}[!h]
    \centering
    \arrayrulecolor{DarkOliveGreen} % Set the color of table rules    
    \begin{threeparttable}
    \resizebox{\textwidth}{!}{\begin{tabular}{>{\columncolor{MintCream}}c  >{\columncolor{MintCream}}c >{\columncolor{MintCream}}c >{\columncolor{MintCream}}c  >{\columncolor{MintCream}}c >{\columncolor{MintCream}}c}
        \hline \rowcolor{MidnightGreen}
         \textcolor{white}{\textbf{Group 1}} & \textcolor{white}{\textbf{Group 2}} & \textcolor{white}{\textbf{\gls{ci} Lower Bound}}& \textcolor{white}{\textbf{Difference in Means}} & \textcolor{white}{\textbf{\gls{ci} Upper Bound}} & \textcolor{white}{\textbf{$p$-value}}\\
        \hline
        3m & 5m & -1.5304& -0.0750& 1.3804& 0.9992\\
        \hline
        3m & 7m & 0.1637& 1.6191& 3.0746& 0.0222\\
        \hline
        3m & 9m & -0.5514& 0.9040& 2.3594& 0.3810\\
        \hline
        5m & 7m & 0.2387& 1.6941& 3.1495& 0.0148\\
        \hline
        5m & 9m & -0.4764& 0.9790& 2.4344& 0.3090\\
        \hline
        7m & 9m & -2.1705& -0.7151& 0.7403& 0.5870\\
        \hline\hline
    \end{tabular}}
    \caption{Outcome of Tukey's \gls{hsd} (\gls{nlos})}
    \label{tab:res/depdist/tukeys_nlos}
    
    \end{threeparttable}
    
\end{table} 

\begin{figure}[htbp]
	\centerline{\includegraphics[width=\textwidth]{Figures/Ch4/evaluation/deployment distance/nlos/tukeys_nlos.jpg}}
	\caption{Comparison of \gls{ci}, Median, and Zero Difference for Each Group of Deployment Distances (\gls{nlos})}
	\label{fig:res/depdist/tukeys_nlos}
\end{figure}

Through the results of \gls{anova} and Tukey's \gls{hsd} tests, both 3 $metres$ and 5 $metres$ from the closest approach of a pedestrian appeared to be more suitable deployment distances than the deployment distances of 7 $metres$ and 9 $metres$. While the results obtained thus far were satisfactory to conclude that those deployment distances are superior, another statistical test was conducted to compare the two candidates, 3 $metres$ and 5 $metres$, that were found to be more adequate than the others in the previous tests. To do this test, \gls{mda}, which is described in the Section \ref{subsec:meth/analysis/mda} in Chapter \ref{ch:meth}, values were calculated for each of the locations on all pathways, combining measurements from all rounds.

\begin{wrapfigure}{l}{0.45\textwidth}
    \begin{tcolorbox}[
        colframe=Indigo, % Border color
        colback=Lavender, % Background color
        coltitle=white, % Title text colour
        title=\footnotesize{Information}, % Title text
        fonttitle=\bfseries, % Title font style
        sharp corners, % Sharp corners for the main box
        enhanced,
        attach boxed title to top left={yshift=-2mm, xshift=2mm}, % Positioning the title
        boxed title style={colback=Indigo, colframe=Indigo, rounded corners=west, boxrule=0pt, top=2mm, bottom=2mm, right=2mm}, % Title box style
        width=\linewidth, % Width of the box
        boxrule=0pt, % No border for the main box
        drop shadow, % Shadow effect
        rounded corners, % Rounded corners for the main box
    ]
        \scriptsize{The \textit{double hump} pattern is not investigated in the \gls{nlos} case because in that case, the \gls{rss} will always be lower at the \textit{centre} point due to full occlusion by the body of the volunteer pedestrian, as seen in Figure \ref{fig:meth/depdist_occ_vs_partial}.}
    \end{tcolorbox}  
    \vspace{-7mm}
\end{wrapfigure}

Error bars were plotted to depict the deviations obtained from the \gls{mda} values and a marker was placed on the error bars to denote the median value. Lines were connected between the subsequent locations on the same deployment distances to represent a hypothetical progression trend in the \gls{rss} values. This is presented in Figure \ref{fig:res/depdist/fluctuations_los} for the \gls{los} case and in Figure \ref{fig:res/depdist/fluctuations_nlos} for the \gls{nlos} case. While the outcome did not provide any evidence of superior signal strength between 3 $metres$ and 5 $metres$ deployment distance, it highlighted another important fact. In the deployment distances of 5 $metres$ and 9 $metres$ for the \gls{los} case, a \textit{double hump} pattern similar to the one obtained in the anechoic chamber was observed. Now, it is noteworthy that the distance between the Observer and the Broadcaster in the anechoic chamber remained constant, whereas, in this experiment, the distances between the two devices increased in either direction along the pathway from the \textit{centre} point. This presence of a double hump pattern despite varying distances on the linear pathway means that there is a likelihood of obtaining this feature during a walk by a pedestrian. Since this is a salient feature in the \gls{rssi} pattern when it appears, it can be used to assert information regarding the movement of the pedestrian as well. This will be revisited in Section \ref{sec:res/deployment_distance} in this Chapter. Finally, the average of all the medians across the entire pathway was calculated and compared with other candidate deployment distances. The horizontal lines in Figure \ref{fig:res/depdist/average_median_los} and Figure \ref{fig:res/depdist/average_median_nlos} depict the average medians for entire pathways for the cases of \gls{los} and \gls{nlos} respectively. 
\\
\begin{figure}[htbp]
	\centerline{\includegraphics[width=\textwidth]{Figures/Ch4/evaluation/deployment distance/los/los - mad fluctuation in rssi from median (all bouts combined).jpg}}
	\caption{Comparison between Deviations in the \gls{rss} Values for All Deployment Distances at Each Key Point on the Pathway (\gls{los})}
	\label{fig:res/depdist/fluctuations_los}
\end{figure}

\begin{figure}[htbp]
	\centerline{\includegraphics[width=\textwidth]{Figures/Ch4/evaluation/deployment distance/nlos/nlos - mad fluctuation in rssi from median (all bouts combined).jpg}}
	\caption{Comparison between Deviations in the \gls{rss} Values for All Deployment Distances at Each Key Point on the Pathway (\gls{nlos})}
	\label{fig:res/depdist/fluctuations_nlos}
\end{figure}

\begin{figure}[htbp]
	\centerline{\includegraphics[width=\textwidth]{Figures/Ch4/evaluation/deployment distance/los/los - average of mean of medians.jpg}}
	\caption{Average Median Across All Key Points on Each Pathway (\gls{los})}
	\label{fig:res/depdist/average_median_los}
\end{figure}

\begin{figure}[htbp]
	\centerline{\includegraphics[width=\textwidth]{Figures/Ch4/evaluation/deployment distance/nlos/nlos - average of mean of medians.jpg}}
	\caption{Average Median Across All Key Points on Each Pathway (\gls{nlos})}
	\label{fig:res/depdist/average_median_nlos}
\end{figure}

Finally, an assessment of fluctuations at each pathway was performed. This analysis of fluctuations provided an opportunity to identify the pathway that produced \gls{rssi} patterns that are more accurate representation of pedestrian's walk, that is they are less influenced by environmental topology. This analysis was performed by assessing the \gls{ss} Fading through Rician distribution fitting, as mentioned in Section \ref{subsec:meth/evalexp/fading} in Chapter \ref{ch:meth}. The explanation behind the analysis was already described in Section \ref{sec:res/fading} of this chapter, and the result for the deployment distance of 3 $metres$ was also presented in the same section in Figure \ref{fig:res/fading/ss}. Only \gls{los} cases were considered for this analysis, as \gls{ss} fading assesses the dominance of \gls{los} component which will clearly be diminished in the \gls{nlos} case. Thus, the purpose of this investigation was to compare the dominance of \gls{los} components across all pathways. Figures \ref{fig:res/depdist/fading_5m}, \ref{fig:res/depdist/fading_7m}, and \ref{fig:res/depdist/fading_9m} depicts \gls{ss} fading at 5 $metres$, 7 $metres$, and 9 $metres$ deployment distance respectively. Table \ref{tab:depdist_fading} summarises the results obtained from Rician fitting for all deployment distances.

\begin{figure}[htbp]
	\centerline{\includegraphics[width=\textwidth]{Figures/Ch4/evaluation/deployment distance/small_scale_fading_5m.png}}
	\caption{Rician Distribution Fitting to \gls{rssi} at Each Key-points on 5 $m$ Pathway (\gls{los})}
	\label{fig:res/depdist/fading_5m}
\end{figure}

\begin{figure}[htbp]
	\centerline{\includegraphics[width=\textwidth]{Figures/Ch4/evaluation/deployment distance/small_scale_fading_7m.png}}
	\caption{Rician Distribution Fitting to \gls{rssi} at Each Key-points on 7 $m$ Pathway (\gls{los})}
	\label{fig:res/depdist/fading_7m}
\end{figure}

\begin{figure}[htbp]
	\centerline{\includegraphics[width=\textwidth]{Figures/Ch4/evaluation/deployment distance/small_scale_fading_9m.png}}
	\caption{Rician Distribution Fitting to \gls{rssi} at Each Key-points on 9 $m$ Pathway (\gls{los})}
	\label{fig:res/depdist/fading_9m}
\end{figure}


\begin{table}[htbp]
    \centering
    
    \resizebox{\textwidth}{!}{
    \begin{tabular}{>{\columncolor{MintCream}}c >{\columncolor{MintCream}}c >{\columncolor{MintCream}}c >{\columncolor{MintCream}}c >{\columncolor{MintCream}}c >{\columncolor{MintCream}}c}
    \hline
        \rowcolor{MidnightGreen}
        \textcolor{white}{\textbf{Deployment Distance}} & \textcolor{white}{\textbf{Location}} & \textcolor{white}{\textbf{Mean RSSI (dBm)}} & \textcolor{white}{\textbf{Fitted RSSI (dBm)}} & \textcolor{white}{\textbf{Shape Param ($K$)}} & \textcolor{white}{\textbf{Scale Param ($\sigma$)}} \\
    \hline    
        & start & -63.26 & -45.75 & 37.18 & 5.47 \\
        & approach & -47.54 & -53.28 & 0.20 & 13.93 \\
        & centre & -44.71 & -57.68 & 0.14 & 17.14 \\
        & depart & -57.09 & -60.80 & 0.00 & 6.12 \\
        \multirow{-5}{*}{\rotatebox[origin=c]{90}{3 $metres$}} & end & -68.13 & -63.22 & 0.00 & 5.04 \\
    \hline
        & start & -63.26 & -45.75 & 117.84 & 5.45 \\
        & approach & -47.54 & -53.28 & 86.35 & 3.13 \\
        & centre & -44.71 & -57.68 & 7.26 & 5.71 \\
        & depart & -57.09 & -60.80 & 0.00 & 8.26 \\
        \multirow{-5}{*}{\rotatebox[origin=c]{90}{5 $metres$}} & end & -68.13 & -63.22 & 0.15 & 17.72 \\
    \hline
        & start & -66.99 & -49.39 & 0.00 & 5.17 \\
        & approach & -56.46 & -56.91 & 0.00 & 3.65 \\
        & centre & -54.64 & -61.31 & 0.00 & 16.77 \\
        & depart & -57.55 & -64.44 & 0.05 & 17.24 \\
        \multirow{-5}{*}{\rotatebox[origin=c]{90}{7 $metres$}} & end & -63.27 & -66.86 & 0.21 & 15.87 \\
    \hline
        & start & -66.64 & -46.42 & 100.05 & 7.38 \\
        & approach & -50.38 & -53.94 & 0.50 & 14.10 \\
        & centre & -55.01 & -58.35 & 0.00 & 5.04 \\
        & depart & -48.32 & -61.47 & 122.74 & 4.08 \\
        \multirow{-5}{*}{\rotatebox[origin=c]{90}{9 $metres$}} & end & -63.72 & -63.89 & 102.79 & 6.33 \\
    \hline\hline
    \end{tabular}}
    \caption{\gls{ls} Fading}
    \label{tab:depdist_fading}
\end{table}

\vspace{10pt}

    \begin{tcolorbox}[
        colframe=white, % Border color
        colback=Ivory, % Background color
        coltitle=white, % Title text color
        title=Section Summary, % Title text
        fonttitle=\bfseries, % Title font style
        sharp corners, % Sharp corners for the main box
        enhanced,
        attach boxed title to top left={yshift=-2mm, xshift=2mm}, % Positioning the title
        boxed title style={colback=DarkSlateGray, colframe=SlateBlue, rounded corners=west, boxrule=0pt, top=2mm, bottom=2mm, right=2mm}, % Title box style
        width=\linewidth, % Width of the box
        boxrule=0pt, % No border for the main box
        drop shadow, % Shadow effect
        rounded corners, % Rounded corners for the main box
    ]
        The outcome of this experiment highlights both 3 $metres$ and 5 $metres$ as optimal candidates for the deployment distance of the Observer for the selected hardware components for this PhD research. While this set of results could be intuitive as it indicates that the strength of \gls{ble} signals dissipates with distance, this experiment proved to be useful. Despite that intuition, it can be seen that the \gls{rssi} of the signal emanating from the distance of 9 $metres$ are similar to the 5 $metres$ and 3 $metres$ deployment distances in some locations on the pathways. Surprisingly, the deployment distance of 9 $metres$ has provided better results in comparison to the deployment distance of 7 $metres$. 
% \\
% \\
% The work presented in this section is part of the publication titled, "Indication of Pedestrian's Travel Direction Through Bluetooth Low Energy Signals Perceived by a Single Observer Device" \citep{Parmar2023}.
    \end{tcolorbox}
    
\subsection{Detecting Occlusion between the Broadcaster and the Observer}\label{subsec:res/occlusion}


\begin{figure}[htbp]
	\centerline{\includegraphics[width=\textwidth]{Figures/Ch4/timeline/timeline_occlusion.png}}

\end{figure}

The results of the previous experiments show that the behaviour of the \gls{ble} based platform is comparable in an outdoor setting with the noiseless anechoic chamber. However, to assess the feasibility of the technology and the platform, an investigation to understand whether occlusion can be detected between the \gls{ble} Broadcaster and the Observer was conducted. As stated in the Section \ref{sec:meth/experiments} Methodology chapter \ref{ch:meth}, three experimental setups were defined to test this case. 
\\

This experiment was devised to understand the effect of body occlusion on the \gls{rssi} of the \gls{ble} device carried by a pedestrian on a linear pathway. To do this, the Observer was deployed at a distance of 3 $metres$ and \gls{rssi} was collected from a Broadcaster with an advertisement rate of 2Hz. The experiment was divided into the following three sub-experiments:

\begin{enumerate}
    \item \textit{Sub-experiment 1: \gls{rss} values are recorded from a stationary pedestrian at each waypoint from A to E}
    
    \item \textit{Sub-experiment 2: \gls{rss} values are recorded from a stationary pedestrian at each waypoint from P to S)}
        
    \item \textit{Sub-experiment 3: Continuous \gls{rssi} collection while pedestrian traverses the path}
    \begin{enumerate}
        \item Pedestrian walking from \textit{start} to \textit{end} with Broadcaster in \gls{los} of the Observer (no occlusion).
        \item Pedestrian walking from \textit{end} to \textit{start} with Broadcaster in \gls{los} of the Observer (no occlusion).
        \item Pedestrian walking from \textit{start} to \textit{end} with Broadcaster in \gls{nlos} of the Observer (occlusion).
        \item Pedestrian walking from \textit{end} to \textit{start} with Broadcaster in \gls{nlos} of the Observer (occlusion).
\end{enumerate}
\end{enumerate}

The first two sub-experiments allowed the comparison of occlusion and non-occlusion cases when the movement of pedestrians is not affecting the patterns on the resulting \gls{rssi} as captured by the Observer. This could be beneficial in the future to calibrate models that can be used to automate the identification of activities and movement dynamics, develop a general understanding of the behaviour of \gls{ble}, and identify useful patterns that may help in asserting the location of the pedestrian, based solely on \gls{rssi} patterns.
\\

Measurements for the first sub-experiment were acquired on March 23, 2022 between 14:16 and 14:27 hours Irish time. These measurements were taken using a single volunteer. Measurements for the second sub-experiment followed the first sub-experiment, and hence were taken on the same day between 14:29 and 14:38 hours. The weather at 14:00 hours on the day of the experiment is presented below:
\\

\noindent\textbf{At 12:00 hours on February 10, 2023}
\begin{itemize}
    \item \textbf{Precipitation (Rain):} $0.0\,\text{mm}$
    \item \textbf{Air Temperature:} $16.8\,^\circ\mathrm{C}$
    \item \textbf{Wet Bulb Temperature:} $11.5\,^\circ\mathrm{C}$
    \item \textbf{Dew Point Temperature:} $5.9\,^\circ\mathrm{C}$
    \item \textbf{Vapour Pressure:} $9.3\,\text{hPa}$
    \item \textbf{Relative Humidity:} $48\,\%$
    \item \textbf{Mean Sea Level Pressure:} $1027.8\,\text{hPa}$
\end{itemize}

Measurements for the third and final sub-experiment started after the second sub-experiment on the same day between 14:41 and 15:07 hours. The weather information at 15:00 hours on the data is as follows:

\vspace{1em}
\noindent\textbf{At 15:00 hours on February 10, 2023}
\begin{itemize}
    \item \textbf{Precipitation (Rain):} $0.0\,\text{mm}$
    \item \textbf{Air Temperature:} $17.4\,^\circ\mathrm{C}$
    \item \textbf{Wet Bulb Temperature:} $11.8\,^\circ\mathrm{C}$
    \item \textbf{Dew Point Temperature:} $6.0\,^\circ\mathrm{C}$
    \item \textbf{Vapour Pressure:} $9.3\,\text{hPa}$
    \item \textbf{Relative Humidity:} $47\,\%$
    \item \textbf{Mean Sea Level Pressure:} $1027.7\,\text{hPa}$
\end{itemize}

\subsubsection{Sub-experiment 1}

% There is an underlying default advertisement rate at which any \gls{ble} device advertises. If the signals are intercepted without any dropping in the advertisement, they will correspond to regularly-sampled time-series data. However, the Observer may not always be able to reproduce the received advertisements as regularly-sampled signals. Advertisements may be lost due to several phenomena such as obstruction from spatial topology, packet collision, physical characteristics of the journey of signals arising due to multi-path propagation, and latency of the software stack on the Observer. However, for all but the most aggressive advertisement rates, the inability to accurately reproduce regularly-sampled data would be insignificant. This analogy of \gls{ble} advertisements with sparsely populated, regularly-sampled time series signals enables us to understand underlying patterns that could lead us to identify characteristics of space, and behaviour of pedestrians.
% \\

% Now, since \gls{rss} values can be represented as regularly-sampled time series, we observe temporal changes in the signals and assess variability in those values. This allows us to understand how the signals vary and spread comparatively, with and without the body occlusion.

The collected \gls{rssi} values at each location for a period of 30 seconds were plotted with \gls{rssi} on the y-axis and the order of their collection represented on the x-axis. This was performed to visually assess the strength of the signal at each of those key points for both \gls{los} and \gls{nlos} cases. This is depicted in Figure \ref{fig:res/occ/point} where the red error bars and markers depict the fluctuations in the \gls{rss} values and the median \gls{rssi} respectively at each of those locations. Likewise, the blue error bars and markers depict those parameters for the \gls{nlos} case.
\\

\begin{figure}[htbp]
	\centerline{\includegraphics[width=\textwidth]{Figures/Ch4/occlusion/occlusion_median_path.png}}
	\caption{\gls{ble} \gls{rssi} at Key-Points \textit{start} to \textit{end} (\gls{los} and \gls{nlos}) for a Period of 30 Seconds}
	\label{fig:res/occ/point}
\end{figure}

To quantify the variation in \gls{rss} values, \textit{\gls{mad}}, which is discussed in subsection \ref{subsec:meth/analysis/mad} of Chapter \ref{ch:meth}, of \gls{rssi} was calculated at each key point. Table \ref{tab:res/occ/point} highlights the \gls{mad} values for both \gls{los} and \gls{nlos} cases at each of those points.
\\
\begin{table}[htbp]
	
    \centering
    \arrayrulecolor{DarkOliveGreen} % Set the color of table rules 
    \begin{tabular}{>{\columncolor{MintCream}}c  >{\columncolor{MintCream}}c >{\columncolor{MintCream}}c}
	  \hline  \rowcolor{MidnightGreen}
         & \multicolumn{2}{c|}{\textcolor{white}{Median Absolute Deviation (dB)}} \\
        \cline{2-3}
         \rowcolor{MidnightGreen} \multirow{-2}{*}{\textcolor{white}{Waypoint}} & \textcolor{white}{\gls{los}} & \textcolor{white}{\gls{nlos}} \\
        \hline 
        Start& 0.40 & 1.50 \\
        \hline 
		Approach& 0.80 & 1.60 \\
        \hline 
		Centre& 0.30 & 1.25\\
        \hline 
		Depart& 0.96 & 2.07\\
        \hline 
		End& 0.60 & 1.57\\
        \hline\hline
    \end{tabular}
		
  \caption{Median Absolute Deviation at Each Key-Point, \textit{Start} to \textit{End}}
  \label{tab:res/occ/point}
\end{table}

Figure \ref{fig:res/occ/point} reveals that the median \gls{rss} values for the case of \gls{los} are better at two out of five key points, viz. \textit{start} and \textit{depart}, on the pathway when compared against those for the case of \gls{nlos}. While empirically that is equivalent to only 40\% of all the cases, the difference between the median values is significantly higher in the case of \gls{los} at the \textit{centre} point. This was possibly because, at the centre point, body occlusion results in full occlusion as opposed to partial occlusion at other key points, which was presented in Figure \ref{fig:meth/depdist_occ_vs_partial} in Chapter \ref{ch:meth}. That is, at all key points except \textit{centre}, the Broadcaster was fairly visible or in \gls{los} of the Observer as opposed to when at the \textit{centre}, the Broadcaster was completely occluded by the body and the signals must travel through the body or by undergoing reflection, refraction or diffraction to arrive at the Observer. However, Table \ref{tab:res/occ/point} reveals a more accurate insight into the differences between the \gls{los} and \gls{nlos} cases. The \gls{mad} values for the \gls{los} case were lower,  $<1\text{ }dB$, when compared against the \gls{nlos} case, $>1\text{ }dB$. This meant that there was more fluctuation on the signals when the Broadcaster was occluded.

\subsubsection{Sub-experiment 2}

The same analysis as for the previous sub-experiment was performed for this sub-experiment. Figure \ref{fig:res/occ/far_point} presents the error bars and markers to depict the fluctuations and the median value of \gls{rssi} respectively in red for or \gls{los}, and in blue for the case of \gls{nlos}. With the exception of point \textit{Q}, all other points presented the situation analogous to that in the previous sub-experiment, where the case of \gls{nlos} reduced the \gls{rss} of the acquired advertisements. There was no peak formation in this case which is possibly due to the fact that the signals were travelling from a greater distance and hence were subjected to increased path loss by the time they were acquired by the Observer. It is noteworthy however that the lowest value of \gls{rssi} associated with this sub-experiment was significantly higher than the lowest values in \textit{Sub-experiment 1}, that is where the signals are emerging from five key points on a pathway at a mere 3 $metres$ deployment distance of the Observer. Moreover, as seen in Table \ref{tab:res/occ/far_point}, the \gls{mad} values for the \gls{nlos} case were better than the \gls{mad} values for the same case in \textit{Sub-experiment 1}. While no tests were performed to unravel this occurrence, it is likely that this was due to reflection caused by the presence of metal infrastructure (as the literature suggests in Section \ref{sec:lit/ble/signal} in Chapter \ref{ch:lit}) at the experiment location, as can be seen at the bottom right corner of Figure \ref{fig:meth/satellite_exp} in Chapter \ref{ch:meth}. This metal infrastructure could have acting as an antenna to reflect the advertisements to the Observer. However, no experiments were undertaken in this PhD to further investigate this speculation.
\\

\begin{figure}[!t]
	\centerline{\includegraphics[width=\textwidth]{Figures/Ch4/occlusion/occlusion_median_far.png}}
	\caption{\gls{ble} \gls{rssi} at Key Points P to S (\gls{los} and \gls{nlos}) for a Period of 30 $Seconds$}
	\label{fig:res/occ/far_point}
\end{figure}

\begin{table}[htbp]
	
    \centering
    \arrayrulecolor{DarkOliveGreen} % Set the color of table rules 
    \begin{tabular}{>{\columncolor{MintCream}}c  >{\columncolor{MintCream}}c >{\columncolor{MintCream}}c}
	  \hline  \rowcolor{MidnightGreen}
       & \multicolumn{2}{c|}{\textcolor{white}{Median Absolute Deviation (dB)}} \\
        \cline{2-3}
         \rowcolor{MidnightGreen} \multirow{-2}{*}{\textcolor{white}{Waypoint}} & \textcolor{white}{\gls{los}} & \textcolor{white}{\gls{nlos}} \\
              \hline
            P & 0.60 & 1.20 \\
            \hline 
			Q & 0.90 & 0.80 \\
   \hline 
			R & 0.35 & 0.90\\
   \hline 
			S & 0.50 & 1\\
            \hline\hline
		\end{tabular}
		
  \caption{Median Absolute Deviation at Each Key-Point, \textit{P} to \textit{S}}
  \label{tab:res/occ/far_point}
\end{table}
\FloatBarrier

Finally, packet drop (as explained in Section \ref{eq:meth/dap} in Chapter \ref{ch:meth}) was observed for both the \gls{los} and \gls{nlos} cases for all five points on the pathway, and the four far points. This analysis was identified based on the \gls{ble} advertisements process, which is briefly described here first. From the Broadcaster's point of view, advertisements are broadcasted at fixed intervals, dictated by the configured advertisement interval. This configuration of advertisement interval for the Ruuvi beacon is hard-coded by the manufacturer and for the \gls{rpi} Broadcaster, it is configurable by the researcher. This configuration ensures that from the point of view of the Broadcaster, signals are sent out regularly at the set interval. If the Observer were to acquire every single advertisement that the Broadcaster has emitted, the Observer would be a recipient of a regularly-sampled signal. Since the signal is also emitted at fixed intervals, if adjusted for the delay in travel, the advertisements would correspond to a time series. Since 2.4 GHz \gls{rf} signals travel at the speed of light in a vacuum, assuming there is a comparatively negligible affect of the air, the delay in signals due to travel would only be in some tens of nanoseconds for the distances under scrutiny in the experiment presented here. However all advertisements are not received by the Observer due to various reasons including path loss, reflection, refraction, and diffraction, this means the advertisements received by the Observer can be referred to as \textit{sparsely-populated regularly-sampled time series}. This knowledge provides another means to analyse the signal. While such an assumption should facilitate the application of some \gls{dsp} techniques, only simple statistical techniques are employed in this experiment. However, this assumption is important to future work.
\\

In this particular analysis, using the knowledge that the Broadcaster employed had a 500 $ms$ advertisement interval, that is two advertisements per second, the number of advertisements emitted by the Broadcaster for each location was calculated by doubling the duration of the round in seconds. Since the number of advertisements is a whole number, the resulting advertisement counts from the doubling process were rounded off to the nearest integer. Using this, \acrfull{dap} and \acrfull{dar}, as discussed in Sections \ref{eq:meth/dap} and \ref{eq:meth/dar} in Chapter \ref{ch:meth}, combining all key points were calculated. Table \ref{tab:res/occ/droppath} and Table \ref{tab:res/occ/dropfar} summarises this information.
\\

\begin{table}[htbp]	
    \centering
    \arrayrulecolor{DarkOliveGreen} % Set the color of table rules 
    \resizebox{\textwidth}{!}{\begin{tabular}{>{\columncolor{MintCream}}c|  >{\columncolor{MintCream}}c >{\columncolor{MintCream}}c >{\columncolor{MintCream}}c  >{\columncolor{MintCream}}c >{\columncolor{MintCream}}c| >{\columncolor{MintCream}}c}
    \hline
        \rowcolor{MidnightGreen} \textcolor{white}{\textbf{Case}} & \textcolor{white}{\textbf{Locations}} & \textcolor{white}{\textbf{Observation Duration (s)}}& \textcolor{white}{\textbf{Advertisements Emitted}} & \textcolor{white}{\textbf{Advertisements Intercepted}} & \textcolor{white}{\textbf{\gls{dap} (\%)}} & \textcolor{white}{\textbf{Average \gls{dap} (\%)}} \\
        \hline
        & Start & 41.973 & 84 & 17 & 79.76 & \\
        & Approach & 34.989 & 70 & 41 & 41.43 & \\
        & Centre & 38.072 & 76 & 43 & 43.42 & \\
        & Depart & 39.476 & 79 & 49 & 37.97 & \\
        \multirow{-5}{*}{\rotatebox{90}{\textbf{\gls{los}}}}& End & 40.94 & 82 & 64 & 21.95 &\multirow{-5}{*}{\textbf{45.27}} \\
        \hline
        & Start & 50.434 & 101 & 58 & 42.57 & \\
        & Approach & 54.496 & 109 & 36 & 66.97 & \\
        & Centre & 43.07 & 86 & 22 & 74.42 & \\
        & Depart & 39.97 & 80 & 51 & 36.25 &\\
        \multirow{-5}{*}{\rotatebox{90}{\textbf{\gls{nlos}}}} & End & 45.519 & 91 & 26 & 71.43 & \multirow{-5}{*}{\textbf{58.67}} \\
        \hline\hline
    \end{tabular}}
    \caption{Advertisement Drop Percentage at Each Location on Pathway}
    \label{tab:res/occ/droppath}
\end{table}
\begin{table}[htbp]
    \centering
    \arrayrulecolor{DarkOliveGreen} % Set the color of table rules
    \resizebox{\textwidth}{!}{%
    \begin{tabular}{
        >{\columncolor{MintCream}}c|  
        >{\columncolor{MintCream}}c 
        >{\columncolor{MintCream}}c 
        >{\columncolor{MintCream}}c  
        >{\columncolor{MintCream}}c 
        >{\columncolor{MintCream}}c |
        >{\columncolor{MintCream}}c
    }
        \hline
        \rowcolor{MidnightGreen} \textcolor{white}{\textbf{Case}} & \textcolor{white}{\textbf{Locations}} & \textcolor{white}{\textbf{Duration of Walk (s)}} & \textcolor{white}{\textbf{Advertisements Emitted}} & \textcolor{white}{\textbf{Advertisements Intercepted}} & \textcolor{white}{\textbf{\gls{dap} (\%)}} & \textcolor{white}{\textbf{Average \gls{dap} (\%)}} \\
        \hline
       & P& 38.404 & 77 & 17 & 77.92 & \\
        & Q& 45.488 & 91 & 41 & 54.95 & \\
        & R& 44.057 & 88 & 43 & 51.14 & \\
         \multirow{-4}{*}{\rotatebox{90}{\textbf{\gls{los}}}}& S& 43.937 & 88 & 49 & 44.32 & \multirow{-4}{*}{\textbf{56.4}} \\
        \hline
        & P& 41.507 & 83 & 17 & 79.52 &  \\
        & Q& 44.488 & 89 & 41 & 53.93 & \\
        & R& 41.522 & 83 & 43 & 48.19 & \\
        \multirow{-4}{*}{\rotatebox{90}{\textbf{\gls{nlos}}}}& S& 45.005 & 90 & 49 & 45.56 &\multirow{-4}{*}{\textbf{56.52}} \\
        \hline
        \hline
    \end{tabular}}
    \caption{Advertisement Drop Percentage at Each Far Points, \textit{P, Q, R, and S}}
    \label{tab:res/occ/dropfar}
\end{table}

It can be seen from Table \ref{tab:res/occ/droppath} that the \gls{dap} was considerably lower in the case of \gls{los} compared to the case of \gls{nlos} at \textit{approach}, \textit{centre}, and \textit{end} key points. However, the same did not apply to the case when the advertisements are emitted from a device at a greater distance, as can be seen in Table \ref{tab:res/occ/dropfar}, where the difference was found to be marginal even if at certain key points the \gls{dap} was low. This further increased the likelihood that metal infrastructure at the far reaches of the depicted site could have acted as an antenna reflecting the signals to the Observer, thereby enabling the broadcasted advertisements to arrive at the occluded Observer just as they did without the occlusion.

\subsubsection{Sub-experiment 3}

To understand the effect of occlusion on the resulting \gls{rssi}, a ratio of the number of advertisements received within 10 percentage of the maximum values of \gls{rssi} over total number of advertisements was evaluated, as described in Subsection \ref{subsec:meth/analysis/adratio} in Chapter \ref{ch:meth}. Along with the ratio calculated in the previous step, the minimum, maximum, and median values of all the rounds for both cases were also calculated. Table \ref{tab:res/occ/run} presents the data evaluated from the measured \gls{rssi} signals. The \textit{interception rate} in the table is the ratio of total advertisements received by the Observer in the duration of the round over the total number of advertisements emitted by the Broadcaster during the walk. \textit{Max}, \textit{Min}, and \textit{Median} \gls{rssi} are the maximum, minimum, and median values of all \gls{rss} values collected in each round. The mean of these values for each individual round is presented in the \textit{Mean of Max}, \textit{Mean of Min}, and \textit{Mean of Median} \gls{rssi} columns. Using the maximum values of each individual round, the count of \gls{rssi} values within the 10\% range of the peak was evaluated, which is presented in the \textit{Count within 10\% of Peak} column. Finally, the ratio of the number of advertisements within 10\% of the peak and the total number of advertisements were calculated and are presented in the last column of the table. This ratio represents the accumulation of advertisements in the range of the peak. This empirical value helped in identifying the sharpness of the peak. If the \gls{rssi} values rose steeply, followed by a sharp decline, this ratio would be lower. Whereas, the higher value of the ratio denotes flatness around the region of the peak. 
\\

Evaluating the data provided in Table \ref{tab:res/occ/run}, it can be seen that the difference between the minimum and median values of the \gls{rssi} across the \gls{nlos} and \gls{los} cases was insignificant. Whereas, the maximum value of \gls{rssi} across the two cases demonstrated a difference, with the maximum \gls{rssi} values for \gls{los} case being -45.5 $dB$ and -43 $dB$ for rounds between \textit{start} to \textit{end} and \textit{end} to \textit{start} respectively, as opposed to -55 $dB$ and -48.9 $dB$ for the case of \gls{nlos}. The improvement in the lowest maximum \gls{rssi} of the two \gls{los} cases, -45.5 $dB$, against the highest maximum \gls{rssi} of the two \gls{nlos} cases, -48.9 $dB$, being at 7.47\%. Conversely, the improvement in the highest of the \gls{los} case, -43 $dB$, over the lowest of the \gls{nlos} case, -55 $dB$, standing at 27.90\%. Categorically, for the journeys from \textit{start} to \textit{end}, the maximum value of \gls{rssi} for \gls{los} case was 20.87\% over the \gls{nlos} case, whereas, for the journeys starting from \textit{end} to \textit{start}, the maximum value of \gls{rssi} was 13.72\% better in the case of \gls{los}. This further strengthened the likelihood of the presence of a sharper peak in the case of \gls{los}.
\\

The ratio of the count of \gls{rss} values within the 10\% range of the peak \gls{rssi} and the total number of advertisements measured by the Observer painted a clearer picture. The highest ratio across all rounds for combined journeys in both directions for the case of \gls{los} stood at 0.283, whereas that from the case of \gls{nlos} was 0.409. The mean of the ratios across all rounds while walking from \textit{start} to \textit{end} for the case of \gls{los} was 0.235 as compared to 0.305 in the case of \gls{nlos}. The same comparison for the journeys between \textit{end} and \textit{start} was more significant. With the mean ratio of only 0.167 for the case of \gls{los} against the mean of ratios of 0.305 in the case of \gls{nlos}, it was clear that \textit{the case of \gls{nlos} results in the flatness of plateau} in the resulting pattern from the measured \gls{rssi}. Therefore, \textit{body occlusion appears to act as a low-pass filter}.
\\

\begin{sidewaystable}[htbp]
    \centering
    \arrayrulecolor{DarkOliveGreen} % Set the color of table rules

    \begin{threeparttable}
    \resizebox{\textwidth}{!}{%
    \begin{tabular}{
        >{\columncolor{MintCream}}c|
        >{\columncolor{MintCream}}c 
        >{\columncolor{MintCream}}c 
        >{\columncolor{MintCream}}c 
        >{\columncolor{MintCream}}c  
        >{\columncolor{MintCream}}c 
        >{\columncolor{MintCream}}c 
        >{\columncolor{MintCream}}c 
        >{\columncolor{MintCream}}c 
        >{\columncolor{MintCream}}c  
        >{\columncolor{MintCream}}c 
        >{\columncolor{MintCream}}c 
        >{\columncolor{MintCream}}c 
        >{\columncolor{MintCream}}c
        >{\columncolor{MintCream}}c
    }
        \hline
        \rowcolor{MidnightGreen} &
        & 
        & 
        \textcolor{white}{\textbf{Advertisements}} & 
        \textcolor{white}{\textbf{Advertisements}} &
        \textcolor{white}{\textbf{Interception}} & 
        &
        \textcolor{white}{\textbf{Mean of}} &
        & 
        \textcolor{white}{\textbf{Mean of}} &
        & 
        \textcolor{white}{\textbf{Mean of}} &
        \textcolor{white}{Count within} 
        & 
        & \textcolor{white}{\textbf{Mean of}}\\
        
        \rowcolor{MidnightGreen} \multirow{-2}{*}{\textcolor{white}{\textbf{Case}}} & 
        \multirow{-2}{*}{\textcolor{white}{\textbf{Direction, round}}} & 
        \multirow{-2}{*}{\textcolor{white}{\textbf{Duration (s.sss)}}} & 
        \textcolor{white}{\textbf{Emitted}} & 
        \textcolor{white}{\textbf{Intercepted}} & 
        \textcolor{white}{\textbf{Rate (Hz)}} & 
        \multirow{-2}{*}{\textcolor{white}{\textbf{Max \gls{rssi} (dB)}}}& 
        \textcolor{white}{\textbf{Max \gls{rssi} (dB)}}& 
        \multirow{-2}{*}{\textcolor{white}{\textbf{Min \gls{rssi} (dB)}}}& 
        \textcolor{white}{\textbf{Min \gls{rssi} (dB)}}& 
        \multirow{-2}{*}{\textcolor{white}{\textbf{Median \gls{rssi} (dB)}}}& 
        \textcolor{white}{\textbf{Median \gls{rssi} (dB)}}& 
        \textcolor{white}{\textbf{10\% of Peak$^1$}} & 
        \multirow{-2}{*}{\textcolor{white}{\textbf{Ratio$^2$}}}&
        \textcolor{white}{\textbf{Ratios}} \\
        
        \hline
            & Start to End, 1 & 43.984 & 88 & 46 & 1.0458 & -53.4 & & -77 & & -63.65 & & 13 & 0.283 &\\
            & Start to End, 2 & 40.97 & 82 & 40 & 0.9763 & -52.8 & & -76.1 & & -65.7 & & 11 & 0.275 &\\
            & Start to End, 3 & 36.635 & 73 & 43 & 1.1737 & -45.5 & & -69.4 & & -59.7 & & 7 & 0.163 &\\
            & Start to End, 4 & 40.989 & 82 & 44 & 1.0735 & -47.1 & & -70.3 & & -62 & & 10 & 0.227 &\\
            & Start to End, 5 & 38.029 & 76 & 48 & 1.2622 & -52.6 & \multirow{-5}{*}{-45.5} & -68.6 & \multirow{-5}{*}{-77} & -61.3 & \multirow{-5} {*}{-62.47} & 11 & 0.229 & \multirow{-5}{*}{0.235}\\
            \cline{2-15}
            & End to Start, 1 & 42.277 & 85 & 49 & 1.159 & -45.2 & & -69 & & -62.2 & & 8 & 0.163 &\\
            & End to Start, 2 & 40.006 & 80 & 53 & 1.3248 & -43.2 & & -83 & & -59.8 & & 7 & 0.132 &\\
            & End to Start, 3 & 36.816 & 74 & 48 & 1.3038 & -49.3 & & -73 & & -59.2 & & 10 & 0.208 &\\
            & End to Start, 4 & 39.988 & 80 & 51 & 1.2754 & -45.1 & & -75 & & -60 & & 9 & 0.176 &\\
            \multirow{-10}{*}{\rotatebox{90}{\gls{los}}} & End to Start, 5 & 38.5 & 77 & 51 & 1.3247 & -43 & \multirow{-5}{*}{-43} & -65.6 &\multirow{-5}{*}{-83} & -58 & \multirow{-5}{*}{-59.84} & 8 & 0.157  & \multirow{-5}{*}{0.167}\\
            \hline
            
            & Start to End, 1 & 41.5 & 83 & 40 & 0.9639 & -59.5 & & -82 & & -68.45 & & 14 & 0.350 &\\
            & Start to End, 2 & 41.017 & 82 & 49 & 1.1946 & -55 & & -75 & & -64.7 & & 16 & 0.327 &\\
            & Start to End, 3 & 37.009 & 74 & 40 & 1.0808 & -56.1 & & -81 & & -65.65 & & 13 & 0.325 &\\
            & Start to End, 4 & 40.067 & 80 & 41 & 1.0233 & -56.5 & & -71.8 & & -67.33 & & 10 & 0.244 &\\
            & Start to End, 5 & 37.736 & 75 & 43 & 1.1395 & -60.7 & \multirow{-5}{*}{-55} & -78 & \multirow{-5}{*}{-82} & -68.8 & \multirow{-5}{*}{-66.986} & 12 & 0.279  & \multirow{-5}{*}{0.305}\\
            \cline{2-15}
            & End to Start, 1 & 41.402 & 83 & 42 & 1.0144 & -50.2 & & -67 & & -60.1 & & 11 & 0.262 &\\
            & End to Start, 2 & 39.035 & 78 & 39 & 0.9991 & -50.9 & & -68.25 & & -59.7 & & 15 & 0.385 &\\
            & End to Start, 3 & 39.488 & 79 & 44 & 1.1143 & -53.6 & & -76.3 & & -62.4 & & 18 & 0.409 &\\
            & End to Start, 4 & 40.494 & 81 & 49 & 1.2101 & -48.9 & & -75 & & -65.4 & & 7 & 0.143 &\\
            \multirow{-10}{*}{\rotatebox{90}{\gls{nlos}}} & End to Start, 5 & 41.906 & 84 & 46 & 1.0977 & -53.5 & \multirow{-5}{*}{-48.9} & -70.5 & \multirow{-5}{*}{-76.3} & -61.55 & \multirow{-5}{*}{-61.83} & 15 & 0.326  & \multirow{-5}{*}{0.305}\\
        \hline
        \hline
    \end{tabular}}
    \caption{Parameters of Observed Advertisements During Walk}
    \label{tab:res/occ/run}
    \begin{tablenotes}
    \item[1] \footnotesize{Number of Advertisements within the 10\% value of the peak or maximum of \gls{rssi}.}
    \item[2] \footnotesize{Ratio of count of advertisements intercepted within 10\% range of the peak or the maximum \gls{rssi} and the total number of advertisements received.}
    \end{tablenotes}
    \end{threeparttable}
\end{sidewaystable}

To visualise this flatness, Figures \ref{fig:res/occ/se_los} and \ref{fig:res/occ/es_los} present the plot between the \gls{rssi} values against the elapsed time of the walk for the case of \gls{los}. Whereas, Figures \ref{fig:res/occ/se_nlos} and \ref{fig:res/occ/es_nlos} present the chart of \gls{rssi} against elapsed time for the \gls{nlos} case. In addition to the \gls{rssi}, each plot contains horizontal lines representing the median of individual cases in respective colours, as shown in the legend of the figures, and another horizontal line in black colour representing the mean of the medians.
\\

Sharp peaks can be visually spotted in the first two figures and plateaus can be identified in the latter two. The figures also assist in the identification of another parameter that can be used as an empirical representation for the presence of occlusion, the distance between the peak and the median value. The empirical values of this distance are presented in Table \ref{tab:res/occ/gap}. It can be seen that the distance between the median and the peak value of \gls{rssi} in the case of \gls{nlos} is less, which further supports the finding that occlusion results in a plateau or flatness pattern in the \gls{rssi}, and the techniques presented here are useful in quantifying these plateaus.

\begin{figure}[htbp]
	\centerline{\includegraphics[width=\textwidth]{Figures/Ch4/evaluation/occlusion/Start to End, No Occlusion (LoS).jpg}}
	\caption{\gls{rssi}, Medians and Mean of all Medians for all Five rounds of Walk from Start to End in case of \gls{los}}
	\label{fig:res/occ/se_los}
\end{figure}

\begin{figure}[htbp]
	\centerline{\includegraphics[width=\textwidth]{Figures/Ch4/evaluation/occlusion/End to Start, No Occlusion (LoS).jpg}}
	\caption{\gls{rssi}, Medians and Mean of all Medians for all Five rounds of Walk from End to Start in case of \gls{nlos}}
	\label{fig:res/occ/es_los}
\end{figure}

\begin{figure}[htbp]
	\centerline{\includegraphics[width=\textwidth]{Figures/Ch4/evaluation/occlusion/Start to End, Occlusion (nLoS).jpg}}
	\caption{\gls{rssi}, Medians and Mean of all Medians for all Five rounds of Walk from Start to End in case of \gls{nlos}}
	\label{fig:res/occ/se_nlos}
\end{figure}

\begin{figure}[htbp]
	\centerline{\includegraphics[width=\textwidth]{Figures/Ch4/evaluation/occlusion/End to Start, Occlusion (nLoS).jpg}}
	\caption{\gls{rssi}, Medians and Mean of all Medians for all Five rounds of Walk from End to Start in case of \gls{nlos}}
	\label{fig:res/occ/es_nlos}
\end{figure}

\begin{table}[htbp]	
    \centering
    \arrayrulecolor{DarkOliveGreen} % Set the color of table rules 
    \resizebox{\textwidth}{!}{\begin{tabular}{
        >{\columncolor{MintCream}}c|  
        >{\columncolor{MintCream}}c 
        >{\columncolor{MintCream}}c
        >{\columncolor{MintCream}}c
        >{\columncolor{MintCream}}c
        >{\columncolor{MintCream}}c}
    \hline
        \rowcolor{MidnightGreen} \textcolor{white}{\textbf{Case}} & 
        \textcolor{white}{\textbf{Direction, round}} & 
        \textcolor{white}{\textbf{Peak (dB)}}& 
        \textcolor{white}{\textbf{Median (dB)}}& 
        \textcolor{white}{\textbf{Distance between Peak and Median (dB)}}& 
        \textcolor{white}{\textbf{Mean of Distances (dB)}}\\

        \hline
            & Start to End, 1 & -53.4 & -63.65 & 10.25 & \\
            & Start to End, 2 & -52.8 & -65.7 & 12.9 & \\
            & Start to End, 3 & -45.5 & -59.7 & 14.2 & \\
            & Start to End, 4 & -47.1 & -62 & 14.9 & \\
            & Start to End, 5 & -52.6 & -61.3 & 8.7 & \multirow{-5}{*}{12.19}\\
            \cline{2-6}& End to Start, 1 & -45.2  & -62.2 & 17 & \\
            & End to Start, 2 & -43.2 & -59.8 & 16.6 & \\
            & End to Start, 3 & -49.3 & -59.2 & 9.9 & \\
            & End to Start, 4 & -45.1 & -60 & 14.9 & \\
            \multirow{-10}{*}{\rotatebox{90}{\gls{los}}} & End to Start, 5 & -43 & -58 & 15 & \multirow{-5}{*}{14.68}\\
            \hline
            
            & Start to End, 1 & -59.5 & -68.45 & 8.95 & \\
            & Start to End, 2 & -55 & -64.7 & 9.7 & \\
            & Start to End, 3 & -56.1 & -65.65 & 9.55 & \\
            & Start to End, 4 & -56.5 & -67.33 & 10.83 & \\
            & Start to End, 5 & -60.7 & -68.8 & 8.1 & \multirow{-5}{*}{9.426}\\
            \cline{2-6}& End to Start, 1 & -50.2 & -60.1 & 9.9 & \\
            & End to Start, 2 & -50.9 & -59.7 & 8.8 & \\
            & End to Start, 3 & -53.6 & -62.4 & 8.8 & \\
            & End to Start, 4 & -48.9 & -65.4 & 16.5 & \\
            \multirow{-10}{*}{\rotatebox{90}{\gls{nlos}}} & End to Start, 5 & -53.5 & -61.55 & 8.05 & \multirow{-5}{*}{10.41}\\
        \hline
        \hline
    \end{tabular}}
    \caption{Gap Between Median \gls{rssi} and the Maximum (Peak) \gls{rssi}}
    \label{tab:res/occ/gap}
\end{table}

\vspace{10pt}

    \begin{tcolorbox}[
        colframe=white, % Border color
        colback=Ivory, % Background color
        coltitle=white, % Title text color
        title=Section Summary, % Title text
        fonttitle=\bfseries, % Title font style
        sharp corners, % Sharp corners for the main box
        enhanced,
        attach boxed title to top left={yshift=-2mm, xshift=2mm}, % Positioning the title
        boxed title style={colback=DarkSlateGray, colframe=SlateBlue, rounded corners=west, boxrule=0pt, top=2mm, bottom=2mm, right=2mm}, % Title box style
        width=\linewidth, % Width of the box
        boxrule=0pt, % No border for the main box
        drop shadow, % Shadow effect
        rounded corners, % Rounded corners for the main box
    ]
        The experiment conclusively shows that occlusion introduces an effect akin to a "clipping" filter, suppressing the peak on the \gls{rssi} time series in the presence of body occlusion. This subsequently results in a characteristic plateau in the shape of \gls{rssi} when plotted against elapsed time. Using common statistical methods, median and \gls{mad}, the results prove the possibility of identifying the plateau and therefore assert the possible presence of occlusion in most of the cases. This is true in both the scenarios where the pedestrian is stationary and where the pedestrian is moving on a linear path. As seen in figure \ref{fig:res/occ/point}, the \gls{rssi} is significantly higher near the Observer (point C) in the \gls{los} scenario when compared against the \gls{rssi} at other locations on the path, whereas, in the \gls{nlos} scenario, \gls{rssi} near the Observer remains in the same range as on the other points on the path. 
        \\

        It is also seen that the likelihood of the presence of body occlusion, as detected by the \gls{mad} analysis, can be further verified by measuring the difference between the median of all the collected \gls{rss} values during the walk and the maximum \gls{rssi} of those collected values. For the case of body occlusion, since there should be an absence of a sharp peak, the difference obtained between the median and peak \gls{rssi} must be smaller than the difference between the median and the peak of \gls{rssi} when there is no occlusion. Finally, the \gls{dar} is also an aid to detect occlusion. This is because with occlusion there is an increased probability of the advertisements losing their strength while traversing through the obstruction and thus not reaching the Observer. However, the \gls{dar} measure can only be employed in conjunction with other analysis methods if the advertisement rate of the Broadcaster is known. Otherwise, there is no way of knowing the number of advertisements emitted during the time the Broadcaster was \textit{visible} to the Observer.
    \\
    \\
    
    The work described in this section is presented in a paper titled, "Effects of Body Occlusion on Bluetooth Low Energy RSSI in Identifying Close Proximity of Pedestrians in Outdoor Environments" \citep{Parmar2022}.
    \end{tcolorbox}

\vspace{10pt}

\subsection{Effect of Advertisement Interval of \gls{ble} Broadcaster on Acquisition Capabilities of the Observer} \label{res/advrate}


\begin{figure}[!h]
	\centerline{\includegraphics[width=\textwidth]{Figures/Ch4/timeline/timeline_advert.png}}

\end{figure}

Understanding the effect of the advertisement interval or the advertisement rate is crucial. The range of \gls{ble} is limited and the strength of the signal reduces with distance. Therefore, when deployed in an outdoor pathway, an observing device is capable of producing useful results only for short distances. For example, a pathway of 50 $metres$ with a \gls{ble} Observer deployed at 25 $metres$ from each end, can be traversed by a casual walker at a speed of 1.4 $metres$ per $second$ in 35 $seconds$. During this time, the greater the number of advertisement packets received, the finer the granularity of the captured observations of the walk. However, there could be other bottlenecks with a very aggressive advertisement rate from a \gls{ble} Broadcaster. There could be competing packets on the same channel, resulting in more packet collisions, or the rate of packets intercepted could be too much for the Observer to process.  On the other hand, a lazy advertisement rate might produce only two or three measurements for the entire journey, resulting in insufficient data for any analysis of the type presented in this chapter to be possible. Therefore, an optimal advertisement rate must be investigated to ensure that a balance between the advertisement rate and the usefulness of the acquired data is maintained.
\\

Here, three advertisement interval candidates, viz. 100 $ms$, 500 $ms$, and 1000 $ms$ (or advertisement rate of 10 $Hz$, 2 $Hz$, and 1 $Hz$ respectively)  were investigated at a deployment distance of 3 $metres$. Since the Broadcaster in this experiment was based on a \gls{rpi}, the advertisement interval could be modified through reprogramming. Each advertisement packet emitted by the Broadcaster was supplemented with the sequence number of the advertisement itself so the receiver could identify the order of the packet it received. The advertisement record was also stored on the Broadcaster to enable a comparison between the transmitted advertisement and the received advertisement. 
\\

The measurements were collected on April 2, 2024 starting at 12:50 hours and lasting at 13:39 hours Irish time. A single volunteer pedestrian was required to partake in the experiment. The weather conditions at the time of the experiment is described below:
\\

\noindent\textbf{At 13:00 hours on April 2, 2024}
\begin{itemize}
    \item \textbf{Precipitation (Rain):} $0.0\,\text{mm}$
    \item \textbf{Air Temperature:} $11.3\,^\circ\mathrm{C}$
    \item \textbf{Wet Bulb Temperature:} $9.1\,^\circ\mathrm{C}$
    \item \textbf{Dew Point Temperature:} $6.6\,^\circ\mathrm{C}$
    \item \textbf{Vapour Pressure:} $9.8\,\text{hPa}$
    \item \textbf{Relative Humidity:} $73\,\%$
    \item \textbf{Mean Sea Level Pressure:} $1000.2\,\text{hPa}$
\end{itemize}


The most direct approach was comparing the total number of emitted advertisements against the total number of received advertisements. This comparison highlighted the number of packets dropped for each advertisement interval candidate. The hypothesis was that a smaller advertisement interval or an aggressive advertisement rate results in greater packet collision and an unmanageable workload on the Observer. To address this, the total count of advertisements, obtained through the last sequence number of the advertisement packet on the Broadcaster, was compared against the total count of advertisements observed by the Observer, obtained through counting the number of acquired advertisements.
\\

Table \ref{tab:res/advrate/droppedpercent} shows the number of advertisements sent, the number of advertisements received, the repetition of advertisements received, and the percentage of dropped advertisements by the Observer. Figures \ref{fig:res/advrate/los} and \ref{fig:res/advrate/nlos} also illustrate the percentage of dropped advertisement in the cases of \gls{los} and \gls{nlos} respectively for each advertisement rate. The repetition of advertisements signified the advertisement packets that were received more than once due to reflected transmissions arriving at the Observer from a different route. The percentage of dropped advertisements was found to be substantially higher for the advertisement interval of 100 $ms$, ranging in the high 80s and 90s. The performance of 500 $ms$ and 1000 $ms$ advertisement interval were comparable where almost half of the advertised signals were captured. Regardless, the number of intercepted signals with a 100 $ms$ advertisement interval, barring one \gls{los} travel from \textit{end} to \textit{start} were also comparable to the other advertisement intervals. So, the question then is, is there any significance of testing the advertisement rate/advertisement interval?
\\

\begin{table}[!h]
    \centering
    \arrayrulecolor{DarkOliveGreen} % Set the color of table rules    
    \begin{threeparttable}
    \resizebox{\textwidth}{!}{\begin{tabular}{>{\columncolor{MintCream}}c|  >{\columncolor{MintCream}}c| >{\columncolor{MintCream}}c| >{\columncolor{MintCream}}c|  >{\columncolor{MintCream}}c| >{\columncolor{MintCream}}c|  >{\columncolor{MintCream}}c| >{\columncolor{MintCream}}c}
        \hline \rowcolor{MidnightGreen}
        &&&&&& \\
        \rowcolor{MidnightGreen}
        &&&&&& \\
         \rowcolor{MidnightGreen}
         \multirow{-3}{*}{\textcolor{white}{\textbf{Advert Interval}}} & 
         \multirow{-3}{*}{\textcolor{white}{\textbf{Case}}} & 
         \multirow{-3}{2cm}{\textcolor{white}{\textbf{Travel Direction}}} & 
         \multirow{-3}{2cm}{\textcolor{white}{\textbf{Adverts Sent}}} & 
         \multirow{-3}{2cm}{\textcolor{white}{\textbf{Adverts Received}}} & 
         \multirow{-3}{3.5cm}{\textcolor{white}{\textbf{Repeated Packets Received}}} & 
         \multirow{-3}{2cm}{\textcolor{white}{\textbf{Drop percentage}}}\\
        \hline
         & & S -> E$^1$ & 295 & 18 & 0 & 93.89\%\\
         & \multirow{-2}{*}{\gls{los}} & E -> S$^2$ & 290 & 2 & 0 & 99.31\% \\
         \cline{3-7}
         &  & S -> E & 312 & 16 & 0 & 94.87\% \\
         \multirow{-4}{*}{100 $ms$} & \multirow{-2}{*}{\gls{nlos}} & E -> S & 222 & 40 & 0 & 81.98\% \\         
        \hline
        &  & S -> E & 45 & 24 & 1 & 46.66\% \\ 
        &  \multirow{-2}{*}{\gls{los}} & E -> S & 44 & 26 & 0 & 40.90\% \\
        \cline{3-7}
        & & S -> E & 46 & 20 & 0 & 56.52\% \\
        \multirow{-4}{*}{500 $ms$} & \multirow{-2}{*}{nLoS} & E -> S & 45 & 22 & 0 & 51.11\% \\
        \hline
        &  & S -> E & 26 & 14 & 0 & 46.15\% \\        
        & \multirow{-2}{*}{\gls{los}} & E -> S & 23 & 11 & 0 & 52.17\% \\
        \cline{3-7}
        &  & S -> E & 23 & 13 & 0 & 43.47\% \\
        \multirow{-4}{*}{1000 $ms$} & \multirow{-2}{*}{\gls{nlos}} & E -> S & 29 & 18 & 0 & 37.93\% \\

        \hline\hline
    \end{tabular}}
    \caption{Percentage Dropped Advertisements by Observer}
    \label{tab:res/advrate/droppedpercent}
    \begin{tablenotes}
    \item[1] \footnotesize{Travelling from the \textit{start} point to the \textit{end} point on the pathway.}
    \item[2] \footnotesize{Travelling from the \textit{end} point to the \textit{start} point on the pathway.}
    \end{tablenotes}
    \end{threeparttable}
    
\end{table} 

\begin{figure}[!htbp]
	\centerline{\includegraphics[width=\textwidth]{Figures/Ch4/evaluation/advert rate/LoS.jpg}}
	\caption{Comparison of Dropped Advertisements at Different Advertisement Intervals in \gls{los} Case}
	\label{fig:res/advrate/los}
\end{figure}

\begin{figure}[!h]
	\centerline{\includegraphics[width=\textwidth]{Figures/Ch4/evaluation/advert rate/nLoS.jpg}}
	\caption{Comparison of Dropped Advertisements at Different Advertisement Intervals in \gls{nlos} Case}
	\label{fig:res/advrate/nlos}
\end{figure}

To investigate further, a comparison of the number of signals acquired concerning the course of the walk of volunteer pedestrians was charted on a \textit{geoplot}. The data markers on the geoplot represent the measurement of location from the GPS and the number next to the markers represents the sequence number of the advertisements emitted by the Broadcaster around that region. Alongside each geoplot, the \gls{rssi} and the sequence number collected by the Observer for the respective round was plotted on a scatter chart. The marker on the graph represents the \gls{rssi} of the captured signal and the horizontal axis depicts the sequence number of the captured advertisement. For the case of 100 $ms$ advertisement interval, Figures \ref{fig:res/advrate/geo_100_los_se}, \ref{fig:res/advrate/geo_100_nlos_se}, \ref{fig:res/advrate/geo_100_los_es}, and \ref{fig:res/advrate/geo_100_nlos_es} represents the aforementioned geoplots for \gls{los} \textit{start} to \textit{end} case, \gls{nlos} \textit{start} to \textit{end} case, \gls{los} \textit{end} to \textit{start} case, and \gls{nlos} \textit{end} to \textit{start} case respectively. It can be observed that the interception of advertisements for the 100 $ms$ advertisement interval was patchy, especially in Figures \ref{fig:res/advrate/geo_100_nlos_se}, \ref{fig:res/advrate/geo_100_los_es}. This indicates a failure to acquire significant measurements between the walks to result in useful patterns. However, there was also the case in Figure \ref{fig:res/advrate/geo_100_nlos_es}, where the advertisements were evenly distributed, however, that is the best result out of  all the rounds. While the advertisement interval of 100 $ms$ does provide valuable information, its likelihood of working reliably is lower. Referring back to Table \ref{tab:res/advrate/droppedpercent}, the percentage of dropped advertisement packets was observed to be  extremely high for the 100 $ms$ advertisement interval to be considered reliable.
\\

\begin{figure}[!htbp]
	\centerline{\includegraphics[width=\textwidth]{Figures/Ch4/evaluation/advert rate/100ms/Geoplot 3m los s-e 100ms.jpg}}
	\caption{Geoplot and Captured Advertisement for 100 $ms$ Advertisement Interval on a Walk from \textit{Start} to \textit{End} in \gls{los} case}
	\label{fig:res/advrate/geo_100_los_se}
\end{figure}

\begin{figure}[!htbp]
	\centerline{\includegraphics[width=\textwidth]{Figures/Ch4/evaluation/advert rate/100ms/Geoplot 3m nlos s-e 100ms.jpg}}
	\caption{Geoplot and Captured Advertisement for 100 $ms$ Advertisement Interval on a Walk from \textit{Start} to \textit{End} in \gls{nlos} case}
	\label{fig:res/advrate/geo_100_nlos_se}
\end{figure}

\begin{figure}[!htbp]
	\centerline{\includegraphics[width=\textwidth]{Figures/Ch4/evaluation/advert rate/100ms/Geoplot 3m los e-s 100ms.jpg}}
	\caption{Geoplot and Captured Advertisement for 100 $ms$ Advertisement Interval on a Walk from \textit{End} to \textit{Start} in \gls{los} case}
	\label{fig:res/advrate/geo_100_los_es}
\end{figure}

\begin{figure}[!htbp]
	\centerline{\includegraphics[width=\textwidth]{Figures/Ch4/evaluation/advert rate/100ms/Geoplot 3m nlos e-s 100ms.jpg}}
	\caption{Geoplot and Captured Advertisement for 100 $ms$ Advertisement Interval on a Walk from \textit{End} to \textit{Start} in \gls{nlos} case}
	\label{fig:res/advrate/geo_100_nlos_es}
\end{figure}
\FloatBarrier

Similarly, Figures \ref{fig:res/advrate/geo_500_los_se}, \ref{fig:res/advrate/geo_500_nlos_se}, \ref{fig:res/advrate/geo_500_los_es}, and \ref{fig:res/advrate/geo_500_nlos_es} represents the geoplots for \gls{los} \textit{start} to \textit{end} case, \gls{nlos} \textit{start} to \textit{end} case, \gls{los} \textit{end} to \textit{start} case, and \gls{nlos} \textit{end} to \textit{start} case respectively, for the advertisement interval of 500 $ms$. In this case, while empty patches can be seen just as in the case of the 100 $ms$ advertisement interval, they were equally distributed to generate meaningful patterns.
\\

\begin{figure}[!htbp]
	\centerline{\includegraphics[width=\textwidth]{Figures/Ch4/evaluation/advert rate/500ms/Geoplot 3m los s-e 500ms.jpg}}
	\caption{Geoplot and Captured Advertisement for 500 $ms$ Advertisement Interval on a Walk from \textit{Start} to \textit{End} in \gls{los} case}
	\label{fig:res/advrate/geo_500_los_se}
\end{figure}

\begin{figure}[!htbp]
	\centerline{\includegraphics[width=\textwidth]{Figures/Ch4/evaluation/advert rate/500ms/Geoplot 3m nlos s-e 500ms.jpg}}
	\caption{Geoplot and Captured Advertisement for 500 $ms$ Advertisement Interval on a Walk from \textit{Start} to \textit{End} in \gls{nlos} case}
	\label{fig:res/advrate/geo_500_nlos_se}
\end{figure}

\begin{figure}[!h]
	\centerline{\includegraphics[width=\textwidth]{Figures/Ch4/evaluation/advert rate/500ms/Geoplot 3m los e-s 500ms.jpg}}
	\caption{Geoplot and Captured Advertisement for 500 $ms$ Advertisement Interval on a Walk from \textit{End} to \textit{Start} in \gls{los} case}
	\label{fig:res/advrate/geo_500_los_es}
\end{figure}

\begin{figure}[!h]
	\centerline{\includegraphics[width=\textwidth]{Figures/Ch4/evaluation/advert rate/500ms/Geoplot 3m nlos e-s 500ms.jpg}}
	\caption{Geoplot and Captured Advertisement for 500 $ms$ Advertisement Interval on a Walk from \textit{End} to \textit{Start} in \gls{nlos} case}
	\label{fig:res/advrate/geo_500_nlos_es}
\end{figure}
\FloatBarrier

Finally, from Figures  \ref{fig:res/advrate/geo_1000_los_se}, \ref{fig:res/advrate/geo_1000_nlos_se}, \ref{fig:res/advrate/geo_1000_los_es}, and \ref{fig:res/advrate/geo_1000_nlos_es}, representing the geoplots for \gls{los} \textit{start} to \textit{end} case, \gls{nlos} \textit{start} to \textit{end} case, \gls{los} \textit{end} to \textit{start} case, and \gls{nlos} \textit{end} to \textit{start} case respectively for the advertisement interval of 1000 $ms$, it can be seen that the intercepted advertisements were evenly spread out throughout the duration of the walk, as was also the case for 500 $ms$ advertisement interval. Note, that while the patches appear to have large empty spaces in the figure, they are evenly spread and only appear larger because of the scale of the x-axis. While the space of the x-axis could be kept uniform for all the observations, it is intentionally auto-adjusted to ensure that the axis is no too tightly packed and is readable.
\\
\begin{figure}[!h]
	\centerline{\includegraphics[width=\textwidth]{Figures/Ch4/evaluation/advert rate/1000ms/Geoplot 3m los s-e 1000ms.jpg}}
	\caption{Geoplot and Captured Advertisement for 1000 $ms$ Advertisement Interval on a Walk from \textit{Start} to \textit{End} in \gls{los} case}
	\label{fig:res/advrate/geo_1000_los_se}
\end{figure}

\begin{figure}[!htbp]
	\centerline{\includegraphics[width=\textwidth]{Figures/Ch4/evaluation/advert rate/1000ms/Geoplot 3m nlos s-e 1000ms.jpg}}
	\caption{Geoplot and Captured Advertisement for 1000 $ms$ Advertisement Interval on a Walk from \textit{Start} to \textit{End} in \gls{nlos} case}
	\label{fig:res/advrate/geo_1000_nlos_se}
\end{figure}

\begin{figure}[!htbp]
	\centerline{\includegraphics[width=\textwidth]{Figures/Ch4/evaluation/advert rate/1000ms/Geoplot 3m los e-s 1000ms.jpg}}
	\caption{Geoplot and Captured Advertisement for 1000 $ms$ Advertisement Interval on a Walk from \textit{End} to \textit{Start} in \gls{los} case}
	\label{fig:res/advrate/geo_1000_los_es}
\end{figure}

\begin{figure}[!htbp]
	\centerline{\includegraphics[width=\textwidth]{Figures/Ch4/evaluation/advert rate/1000ms/Geoplot 3m nlos e-s 1000ms.jpg}}
	\caption{Geoplot and Captured Advertisement for 1000 $ms$ Advertisement Interval on a Walk from \textit{End} to \textit{Start} in \gls{nlos} case}
	\label{fig:res/advrate/geo_1000_nlos_es}
\end{figure}
\FloatBarrier

The issue with the uneven scaling of the x-axis on the geoplots, as mentioned earlier, was however tackled by directly analysing the number of advertisements initiated by the Broadcaster at each location marker along the path measured by the \textit{GPS Logger} and the number of those advertisements captured by the Observer. For instance, say at the third GPS marker, the Broadcaster emitted 10 advertisements at a 100 $ms$ advertisement interval, how many of those 10 advertisements emitted by the Broadcaster are captured by the Observer? Figures \ref{fig:res/advrate/adv_100_los_se}, \ref{fig:res/advrate/adv_100_nlos_se}, \ref{fig:res/advrate/adv_100_los_es}, and \ref{fig:res/advrate/adv_100_nlos_es} compare the number of advertisements emitted against the number of advertisements captured at each point during a walk in the \gls{los} case \textit{start} to \textit{end}, \gls{nlos} case \textit{start} to \textit{end}, \gls{los} case \textit{end} to \textit{start}, and \gls{nlos} case \textit{end} to \textit{start} respectively, for the 100 $ms$ advertisement interval. As seen in the figures, despite broadcasting ten advertisements every second, the Observer failed to retrieve them in three of the four presented cases. This may be due to the repeated interception of advertisement messages creating a bottleneck for the \gls{rpi} Observer such that the overload resulted in momentary freezes in the system.
\\

\begin{figure}[!h]
	\centerline{\includegraphics[width=\textwidth]{Figures/Ch4/evaluation/advert rate/100ms/adv_los_se_100ms_.jpg}}
	\caption{Advertisements Emitted vs Advertisements Captured at Each GPS Measurement for 100 $ms$ Advertisement Interval on a Walk from \textit{Start} to \textit{End} in \gls{los} case}
	\label{fig:res/advrate/adv_100_los_se}
\end{figure}

\begin{figure}[!h]
	\centerline{\includegraphics[width=\textwidth]{Figures/Ch4/evaluation/advert rate/100ms/adv_nlos_se_100ms_.jpg}}
	\caption{Advertisements Emitted vs Advertisements Captured at Each GPS Measurement for 100 $ms$ Advertisement Interval on a Walk from \textit{Start} to \textit{End} in \gls{nlos} case}
	\label{fig:res/advrate/adv_100_nlos_se}
\end{figure}

\begin{figure}[!h]
	\centerline{\includegraphics[width=\textwidth]{Figures/Ch4/evaluation/advert rate/100ms/adv_los_es_100ms_.jpg}}
	\caption{Advertisements Emitted vs Advertisements Captured at Each GPS Measurement for 100 $ms$ Advertisement Interval on a Walk from \textit{End} to \textit{Start} in \gls{los} case}
	\label{fig:res/advrate/adv_100_los_es}

\end{figure}

\begin{figure}[!h]
	\centerline{\includegraphics[width=\textwidth]{Figures/Ch4/evaluation/advert rate/100ms/adv_nlos_es_100ms_.jpg}}
	\caption{Advertisements Emitted vs Advertisements Captured at Each GPS Measurement for 100 $ms$ Advertisement Interval on a Walk from \textit{End} to \textit{Start} in \gls{nlos} case}
	\label{fig:res/advrate/adv_100_nlos_es}
\end{figure}
\FloatBarrier

Likewise, Figures  \ref{fig:res/advrate/adv_500_los_se}, \ref{fig:res/advrate/adv_500_nlos_se}, \ref{fig:res/advrate/adv_500_los_es}, and \ref{fig:res/advrate/adv_500_nlos_es} compare the number of advertisements emitted against the number of advertisements captured at each point during a walk in \gls{los} case \textit{start} to \textit{end}, \gls{nlos} case \textit{start} to \textit{end}, \gls{los} case \textit{end} to \textit{start}, and \gls{nlos} case \textit{end} to \textit{start} respectively, for the 500 $ms$ advertisement interval. It is apparent from the figure that the advertisements in this case were more evenly captured. In many locations, at least one of the two broadcasts was acquired by the Observer. The results suggest the superiority of the 500 $ms$ advertisement interval over the 100 $ms$ advertisement interval, and perhaps, this is the reason that 500 $ms$ is generally the minimum advertising interval on available beacons.
\\
\begin{figure}[htbp]
	\centerline{\includegraphics[width=\textwidth]{Figures/Ch4/evaluation/advert rate/500ms/adv_los_se_500ms_.jpg}}
	\caption{Advertisements Emitted vs Advertisements Captured at Each GPS Measurement for 500 $ms$ Advertisement Interval on a Walk from \textit{Start} to \textit{End} in \gls{los} case}
	\label{fig:res/advrate/adv_500_los_se}
\end{figure}

\begin{figure}[htbp]
	\centerline{\includegraphics[width=\textwidth]{Figures/Ch4/evaluation/advert rate/500ms/adv_nlos_se_500ms_.jpg}}
	\caption{Advertisements Emitted vs Advertisements Captured at Each GPS Measurement for 500 $ms$ Advertisement Interval on a Walk from \textit{Start} to \textit{End} in \gls{nlos} case}
	\label{fig:res/advrate/adv_500_nlos_se}
\end{figure}

\begin{figure}[htbp]
	\centerline{\includegraphics[width=\textwidth]{Figures/Ch4/evaluation/advert rate/500ms/adv_los_es_500ms_.jpg}}
	\caption{Advertisements Emitted vs Advertisements Captured at Each GPS Measurement for 500 $ms$ Advertisement Interval on a Walk from \textit{End} to \textit{Start} in \gls{los} case}
	\label{fig:res/advrate/adv_500_los_es}

\end{figure}

\begin{figure}[htbp]
	\centerline{\includegraphics[width=\textwidth]{Figures/Ch4/evaluation/advert rate/500ms/adv_nlos_es_500ms_.jpg}}
	\caption{Advertisements Emitted vs Advertisements Captured at Each GPS Measurement for 500 $ms$ Advertisement Interval on a Walk from \textit{End} to \textit{Start} in \gls{nlos} case}
	\label{fig:res/advrate/adv_500_nlos_es}

\end{figure}
\FloatBarrier

And finally, from Figures  \ref{fig:res/advrate/adv_1000_los_se}, \ref{fig:res/advrate/adv_1000_nlos_se}, \ref{fig:res/advrate/adv_1000_los_es}, and \ref{fig:res/advrate/adv_1000_nlos_es}, representing the \gls{los} and \gls{nlos} \textit{start} to \textit{end} walk and \gls{los} and \gls{nlos} \textit{end} to \textit{start} walks, for the 1000 $ms$ advertisement interval, it can be seen that the acquired advertisements were just as evenly spread as in the case of the 500 $ms$ advertisement interval. The results suggest the superiority of the 1000 $ms$ advertisement interval over the 100 $ms$ advertisement interval. The percentage of dropped advertisements, as seen in Table \ref{tab:res/advrate/droppedpercent}, for both 500 $ms$ and 1000 $ms$, are comparable, as is the acquisition of the broadcasts by the Observer at both these advertisement intervals. The choice of the 500 $ms$ advertisement interval should be more useful than the 1000 $ms$ interval because the finer measurement of granularity could provide insights into nuances in the measured walks.
\\

\begin{figure}[htbp]
	\centerline{\includegraphics[width=\textwidth]{Figures/Ch4/evaluation/advert rate/1000ms/adv_los_se_1000ms_.jpg}}
	\caption{Advertisements Emitted vs Advertisements Captured at Each GPS Measurement for 1000 $ms$ Advertisement Interval on a Walk from \textit{Start} to \textit{End} in \gls{los} case}
	\label{fig:res/advrate/adv_1000_los_se}
\end{figure}

\begin{figure}[htbp]
	\centerline{\includegraphics[width=\textwidth]{Figures/Ch4/evaluation/advert rate/1000ms/adv_nlos_se_1000ms_.jpg}}
	\caption{Advertisements Emitted vs Advertisements Captured at Each GPS Measurement for 1000 $ms$ Advertisement Interval on a Walk from \textit{Start} to \textit{End} in \gls{nlos} case}
	\label{fig:res/advrate/adv_1000_nlos_se}
\end{figure}

\begin{figure}[htbp]
	\centerline{\includegraphics[width=\textwidth]{Figures/Ch4/evaluation/advert rate/1000ms/adv_los_es_1000ms_.jpg}}
	\caption{Advertisements Emitted vs Advertisements Captured at Each GPS Measurement for 1000 $ms$ Advertisement Interval on a Walk from \textit{End} to \textit{Start} in \gls{los} case}
	\label{fig:res/advrate/adv_1000_los_es}

\end{figure}

\begin{figure}[htbp]
	\centerline{\includegraphics[width=\textwidth]{Figures/Ch4/evaluation/advert rate/1000ms/adv_nlos_es_1000ms_.jpg}}
	\caption{Advertisements Emitted vs Advertisements Captured at Each GPS Measurement for 1000 $ms$ Advertisement Interval on a Walk from \textit{End} to \textit{Start} in \gls{nlos} case}
	\label{fig:res/advrate/adv_1000_nlos_es}
\end{figure}
\FloatBarrier

One noteworthy observation from this experiment was that the Observer never captured more than two advertisements even when ten advertisements were broadcasted despite not restricting the sampling rate of the \gls{rpi} Observer in the main Python script. This revelation hinted that either \gls{rpi} and/or the Bluepy library \textit{may} have internal restrictions on the sampling rate. This is however a speculation and more robust experimentation is required to investigate it further. 

\vspace{10pt}

    \begin{tcolorbox}[
        colframe=white, % Border color
        colback=Ivory, % Background color
        coltitle=white, % Title text color
        title=Section Summary, % Title text
        fonttitle=\bfseries, % Title font style
        sharp corners, % Sharp corners for the main box
        enhanced,
        attach boxed title to top left={yshift=-2mm, xshift=2mm}, % Positioning the title
        boxed title style={colback=DarkSlateGray, colframe=SlateBlue, rounded corners=west, boxrule=0pt, top=2mm, bottom=2mm, right=2mm}, % Title box style
        width=\linewidth, % Width of the box
        boxrule=0pt, % No border for the main box
        drop shadow, % Shadow effect
        rounded corners, % Rounded corners for the main box
        breakable
    ]
        In a real-world deployment of this technology to understand pedestrians' activities, movements, and space utilisation, the Observer may be measuring data from different types of Broadcasters which may have varying advertisement rates. Therefore, this experiment is an investigation of the important parameter of advertisement interval. Three sample advertisement intervals are tested, 100 $ms$, 500 $ms$, and 1000 $ms$.
        \\

        The first analysis is a simple check of the dropped advertisement interval, which presents a significantly higher drop percentage for the 100 $ms$ advertisement interval, whereas the other two candidate advertisement intervals both showcase a lower drop percentage. However, this is not a very useful parameter in the real-world scenario as the advertisement interval, which is a required attribute to calculate the drop percentage, is unavailable. A crucial point to note here however is that the number of received advertisements remains similar in all the cases. This may suggest that the bottleneck is in the Observer.
        \\

        When the \gls{rssi} of the advertisements are plotted against elapsed time or against the sequence number of the \gls{rssi}, a much clearer insight is obtained. While the number of advertisements captured in the case of 100 $ms$ is in the same ballpark as the other two candidates, the received advertisements are spread significantly far apart. This hinders the intention of identifying patterns in the collected \gls{rssi} and subsequently, asserting associated activity or movement dynamics. The advertisement intervals of 500 $ms$ and 1000 $ms$ however produce similar results with the advertisements close enough to enable the identification of patterns. Another crucial yet unintended outcome of the experiment is the unravelling of the bottleneck of the Observer. When the number of advertisements broadcasted and captured are studied in figures \ref{fig:res/advrate/adv_100_los_se}, \ref{fig:res/advrate/adv_100_los_es}, \ref{fig:res/advrate/adv_500_los_se}, \ref{fig:res/advrate/adv_500_los_es}, \ref{fig:res/advrate/adv_1000_los_se}, and \ref{fig:res/advrate/adv_1000_los_es}, it is seen that the number of intercepted advertisements by the Observer is never more than two. This supports the inference made through the previous analysis about the limitation of the Observer used in this experiment. Note that this limitation may not apply to any other \gls{sbc} apart from the one used in all the experiments conducted during this PhD.
    \end{tcolorbox}


\section{Evaluation of the Platform's Abilities to Detect, Identify, and Characterise Pedestrian Movement and Activities} \label{sec:res/experiments}

\subsection{Pause Detection in the Movement of the Pedestrians} \label{res/pause}


\begin{figure}[!h]
	\centerline{\includegraphics[width=\textwidth]{Figures/Ch4/timeline/timeline_pause.png}}
\end{figure}
\FloatBarrier

Identification of interactions with other pedestrians or the environment is simply a correlation of the pauses in the journey of a pedestrian. This approach provides no certainty that the pedestrian has paused to interact with either the surrounding environment or with other pedestrians. It is only an indication that the pause in the journey could be likely due to some form of interaction with another pedestrian or with the environment. This uncertainty is notable as it accounts for no intrusive personal behaviour monitoring and yet, if such a measurement is detected in the same region across days or with other pedestrians, it could be asserted with more confidence that the pauses are an indication of something notable or interesting in the environment or a presence, for example, of an obstruction on the path. Similarly, if two or more \gls{ble} Broadcasters pause at the same time for the same duration in the same region, it is more likely to be an interaction. But all of this information is unknown until such nuances are observed.
\\

Measurements for this experiment were collected on June 9, 2023 between 16:51 and 17:18 hours Irish time with the support from a single volunteer pedestrian. The weather information on the day of the experiment are presented below:
\\

\noindent\textbf{At 17:00 hours on June 9, 2023}
\begin{itemize}
    \item \textbf{Precipitation (Rain):} $0.0\,\text{mm}$
    \item \textbf{Air Temperature:} $18.5\,^\circ\mathrm{C}$
    \item \textbf{Wet Bulb Temperature:} $14.2\,^\circ\mathrm{C}$
    \item \textbf{Dew Point Temperature:} $10.6\,^\circ\mathrm{C}$
    \item \textbf{Vapour Pressure:} $12.8\,\text{hPa}$
    \item \textbf{Relative Humidity:} $60\,\%$
    \item \textbf{Mean Sea Level Pressure:} $1013.8\,\text{hPa}$
\end{itemize}


In this experiment, the interaction of the pedestrian with other pedestrians and with the environment was asserted through the identification of pauses in the movement of the pedestrian carrying a Broadcaster on the linear path. A 24-$metre$ long pathway, at a deployment distance of 3 $metres$ with two end points, the \textit{start} point and the \textit{end} point, where the former is towards the 0\textdegree{} region of the Observer and the latter at the 180\textdegree{} region was selected. A point 6 $metres$ away from the \textit{start} point was marked as the \textit{approach} point which was a designated stop point for the volunteer pedestrian. This is depicted in Figure \ref{fig:location}. In this experimental setup, three scenarios were tested with the Broadcaster always in the \gls{los} of the Observer. First, when the pedestrian started the journey at the \textit{start} point and paused at the \textit{approach} point for 5 $seconds$, second, for 10 $seconds$, and finally third, for 15 $seconds$, before the pedestrian continued the onward journey to the \textit{end} point. Each scenario was repeated four times, totalling up to 12 individual repetitions. The ground truth was obtained from the \textit{Blue Dot} application. Collected data included the \gls{mac} address, timestamped raw \gls{rssi} and sliding window \gls{sma} filtered \gls{rssi}, and the ground truth location obtained through the use of Blue Dot.
\\

The collected data was subjected to the analysis steps as described below:

\begin{enumerate}
    \item \textit{Curve Fitting}: This was performed to interpolate a smooth curve that fits the trend of measurement values of the \gls{rssi}. A polynomial equation of a selected order was applied over the measured \gls{rssi}, then curve fitting and \gls{sse} was calculated to check the closeness of the resulting curve against the actual \gls{rssi}. \gls{sse} was also used to assess the fit of curves generated with different orders of the polynomial. This two-step method is described in the Section \ref{subsec:meth/analysis/curve} in Chapter \ref{ch:meth}. This process helped in diminishing the effect of outliers and fluctuations, and is presented as an example in Figure \ref{fig:poly_example}. As seen in the figure, the blue line and marker $\times$, representing the measured \gls{rssi}, on the 14$^{th}$ sample corresponds to an anomalous dip. The effect of which are diminished in the resulting polynomial curve, in red. 
\\

    
    \begin{figure}[htbp]
    	\centerline{\includegraphics[width=\textwidth]{Figures/Ch4/example.jpg}}
    	\caption{Curve Fitting for Example \gls{rss} Values}
    	\label{fig:poly_example}
    \end{figure}

    As mentioned in Section \ref{subsec:meth/analysis/curve} in Chapter \ref{ch:meth}, \gls{sse} was used to identify the order of the polynomial that produces a curve that closely approximates the measurements. The order of the tested polynomials was varied from 2 to 9 for each of the four repetitions of each pause duration: 5, 15 and 25 $seconds$. A polynomial with degree 9 was selected as it demonstrated the lowest \gls{sse}, and hence approximated the measurements more closely than the other orders of the polynomial. This is shown in Figure \ref{fig:fitness}. The curves that resulted from selected degree 9 polynomial fitting in Figure \ref{fig:polyfit_overlay} were overlaid on the original \gls{rss} values. It is important to note that the \gls{rss} values in all the graphs presented in this experiment are interconnected using a line segment only for comprehension. There is no certainty that the same line segment would be observed if there were additional \gls{rss} values obtained between the two measured consecutive \gls{rss} values. 

    \begin{figure}[!ht]
        \centerline{\includegraphics[width=\textwidth]{Figures/Ch4/fitness.jpg}}
        \caption{Comparison of \gls{sse} of Varying Degrees of Polynomial for Curve Fitting}
        \label{fig:fitness}
    \end{figure}

    
    \begin{figure}[!hb]
        \centering
        \begin{subfigure}{0.95\textwidth}
            \includegraphics[width=\textwidth]{Figures/Ch4/5s_polyfit_overlay_spline.jpg}
            \caption{5 $Seconds$ Pause All Cases}
        \end{subfigure}
    
        \begin{subfigure}{0.95\textwidth}
            \includegraphics[width=\textwidth]{Figures/Ch4/15s_polyfit_overlay_spline.jpg}
            \caption{15 $Seconds$ Pause All Cases}
        \end{subfigure}
    \end{figure}    
    \begin{figure}[!ht]\ContinuedFloat
        \centering
        \begin{subfigure}{0.95\textwidth}
            \includegraphics[width=\textwidth]{Figures/Ch4/25s_polyfit_overlay_spline.jpg}
            \caption{25 $Seconds$ Pause All Cases}
        \end{subfigure}
        
        \caption{Polynomial Curve Overlaid on the Original \gls{rssi}}
        \label{fig:polyfit_overlay}
    \end{figure}
    
    \item \textit{Sliding Window \gls{sd}}: A sliding window on the output curve of the polynomial curve fitting was used to calculate the \gls{sd}. Through this, of temporal changes in the trend of \gls{rss} values were observed. This sliding window was based on time rather than number of samples, as described in detail in the Subsection \ref{subsec:meth/analysis/swsd} in Chapter \ref{ch:meth}. As an example, a window size of 4 denotes a window of 4 $seconds$ and all the data within this segment is used to calculate the \gls{sd}. The choice of time-based windows ensured that no information was lost from the collected \gls{rssi}. Consideration of a time-dependent versus sample-dependent sliding window was significant as it highlights another important relevant aspect. To understand this aspect, one can observe the Advertisements from the perspective of both the Broadcaster and the Observer. Through the lens of the Broadcaster, \gls{rss} values ensuing from advertisements are regularly emitted time-series measurements since the advertisement interval for the Broadcaster used in this experiment has a fixed interval of 0.5 $seconds$. However, during transmission, as discussed in the previously in Section \ref{subsec:res/occlusion} of this chapter, these advertisements are sometimes lost or delayed, resulting in irregularly sampled time-series data from the Observer's point-of-view. If a graph of the \gls{rss} data as detected by the Observer without considering their time of detection at the top of the \gls{ble} stack was to be observed, the \gls{rss} values would appear to be evenly distributed, therefore losing the important information about the gap in the time of arrival of two consecutive \gls{rssi} values. If the sliding window approach uses a sample-based window instead of a time-based one, it will consider the values that may be outside a trend in the overall \gls{rssi} if there are lost advertisements in between. This is depicted in Figure \ref{fig:timevsindex}. Hence, to retain these nuances, a time-based windowing approach to draw inferences was adopted.
\\

    \begin{figure}[!h]
        \centerline{\includegraphics[width=\textwidth]{Figures/Ch4/time vs index.jpg}}
        \caption{Curve Fitting for Example \gls{rss} Values}
        \label{fig:timevsindex}
    \end{figure}
    
    The sliding window started at the very first recorded \gls{rss} value in the entire journey and depending on the size of the window, all \gls{rssi} records between the first record and the window size were converted in to a set. The window was then incremented with a step of 1 $second$ to form subsequent sets until, the window was moved as far in the \gls{rssi} records that the final \gls{rssi} record is a part of the last set. \gls{sd} was then computed on the \gls{rssi} records for each of those sets..
\\

    Each of the 4 repetitions from every pause duration of 5, 15 and 25 $seconds$ was subjected to the sliding window subset generation that varied between 2 and 10-$second$ window sizes, totalling 108 sets for the calculations of \gls{sd}. Any flatness in the plot of standard deviation indicated that the \gls{rssi} did not vary significantly during that period. Since the \gls{rssi} is a function of the distance between the Observer and Broadcaster, the near constancy of the \gls{rssi} should signify a stop in the movement of the pedestrian.
\clearpage
    \item \textit{Pause Detection and Fine Tuning}
    Pause detection through examination of \gls{rss} values employed a simple threshold function. Any value of \gls{sd} that fell below the threshold was accepted as a pause, as seen in Equation \ref{eq:thresh}. 
    
    \begin{equation}
        P_n = \left\{
        \begin{aligned}
            true\qquad if SD_n < thresh\\
            false\qquad if SD_n \geq thresh
        \end{aligned}
        \right.
        \label{eq:thresh}
        \myequations{Threshold for Detecting Pause in Pedestrian Movement}
    \end{equation}
    \\

    Fine-tuning of detection was performed by calculating the duration of each detected pause. If the detected pause duration was less than 3 $seconds$, it was eliminated from the detected pauses. The elimination of pauses shorter than 3 $seconds$ prevented the effect of any remnant of anomalous data in the interpolated polynomial curve from causing false positives. Each of the 108 sets of \gls{sd} was tested to check the number of correct pause detections and false positive pause detections for each given sliding window size, varying from 2 $seconds$ to 10 $seconds$. The outcomes of that process are presented in Figure \ref{fig:parallelplot}. The coloured lines represent the size of the sliding window. Each line connects to the number of false positives identified and the correct pause detection with that chosen sliding window size.
\\

    \begin{figure}[htbp]
        \centerline{\includegraphics[width=\textwidth]{Figures/Ch4/parallel_plot.jpg}}
        \caption{Sliding Window Size Efficacy Through Parallel Coordinate Plot}
        \label{fig:parallelplot}
    \end{figure}

    To select the optimum size of the window, counts of false positives and correct detection were analysed for each window size. The window size that presented the least false positives and correct detection for the most number of runs is selected. A chart depicting both the false and correct detections is presented in Figure \ref{fig:trend_fps_dets}.
\\

    \begin{figure}[htbp]
        \centerline{\includegraphics[width=\textwidth]{Figures/Ch4/trend_fps_dets.jpg}}
        \caption{Curve Fitting for Example \gls{rss} Values}
        \label{fig:trend_fps_dets}
    \end{figure}

    In this case, a sliding window with a size of 9 $seconds$ provided the best balance between the two conflicting outcomes. Therefore, the final results were verified by plotting the elapsed time against \gls{rssi} overlaid with the period marked where the 9-$second$ sliding window \gls{sd} detected flatness.
    
    \item \textit{Ground Truth}: To validate whether the pause detected by our technique coincided with the point in time that the pause actually occurred, the timestamp of the button the volunteer pedestrian pressed on the \textit{Blue Dot} app was used to confirm the volunteer's pause and the resumption of their journey. These pauses are marked on the final graph obtained from the previous step to validate if the pause detected by the algorithm coincides with the actual pause of the pedestrian. This is depicted in Figure \ref{fig:outcome}.
\\

    \begin{figure}[hb!]
        \centering
        \begin{subfigure}{0.95\textwidth}
            \includegraphics[width=\textwidth]{Figures/Ch4/filter5s_9 (1).jpg}
            \caption{Detection in 5 $Seconds$ Pause, All Cases}
        \end{subfigure}
    
        \begin{subfigure}{0.95\textwidth}
            \includegraphics[width=\textwidth]{Figures/Ch4/filter15s_9 (1).jpg}
            \caption{Detection in 15 $Seconds$ Pause, All Cases}
        \end{subfigure}
    \end{figure}    
    \begin{figure}[!ht]\ContinuedFloat
        \centering
        \begin{subfigure}{0.95\textwidth}
            \includegraphics[width=\textwidth]{Figures/Ch4/filter25s_9 (1).jpg}
            \caption{Detection in 25 $Seconds$ Pause, All Cases}
        \end{subfigure}
        
        \caption{Final Outcome of the Pause Detection Technique}
        \label{fig:outcome}
    \end{figure}

    With a window size of 9 $seconds$, the highest rate of detection of pauses was observed, in 8 out of 12 cases. For this window size, the lowest number of false pause detections were also observed. It is essential to note that even if many patches of below threshold \gls{sd} values were identified within the actual pause duration, only one correct pause detection was added to the tally since the actual pause happened only once. All the instances of false pause detection were added up in the final tally. This further highlighted that the analytical approach resulted in a low number of false positive detections for this experimental setup.
\\

    It is also important to note that a pause was not detected at all when the pedestrian was instructed to stop for 5 $seconds$. This could be due to the stray signals from before the pause happened, overlapping the signals obtained during the pause. And since, the fine-tuning ruled out any pause length of less than 3 $seconds$ duration, it was difficult to identify a pause duration this short. For a 15-$second$ pause, a 100\% pause detection and 0 false detections for a window size of both 8 and 9 were observed. Therefore, a recommendation of detecting a pause duration of 15 seconds or over is suggested by this experiment. This approach indicates that it cannot be used to accurately detect pauses of less than 15 $seconds$ duration, at least for the experimental configuration used in the presented experiment. Its general applicability assessment requires further investigation. This is also in line with the "smudging" effect outlined in the literature review due to fading in WiFi. Since the two technologies operate in the same radio frequency band, \gls{ble} is also prone to this effect, however, the high transmission power of WiFi makes it more susceptible to multipath fading in comparison to \gls{ble}.
    \\

    One limitation of this approach is that there is no surety of pause detection, even at a greater pause duration. Elements present in the physical environment may amplify the reflection of radio signals and/or multipath propagation which may lead to an increased number of stray signals or repeated broadcasts reception. Therefore, a survey of the environment is essential before using this technique. Also, using polynomial curve estimation is a technique that requires further investigation. Identifying the best fit in estimated curves requires thorough research to prevent overfit and underfit, especially when the \gls{rss} values are obtained for an extended period. It should be emphasised that while a lower \gls{sse} indicates a better fit, closely aligning the curve to every data point may be ineffective in mitigating the effect of anomalous data on the \gls{rssi} trend. Additionally, the curves obtained through polynomial fitting consistently commence and conclude at the initial and final \gls{rss} values, respectively. Consequently, when there is a substantial time gap and a significant difference between the starting and ending \gls{rss} values and their adjacent \gls{rss} values, the resulting interpolated curve introduces artefacts that deviate from the actual trend exhibited by the \gls{rss} values as seen in Figure \ref{fig:polyfit_overlay}.
    
\end{enumerate}

\vspace{10pt}

    \begin{tcolorbox}[
        colframe=white, % Border color
        colback=Ivory, % Background color
        coltitle=white, % Title text color
        title=Section Summary, % Title text
        fonttitle=\bfseries, % Title font style
        sharp corners, % Sharp corners for the main box
        enhanced,
        attach boxed title to top left={yshift=-2mm, xshift=2mm}, % Positioning the title
        boxed title style={colback=DarkSlateGray, colframe=SlateBlue, rounded corners=west, boxrule=0pt, top=2mm, bottom=2mm, right=2mm}, % Title box style
        width=\linewidth, % Width of the box
        boxrule=0pt, % No border for the main box
        drop shadow, % Shadow effect
        rounded corners, % Rounded corners for the main box
    ]
        The results provide concrete evidence for the possibility of identifying pauses in the movement of the pedestrians. This can be associated with the interaction of the pedestrian with the environment or with other pedestrians. A multi-fold analysis is carried out to detect pauses. A curve is first fitted to the collected \gls{rssi}, for which, a variety of degrees of polynomials are tested. Once a fitting polynomial is identified, that is a polynomial that results in the least \gls{sse}, a sliding window \gls{sd} is applied. For the sliding window \gls{sd}, different window sizes are first tested. The optimal window size is chosen through those tests. Subsequently, a threshold on the evaluated \gls{sd}s is applied to identify pauses. Finally, the duration of identified pauses is again subjected to a threshold to determine a pause in the movement of a pedestrian.
        \\
        
        The approach however is not sufficiently robust to identify shorter pauses in the duration, as seen in the results. Therefore, detection of pauses of 15 $seconds$ and over is recommended. Shorter pauses may not actually attribute to significant interaction with the space or other pedestrians regardless, and therefore, the inability of this approach to detect shorter pauses is not unfavourable because shorter pauses could mean that the pedestrian has stopped for some trivial reasons, such as tying their shoelaces or picking up something that fell or simply, the result of a set of anomalous readings. This study can be further extended with multiple \gls{ble} devices to detect whether two devices pause in the vicinity at the same time for the same duration. Such information would increase confidence in the assertion that the pedestrians could be interacting with each other.
\\

    The work described in this section is presented in a paper titled, "Detection of Pause in a Pedestrian's Movement on a Linear Walkway using Bluetooth Low Energy Received Signal Strength Indicator" \citep{Parmar2023b}.
    \end{tcolorbox}


\subsection{Identifying Interactions with Other Pedestrians or the Environment} \label{subsec:res/interaction}


\begin{figure}[htbp]
	\centerline{\includegraphics[width=\textwidth]{Figures/Ch4/timeline/timeline_interaction.png}}

\end{figure}

This experiment is an extension of the previous experiment \ref{res/pause}. In the previous experiment, a pause in the movement of one pedestrian is asserted with a multi-fold analysis encompassing curve interpolation, sliding-window \gls{sd}, and thresholding. The strategy revealed that it is likely to detect pauses of duration over 15 $seconds$ in a walk using this approach. However, the strategy may not be suitable for detecting the pauses of two pedestrians simultaneously using this approach since the presence of multiple \gls{ble} advertisements may result in packet collisions. To test this hypothesis, in this experiment, two volunteer pedestrians were instructed to walk simultaneously on the pathway 3 $metres$ away from the Observer, starting from \textit{start} point and walking towards the \textit{end} point. Pauses are tested at three key points, \textit{approach}, \textit{centre} or \textit{depart}. At each of these locations, three configurations were tested -- Broadcasters held in \gls{los} of the Observer by both volunteers, Broadcasters held in \gls{nlos} of the Observer by both volunteers and one Broadcaster held in \gls{los} and the other in \gls{nlos} of the Observer. \gls{gps} measurements were acquired by one of the volunteers through an Android application called GPSLogger.
\\

All the measurements for this experiment were taken on April 9, 2024 between 13:29 and 14:33 hours Irish time with the support from two volunteer pedestrians. The weather information at 13:00 and 14:00 hours on the day of the experiment is following:
\\

\noindent\textbf{At 13:00 hours on April 9, 2024}
\begin{itemize}
    \item \textbf{Precipitation (Rain):} $0.0\,\text{mm}$
    \item \textbf{Air Temperature:} $11.8\,^\circ\mathrm{C}$
    \item \textbf{Wet Bulb Temperature:} $7.5\,^\circ\mathrm{C}$
    \item \textbf{Dew Point Temperature:} $1.8\,^\circ\mathrm{C}$
    \item \textbf{Vapour Pressure:} $7\,\text{hPa}$
    \item \textbf{Relative Humidity:} $50\,\%$
    \item \textbf{Mean Sea Level Pressure:} $1012.2\,\text{hPa}$
\end{itemize}

\vspace{1em}
\noindent\textbf{At 14:00 hours on April 9, 2024}
\begin{itemize}
    \item \textbf{Precipitation (Rain):} $0.0\,\text{mm}$
    \item \textbf{Air Temperature:} $12.6\,^\circ\mathrm{C}$
    \item \textbf{Wet Bulb Temperature:} $7.9\,^\circ\mathrm{C}$
    \item \textbf{Dew Point Temperature:} $1.7\,^\circ\mathrm{C}$
    \item \textbf{Vapour Pressure:} $6.9\,\text{hPa}$
    \item \textbf{Relative Humidity:} $47\,\%$
    \item \textbf{Mean Sea Level Pressure:} $1013\,\text{hPa}$
\end{itemize}

The collected \gls{rssi} values were cycled through the same curve interpolation and sliding-window \gls{sd} techniques as in the pause detection experiment but, for advertisements for both the Broadcasters separately. The threshold for the \gls{sd} was chosen to be lower for at least 5 $seconds$. Since the pause duration here is 20 $seconds$, and the advertisement rate is 1.28 $seconds$, the number of advertisements that can be intercepted during the pause period of 20 $seconds$ is approximately 15. And, to account for packet collision and packet losses, the thresholding of 20 $seconds$, for just as long as the pause duration, was not viable, as also discussed previously in the Section \ref{res/pause}.
\\

The geolocations acquired from the GPSLogger application were plotted using geoplot and the elapsed time from the start of the walk were posted next to each marker on the geoplot. On the subplot, \gls{rssi} was graphed against elapsed time. This allows for easy visual inspection of the \gls{rssi} against the ground truth location of the pedestrians. The entire process was applied to each of  the 27 individual runs. Here in this dissertation however, only one representative round for each unique case is presented. Figures \ref{fig:res/interaction/approach_ll}, \ref{fig:res/interaction/approach_nl}, and \ref{fig:res/interaction/approach_nn} represent the charts for the measurements carried out where pedestrians stopped at \textit{approach} location when both Broadcasters were in \gls{los} orientation, where one Broadcaster was in \gls{nlos} and the other in \gls{los}, and where both Broadcasters were \gls{nlos} of the Observer respectively. The red and blue line-marker combination is used to differentiate the data captured from the two Broadcasters. The phase corresponding to the actual pause in the walk is demarcated by two vertical black lines, whereas, the pause detection through the multi-fold analysis is presented in the red and blue regions on the chart. The overlap of the two regions is magenta in colour.
\\

In Figures \ref{fig:res/interaction/approach_ll}, \ref{fig:res/interaction/approach_nl}, and \ref{fig:res/interaction/approach_nn}, a pause is detected successfully only the \gls{los}-\gls{los} orientation of the two Broadcasters. However, even in the case of successful detection, the pause region exceeded the boundary demarcated by the black lines.
\\

\begin{figure}[htbp]
	\centerline{\includegraphics[width=\textwidth]{Figures/Ch4/evaluation/interaction/approach/ll1.jpg}}
	\caption{Geolocation and \gls{rssi} vs Elapsed Time Plot when Pedestrians Paused at \textit{Approach} Point, Both Broadcasters in \gls{los}}
	\label{fig:res/interaction/approach_ll}

\end{figure}

\begin{figure}[htbp]
	\centerline{\includegraphics[width=\textwidth]{Figures/Ch4/evaluation/interaction/approach/nl1.jpg}}
	\caption{Geolocation and \gls{rssi} vs Elapsed Time Plot when Pedestrians Paused at \textit{Approach} Point, One Pedestrian Carrying Broadcasters in \gls{los} and Other in \gls{nlos}}
	\label{fig:res/interaction/approach_nl}

\end{figure}

\begin{figure}[!t]
	\centerline{\includegraphics[width=\textwidth]{Figures/Ch4/evaluation/interaction/approach/nn1.jpg}}
	\caption{Geolocation and \gls{rssi} vs Elapsed Time Plot when Pedestrians Paused at \textit{Approach} Point, Both Broadcasters in \gls{nlos}}
	\label{fig:res/interaction/approach_nn}

\end{figure}
\FloatBarrier
Similarly, Figures \ref{fig:res/interaction/centre_ll}, \ref{fig:res/interaction/centre_nl}, and \ref{fig:res/interaction/centre_nn} represents pausing at the \textit{centre} key point on the pathway in \gls{los}-\gls{los}, \gls{los}-\gls{nlos}, and \gls{nlos}-\gls{nlos} orientations respectively. In this case, in the \gls{los}-\gls{los} orientation, a pause was detected for one advertiser, whereas, in the other two, pause was detected for both the Broadcasters but overshooting the actual pause period.
\\
\begin{figure}[htbp]
	\centerline{\includegraphics[width=\textwidth]{Figures/Ch4/evaluation/interaction/centre/ll1.jpg}}
	\caption{Geolocation and \gls{rssi} vs Elapsed Time Plot when Pedestrians Paused at \textit{Centre} Point, Both Broadcasters in \gls{los}}
	\label{fig:res/interaction/centre_ll}

\end{figure}

\begin{figure}[htbp]
	\centerline{\includegraphics[width=\textwidth]{Figures/Ch4/evaluation/interaction/centre/nl1.jpg}}
	\caption{Geolocation and \gls{rssi} vs Elapsed Time Plot when Pedestrians Paused at \textit{Centre} Point, One Pedestrian Carrying Broadcasters in \gls{los} and Other in \gls{nlos}}
	\label{fig:res/interaction/centre_nl}

\end{figure}

\begin{figure}[htbp]
	\centerline{\includegraphics[width=\textwidth]{Figures/Ch4/evaluation/interaction/centre/nn1.jpg}}
	\caption{Geolocation and \gls{rssi} vs Elapsed Time Plot when Pedestrians Paused at \textit{Centre} Point, Both Broadcasters in \gls{nlos}}
	\label{fig:res/interaction/centre_nn}

\end{figure}
\FloatBarrier
Finally, for pause at the \textit{depart} location, a pause was detected for both the Broadcasters in all of the orientations, as seen in Figures \ref{fig:res/interaction/depart_ll}, \ref{fig:res/interaction/depart_nl}, and \ref{fig:res/interaction/depart_nn}. However, the pause period spilled over the actual pauses demarcated even in this case.
\\

\begin{figure}[!htbp]
	\centerline{\includegraphics[width=\textwidth]{Figures/Ch4/evaluation/interaction/depart/ll1.jpg}}
	\caption{Geolocation and \gls{rssi} vs Elapsed Time Plot when Pedestrians Paused at \textit{Depart} Point, Both Broadcasters in \gls{los}}
	\label{fig:res/interaction/depart_ll}

\end{figure}

\begin{figure}[!htbp]
	\centerline{\includegraphics[width=\textwidth]{Figures/Ch4/evaluation/interaction/depart/nl2.jpg}}
	\caption{Geolocation and \gls{rssi} vs Elapsed Time Plot when Pedestrians Paused at \textit{Depart} Point, One Pedestrian Carrying Broadcasters in \gls{los} and Other in \gls{nlos}}
	\label{fig:res/interaction/depart_nl}

\end{figure}

\begin{figure}[!htbp]
	\centerline{\includegraphics[width=\textwidth]{Figures/Ch4/evaluation/interaction/depart/nn1.jpg}}
	\caption{Geolocation and \gls{rssi} vs Elapsed Time Plot when Pedestrians Paused at \textit{Depart} Point, Both Broadcasters in \gls{nlos}}
	\label{fig:res/interaction/depart_nn}
\end{figure}
\FloatBarrier

From the results, it is clear that the approach identified in the previous experiment \ref{res/pause}, applies to most of the cases when multiple Broadcasters are used simultaneously. However, the output is not accurate enough to pinpoint the exact location of those pauses. Therefore, only a ballpark estimation of the location of the pauses can be estimated. There are many instances in the select cases presented above where the two pause patches overlap. This provides a likelihood of the interaction between the two pedestrians, or, with some element in the environment. 
\\

One possible reason for the extension of the detected pause regions could be the clashing advertisements in the same region. While this is not investigated in this PhD, such an investigation to understand the effects of multiple Broadcasters in the same region on the capabilities of the Observer and resulting \gls{rssi} will be useful in future work. 
\\

Moreover, even when the pause zones are detected, there is insufficient information to treat them as concrete evidence of interaction between the two pedestrians. This is because even when the pauses are identified for the same orientation of Broadcasters, say \gls{los}-\gls{los}, the values of respective \gls{rssi} in the detected pause zones are not closely related. For instance, in Figure \ref{fig:res/depdist/depart_los}, the two detected pause zones overlap, however, the \gls{rssi} of Broadcaster 1 in the pause zone is in the range of -67 $dB$ to -62 $dB$, whereas in the same zone, Broadcaster 2 has \gls{rssi} close to -55 $dB$. So, if there are no \gls{gps} data, it is difficult to predict that both pedestrians stopped at the same location, and their pauses at different locations on the same pathway at the same time could be a mere coincidence. This difference in the \gls{rssi} between the two Broadcaster is very likely because the two pedestrians are side-by-side, meaning that the advertisements from the Broadcaster held by the pedestrian on the farther side from the Observer have to travel through some form of partial occlusion caused by the other pedestrian. This is worse when the pedestrian at the farther side is holding the Broadcaster in \gls{nlos}, since now, the advertisements from that Broadcaster will interact with the bodies of the two pedestrians while traversing the path to the Observer. This is presented in Figure \ref{fig:res/interaction/obstruct}.  

\begin{figure}[htbp]
	\centerline{\includegraphics[width=\textwidth]{Figures/Ch4/evaluation/interaction/Drawing2.jpg}}
	\caption{Obstruction Faced by Advertisement Signals from Broadcaster 2}
	\label{fig:res/interaction/obstruct}

\end{figure}

\vspace{10pt}

    \begin{tcolorbox}[
        colframe=white, % Border color
        colback=Ivory, % Background color
        coltitle=white, % Title text color
        title=Section Summary, % Title text
        fonttitle=\bfseries, % Title font style
        sharp corners, % Sharp corners for the main box
        enhanced,
        attach boxed title to top left={yshift=-2mm, xshift=2mm}, % Positioning the title
        boxed title style={colback=DarkSlateGray, colframe=SlateBlue, rounded corners=west, boxrule=0pt, top=2mm, bottom=2mm, right=2mm}, % Title box style
        width=\linewidth, % Width of the box
        boxrule=0pt, % No border for the main box
        drop shadow, % Shadow effect
        rounded corners, % Rounded corners for the main box
    ]
        Following on from the previous experiment to detect pauses in the movement of a single pedestrian, Section \ref{res/pause} of this chapter, an extension to identify pauses in the movement of two pedestrians is carried out in this experiment. With the outcome of the previous experiment, only the likelihood of an interaction is inferred. However, if pauses can be identified in two or more Broadcasters simultaneously, there is an increased likelihood of asserting the interaction between pedestrians.
        \\
        
        The collected \gls{rssi} are subjected to the same treatment as in the case of the previous experiment. This involved using the curve fitting, sliding-window \gls{sd}, and thresholding. Since the investigation was already performed to identify the optimal polynomial, for this experimental setup, and the size of the sliding window previously, those parameters are not re-evaluated again. The results reveal the consistency of the approach as it can detect pauses for both the Broadcasters in most of the cases tested. It is however not possible to pinpoint the location of the pauses as the detected pause locations have extended beyond the actual pause zones due to the range of \gls{rssi} values differing from one another significantly in the pause regions. It is reasonable to believe that the presence of two Broadcasters in the same region also increases the chances of packet collision, subsequently leading to packet loss. However, this is not part of the investigation in this PhD.
    \end{tcolorbox}

\subsection{Asserting the Direction of Travel} \label{sec:res/direction}

\begin{figure}[htbp]
	\centerline{\includegraphics[width=\textwidth]{Figures/Ch4/timeline/timeline_direction.png}}

\end{figure}

The learning from \ref{subsec:res/antenna} was employed in this experiment to identify the direction of travel of the volunteer pedestrian. Through that experiment, it was found that the antenna on the Observer used in this study was more sensitive towards signals emerging from the \textit{approach} region or the 45\textdegree{} region with respect to the horizontal plane of the Observer. Therefore, if the entire journey, or in other words, all the \gls{rss} values collected throughout the journey of the pedestrian are divided into two parts, the stronger reception of the signals, as represented by the collected \gls{rssi}, should be obtained towards the side of the journey that happened around the 45\textdegree{} region along the plane of the Observer antenna. 
\\

Data for this experiment was collected on May 25, 2023 between 11:35 and 12:27 hours Irish time with the support from a single volunteer pedestrian. The weather information at 11:00 and 12:00 hours on the day of the experiment is following:
\\

\noindent\textbf{At 11:00 hours on May 25, 2023}
\begin{itemize}
    \item \textbf{Precipitation (Rain):} $0.0\,\text{mm}$
    \item \textbf{Air Temperature:} $17.9\,^\circ\mathrm{C}$
    \item \textbf{Wet Bulb Temperature:} $11.7\,^\circ\mathrm{C}$
    \item \textbf{Dew Point Temperature:} $5.0\,^\circ\mathrm{C}$
    \item \textbf{Vapour Pressure:} $8.7\,\text{hPa}$
    \item \textbf{Relative Humidity:} $42\,\%$
    \item \textbf{Mean Sea Level Pressure:} $1033.2\,\text{hPa}$
\end{itemize}

\vspace{1em}
\noindent\textbf{At 12:00 hours on May 25, 2023}
\begin{itemize}
    \item \textbf{Precipitation (Rain):} $0.0\,\text{mm}$
    \item \textbf{Air Temperature:} $18.8\,^\circ\mathrm{C}$
    \item \textbf{Wet Bulb Temperature:} $11.9\,^\circ\mathrm{C}$
    \item \textbf{Dew Point Temperature:} $4.4\,^\circ\mathrm{C}$
    \item \textbf{Vapour Pressure:} $8.4\,\text{hPa}$
    \item \textbf{Relative Humidity:} $38\,\%$
    \item \textbf{Mean Sea Level Pressure:} $1032.9\,\text{hPa}$
\end{itemize}



The experiment was divided into two cases, first, where the Observer and Broadcaster were in \gls{los}, and second, where the two devices were in \gls{nlos}. Each scenario was repeated three times and on two pathways, situated 3 $metres$ and 5 $metres$ away from the deployed Observer. Also, two scenarios were devised, first where the pedestrian started walking from the \textit{start} point to the \textit{end} point and second, where the journey was started from the \textit{end} point to the \textit{start} point. Every case and scenario was repeated 3 times. The \textit{Blue Dot} app was used to collect the actual location of the volunteer pedestrian as ground truth. The location obtained from \textit{Blue Dot} was used to divide the collected \gls{rss} values into two segments. 
\\

The collected \gls{rss} values for each half of the journeys were averaged and compared to identify whether the reception of the signals was stronger on the side of the path where the antenna of the Observer was more sensitive or not. This result is summarised in Table \ref{tab:res/direction/mean_rssi}.

\begin{table}[htbp]

\centering
    \arrayrulecolor{DarkOliveGreen} % Set the color of table rules    
    \begin{threeparttable}
    
    \begin{tabular}{
    >{\columncolor{MintCream}}c|  
    >{\columncolor{MintCream}}c 
    >{\columncolor{MintCream}}c|
    >{\columncolor{MintCream}}c  
    >{\columncolor{MintCream}}c|>{\columncolor{MintCream}}c  
    >{\columncolor{MintCream}}c}
        \hline 
        \rowcolor{MidnightGreen}
        \multicolumn{7}{|c|}{\textcolor{white}{\textbf{Mean \gls{rssi} from Start to End region}}} \\
        \hline
        \rowcolor{MidnightGreen}
        \textcolor{white}{\textbf{Case}} & \multicolumn{2}{|c|} {\textcolor{white}{Rep 1 (dB)}} & \multicolumn{2}{|c|}{\textcolor{white}{Rep 2 (dB)}} & \multicolumn{2}{|c|}{\textcolor{white}{Rep 3 (dB)}} \\
\hline 
&  \tiny{\textbf{S $\rightarrow$ C $^{\mathrm{1}}$}} &  \tiny{\textbf{C $\rightarrow$ E$^{\mathrm{2}}$}} & \tiny{\textbf{S $\rightarrow$ C}} &  \tiny{\textbf{C $\rightarrow$ E}} & \tiny{\textbf{S $\rightarrow$ C}} & \tiny{\textbf{C $\rightarrow$ E}}\\
\hline
\textbf{3m \gls{los}} & -57.68 & -64.47 & -61.82 & -67.75 & -63.53 & -59.88\\
\textbf{3m \gls{nlos}} & -60.25 & -76.75 & -63.38 & -73.13 & -65.25 & -66.50\\
\textbf{5m \gls{los}} & -59.97 & -64.27 & -65.64 & -64.66 & -59.23 & -65.47\\
\textbf{5m \gls{nlos}} & -67.61 & -65.78 & -65.83 & -68.28 & -66.96 & -70.23\\
\hline

\rowcolor{MidnightGreen}
        \multicolumn{7}{|c|}{\textcolor{white}{\textbf{Mean \gls{rssi} from End to Start region}}} \\
\hline 
\rowcolor{MidnightGreen}
        \textcolor{white}{\textbf{Case}} & \multicolumn{2}{|c|} {\textcolor{white}{Rep 1 (dB)}} & \multicolumn{2}{|c|}{\textcolor{white}{Rep 2 (dB)}} & \multicolumn{2}{|c|}{\textcolor{white}{Rep 3 (dB)}} \\
\hline
&  \tiny{\textbf{E $\rightarrow$ C$^{\mathrm{3}}$}} &  \tiny{\textbf{C $\rightarrow$ S$^{\mathrm{4}}$}} & \tiny{\textbf{E $\rightarrow$ C}} &  \tiny{\textbf{C $\rightarrow$ S}} & \tiny{\textbf{E $\rightarrow$ C}} & \tiny{\textbf{C $\rightarrow$ S}}\\
\hline
\textbf{3m \gls{los}} & -68.93 & -64.16 & -64.19 & -59.58 & -77.77 & -63.88\\
\textbf{3m \gls{nlos}} & -75.42 & -69.28 & -71.16 & -67.22 & -74.38 & -62.54\\
\textbf{5m \gls{los}} & -63.33 & -61.60 & -57.29 & -67.17 & -65.81 & -63.41\\
\textbf{5m \gls{nlos}} & -80.53 & -70.40 & -66.47 & -67.38 & -72.29 & -68.59\\

\hline
\hline

\end{tabular}
\begin{tablenotes}
    \item \tiny{$^{\mathrm{1}}$ Start $\rightarrow$ Centre.}
    \item \tiny{$^{\mathrm{2}}$ Centre $\rightarrow$ Start.}
    \item \tiny{$^{\mathrm{3}}$ End $\rightarrow$ Centre.}
    \item \tiny{$^{\mathrm{4}}$ Centre $\rightarrow$ Start.}
\end{tablenotes}
\caption{Mean \gls{rss} Value Between the Two Parts of the Journey for Each Deployment Distance for Both \gls{los} and \gls{nlos} Cases}
\label{tab:res/direction/mean_rssi}

\end{threeparttable}
\end{table}

Looking at the pedestrian walks from the \textit{start} point to the \textit{end} point in the case where the path is 3 $metres$ away from the Observer, it was observed that there was only one journey, in both \gls{los} and \gls{nlos} scenarios combined, that had a higher mean \gls{rss} value in the region between the \textit{centre} point and the \textit{end} point. Further, averaging all of the mean \gls{rss} values for both scenarios combined, categorised by the region in which the walk took place, journey between the \textit{start} point to the \textit{centre} point were observed to have an average \gls{rss} value of -61.85 $dB$ and that between the \textit{centre} point to the end point had an average \gls{rss} value of -67.64 $dB$. This was consistent with expectations based on the results obtained from the experiment conducted in \ref{subsec:res/antenna} presented in this chapter.
\\

Similarly, at a deployment distance of 5 $metres$, two occurrences of higher mean \gls{rss} values in the region between the \textit{centre} and \textit{end} point out of six total journeys were observed. The average of the mean \gls{rss} values between the start to the centre region was -64.20 $dB$, whereas, the equivalent for the centre-to-end part of the journey was -66.45 $dB$. Again, this was consistent with the results obtained at the 3-$metre$ deployment distance.
\\

For journeys that started at the \textit{end} point and finished at the \textit{start} point, the outcome was similar. At a deployment distance of 3 $metres$, there was no instance out of the six repetitions combined across \gls{los} and \gls{nlos} cases where the mean \gls{rss} value during the journey between the \textit{end} point and the \textit{centre} point was higher than that of the journey between the \textit{centre} point and \textit{start} point. The average of mean \gls{rssi} for the former was -70.98 $dB$, whereas that for the latter was -64.44 $dB$. At a distance of 5 $metres$, there were two occasions out of six where the mean \gls{rss} values for a journey between the \textit{end} to the \textit{centre} point was greater than that at the \textit{centre} to the \textit{start} point. The average of those means had a marginal difference but nevertheless, was in favour of the journey between the \textit{centre} and the \textit{start} region with the value of -66.42 $dB$ against -67.62 $dB$. 
\\

Out of the combined 24 cases, a higher mean \gls{rss} value was found between the centre and end region 5 times, that is for 20.83\% of all the cases. This implies that there was a high likelihood, 79.17\% to be exact, of identifying the direction of travel of the pedestrian through simple mean calculation of the collected \gls{rss} values. Looking at the charts, however, the double hump pattern discussed in the earlier experiment, in Section \ref{subsec:res/antenna} in this chapter, was a rarity outside of the anechoic chamber. This could be due to the strong \gls{los} component found at the \textit{approach} and \textit{centre} key points and the effect of shadowing observed at \textit{depart} key point, as was identified in the outcome of experiment presented in Figure \ref{fig:res/fading/ls} and Table \ref{tab:ls_fading} of Section \ref{sec:res/fading} in this chapter. Figures \ref{fig:3m_sma}, \ref{fig:3m_sma_occ}, \ref{fig:5m_sma}, and \ref{fig:5m_sma_occ} depict the chart of \gls{rssi} against elapsed time, with pseudo-names of the location on the path mentioned in the x-axis obtained through the Blue Dot app, for the 3 $metres$ deployment distance \gls{los}, 3 $metres$ deployment distance \gls{nlos}, 5 $metres$ deployments distance \gls{los} and 5 $metres$ deployment distance \gls{nlos}. On the second y-axis, the charts also depict the number of advertisements obtained per second during the walk between each subsequent location on the path. It can be seen on those figures that the advertisement rate at the Observer dropped in the case of \gls{nlos}. This result was also consistent with the results obtained from the experiment to test the effect of body occlusion, described in Section \ref{subsec:res/occlusion} in this chapter. The averaged duration, samples, and sample count for each distance and case are presented in Table \ref{tab:duration_samples}

\begin{table}[htbp]

\begin{center}
\begin{tabular}{
    >{\columncolor{MintCream}}c|  
    >{\columncolor{MintCream}}c 
    >{\columncolor{MintCream}}c
    >{\columncolor{MintCream}}c }
        \hline 
        \rowcolor{MidnightGreen}
        \textbf{\textcolor{white}{Locations}}&\multicolumn{3}{c}{\textbf{\textcolor{white}{3 $metres$ \gls{los}}}} \\
        \cline{2-4} 
        \rowcolor{MidnightGreen}
        & \textcolor{white}{Duration (s)}& \textcolor{white}{Samples} & \textcolor{white}{Samples per second} \\        
        \hline
\textbf{Start $\rightarrow$ Approach} & 5.40 & 5.25 & 0.97\\
\textbf{Approach $\rightarrow$ Centre} & 5.06 & 4.16 & 0.82 \\
\textbf{Centre $\rightarrow$ Depart} & 5.52 & 5.16 & 0.93\\
\textbf{Depart $\rightarrow$ End} & 5.58 & 5.08 & 0.91\\
\hline
\rowcolor{MidnightGreen}&\multicolumn{3}{c}{\textbf{\textcolor{white}{3 $metres$ \gls{nlos}}}} \\
\hline
\textbf{Start $\rightarrow$ Approach} & 5.06 & 3.92 & 0.77\\
\textbf{Approach $\rightarrow$ Centre} & 4.88 & 3.92 & 0.80 \\
\textbf{Centre $\rightarrow$ Depart} & 5.06 & 4.50 & 0.89\\
\textbf{Depart $\rightarrow$ End} & 4.85 & 3.50 & 0.72\\
\hline
\rowcolor{MidnightGreen}&\multicolumn{3}{c}{\textbf{\textcolor{white}{5 $metres$ \gls{los}}}} \\
\hline
\textbf{Start $\rightarrow$ Approach} & 5.01 & 4.58 & 0.91\\
\textbf{Approach $\rightarrow$ Centre} & 4.95 & 4.67 & 0.94 \\
\textbf{Centre $\rightarrow$ Depart} & 5.14 & 5.08 & 0.99\\
\textbf{Depart $\rightarrow$ End} & 4.78 & 3.3 & 0.70\\
\hline
\rowcolor{MidnightGreen}&\multicolumn{3}{c}{\textbf{\textcolor{white}{5 $metres$ \gls{nlos}}}} \\
\hline
\textbf{Start $\rightarrow$ Approach} & 5.64 & 4.25 & 0.75\\
\textbf{Approach $\rightarrow$ Centre} & 5.18 & 4.5 & 0.87 \\
\textbf{Centre $\rightarrow$ Depart} & 5.57 & 4.75 & 0.85\\
\textbf{Depart $\rightarrow$ End} & 5.34 & 3.83 & 0.72\\
\hline
\hline
\end{tabular}
\caption{Mean Duration and Samples at Each Deployment Distance for Both \gls{los} and \gls{nlos} Cases}
\label{tab:duration_samples}
\end{center}
\end{table}

\begin{figure}[htbp]
	\centerline{\includegraphics[width=\textwidth]{Figures/Ch4/3m_sma (2).jpg}}
	\caption{RSSI values of a Broadcaster in \gls{los} of Observer when the Pedestrian is Walking on a Pathway 3-$metres$ Away, \gls{los}}
	\label{fig:3m_sma}
\end{figure}

\begin{figure}[htbp]
	\centerline{\includegraphics[width=\textwidth]{Figures/Ch4/3m_sma_occ (2).jpg}}
	\caption{RSSI values of a Broadcaster is in \gls{nlos} of Observer when the Pedestrian is Walking on a Pathway 3-$metres$ Away}
	\label{fig:3m_sma_occ}
\end{figure}

\begin{figure}[htbp]
	\centerline{\includegraphics[width=\textwidth]{Figures/Ch4/5m_sma (1).jpg}}
	\caption{RSSI values of a Broadcaster in \gls{los} of Observer when the Pedestrian is Walking on a Pathway 5-$metres$ Away}
	\label{fig:5m_sma}
\end{figure}

\begin{figure}[htbp]
	\centerline{\includegraphics[width=\textwidth]{Figures/Ch4/5m_sma_occ (1).jpg}}
	\caption{RSSI values of a Broadcaster is in \gls{nlos} of Observer when the Pedestrian is Walking on a Pathway 5-$metres$ Away}
	\label{fig:5m_sma_occ}
\end{figure}

\vspace{10pt}

    \begin{tcolorbox}[
        colframe=white, % Border color
        colback=Ivory, % Background color
        coltitle=white, % Title text color
        title=Section Summary, % Title text
        fonttitle=\bfseries, % Title font style
        sharp corners, % Sharp corners for the main box
        enhanced,
        attach boxed title to top left={yshift=-2mm, xshift=2mm}, % Positioning the title
        boxed title style={colback=DarkSlateGray, colframe=SlateBlue, rounded corners=west, boxrule=0pt, top=2mm, bottom=2mm, right=2mm}, % Title box style
        width=\linewidth, % Width of the box
        boxrule=0pt, % No border for the main box
        drop shadow, % Shadow effect
        rounded corners, % Rounded corners for the main box
        breakable,
    ]

        Detection of direction is a useful measure to understand the usage pattern of a pathway and to characterise the usage requirements of the pedestrians from the pathway. It is an easier task to solve with \gls{ble} by using multiple Observers and placing them strategically such that the strength of signals in one Observer is faded when the pedestrian is close to the other Observer and vice versa. However, achieving this with a single Observer is a challenge. While modern versions of \gls{ble}, version 5.1 and onwards inherently provide this capability by detecting the angle of arrival of signals, previous versions of \gls{ble} lack this functionality. Although, \gls{ble} 5.1 was released in January 2019 \citep{Bluetooth5.1}, the most recent version, version 5, of one of the market leading \gls{sbc}, \gls{rpi} which was released in 2023, comes with \gls{ble} v5.0. Throughout the course of this experiment, from conceptualised to publication, the most recent \gls{rpi} device was version 4B, which used \gls{ble} version 4.0. I conceptualised this research experiment based on the results of the antenna characteristics experiment presented in section \ref{subsec:res/antenna}. The results of those experiments revealed the sensitivity of the antenna on the Observer towards the signals arising between 0\textdegree{} and 90\textdegree{} region, and therefore, this characteristic of the antenna favoured signals from one direction over the other.
        \\
        
        The pattern in the \gls{rss} values here is not apparent when the plots of those values against time are observed and the results look inconclusive. However, some basic statistics unearth nuances that are useful for forming assertions. By dividing each journey into phases, travelling to and from the \textit{start} and \textit{centre}, and to and from the \textit{centre} and \textit{end}, we are able to see that the mean of the \gls{sma}-filtered \gls{rss} values is higher for travelling in either direction between the \textit{start} and \textit{centre} region. If we know the orientation of the Observer and the topology of the surrounding environment, through the temporal location of the onset of that local peak, we can identify the direction of travel. In other words, a sharp climb to the local peak \gls{rss} value is followed by a long tail with the possibility of a second lesser peak (double hump) when the direction of travel is from start to end. And, a slow climb with a lesser peak followed by a local peak and a sharp decline when the travel direction is reversed. Moreover, as learnt  from the results of the previous experiments \ref{subsec:res/occlusion}, the sample rate identified in Table \ref{tab:duration_samples} is also indicative of interference between the Observer and the Broadcaster as the number of samples per second in the \gls{nlos} case is consistently lesser compared to the \gls{los}case. While the likelihood of asserting the direction of travel of a pedestrian in a linear pathway using a single Observer by exploiting the characteristics of its antenna is proven through the results, the uncertainties in the environment affect the performance and hence the confidence of the assertions. The chosen pathway here is infrequently used and is devoid of large physical objects. If these factors are present, the likelihood of inference could result in less reliable inferences.
\\
\\

    The results of this experiment are presented in the paper titled, "Indication of pedestrian’s travel direction through Bluetooth Low Energy signals perceived by a single Observer device" \citep{Parmar2023}.
    \end{tcolorbox}
    

\subsection{Analysing the Behaviour of Pedestrians in a University Campus for an Extended Period} \label{sec:res/gg}


\begin{figure}[htbp]
	\centerline{\includegraphics[width=\textwidth]{Figures/Ch4/timeline/timeline_behaviour.png}}

\end{figure}

This experiment was a long-term study of pedestrian behaviour in a university campus in which a total of 17 Broadcasters (mix of indoor \gls{rpi}-based and outdoor ESP32-based) were employed. Measurements were obtained from 28 active volunteers, totalling 272 pedestrian journeys undertaken, and aggregated. The aggregated data was then analysed. Table \ref{tab:bleevents} presents counts of the \gls{ble} events or advertisements that are observed for 24 days of the experiment at each location by each volunteer pedestrian. The presented data appears rudimentary, lacking any statistical analysis, however, through this data, utilisation of the regions of the observed part of the campus was derived. For instance, the \textit{Pedestrian Pathway} reported the highest number of detected advertisements, meaning that it was widely used by volunteer pedestrians. It was important to note that just under half of these advertisements on the \textit{Pedestrian Pathway} result from the BLE Broadcaster carried by one particular volunteer. Even if this was consider as an outlier, a median of the count of advertisements, that can suppress the effect of outliers \citep{Wilcox2021}, suggests that the \textit{Pedestrian Walkway} was still the most frequently visited place on the campus by the participating volunteers. Similarly, the data in the table indicated that places like \textit{Constitution Hill}, \textit{Beresford}, \textit{Broadstone Luas}, \textit{Rathdown Store}, and \textit{Kirwan Street} were less visited by the volunteers.  \\

\begin{sidewaystable}[htbp]
    \centering
    \arrayrulecolor{DarkOliveGreen} % Set the color of table rules
    \scalebox{0.75}{
    % \begin{adjustbox}{max width=\textwidth}
    \begin{tabular}{|>{\columncolor{MintCream}}c|
                    >{\columncolor{MintCream}}c
                    >{\columncolor{MintCream}}c
                    >{\columncolor{MintCream}}c
                    >{\columncolor{MintCream}}c
                    >{\columncolor{MintCream}}c
                    >{\columncolor{MintCream}}c
                    >{\columncolor{MintCream}}c
                    >{\columncolor{MintCream}}c
                    >{\columncolor{MintCream}}c
                    >{\columncolor{MintCream}}c
                    >{\columncolor{MintCream}}c
                    >{\columncolor{MintCream}}c
                    >{\columncolor{MintCream}}c
                    >{\columncolor{MintCream}}c
                    >{\columncolor{MintCream}}c
                    >{\columncolor{MintCream}}c
                    >{\columncolor{MintCream}}c}
    \hline
\rowcolor{MidnightGreen} & \multicolumn{17}{|c}{\textcolor{white}{\textbf{Locations}}}\\
\cline{2-18}
\rowcolor{MidnightGreen}
\rot{\textcolor{white}{\textbf{Volunteer ID}}} & 
\rot{\textcolor{white}{\textbf{Constitution Hill}}} & 
\rot{\textcolor{white}{\textbf{Breseford}}} &
\rot{\textcolor{white}{\textbf{Broadstone LUAS}}} & 
\rot{\textcolor{white}{\textbf{GG LUAS}}} &
\rot{\textcolor{white}{\textbf{Clocktower Meeting}}} & 
\rot{\textcolor{white}{\textbf{Clocktower Office}}} &
\rot{\textcolor{white}{\textbf{Rathdown Office}}} & 
\rot{\textcolor{white}{\textbf{Rathdown Store}}} &
\rot{\textcolor{white}{\textbf{Northhouse Art}}} & 
\rot{\textcolor{white}{\textbf{Northhouse Annex}}} &
\rot{\textcolor{white}{\textbf{Blend Cafe}}} & 
\rot{\textcolor{white}{\textbf{Fingal Place}}} &
\rot{\textcolor{white}{\textbf{Pedestrian Pathway}}} & 
\rot{\textcolor{white}{\textbf{Kirwan St}}} &
\rot{\textcolor{white}{\textbf{GG by church}}} &
\rot{\textcolor{white}{\textbf{Parkhouse}}} &
\rot{\textcolor{white}{\textbf{HSE gate}}}\\
\hline
2 & 0 & 44 & 0 & 158 & 30 & 259 & 37 & 0 & 145 & 69 & 211 & 0 & 13 & 0 & 264 & 154 & 798 \\
3 & 0 & 4 & 0 & 0 & 2 & 8 & 9 & 0 & 103 & 22 & 179 & 0 & 19 & 0 & 55 & 105 & 346 \\
4 & 0 & 0 & 0 & 0 & 0 & 2 & 1 & 0 & 0 & 0 & 0 & 0 & 0 & 0 & 78 & 0 & 0 \\
5 & 181 & 3 & 175 & 3 & 19 & 90 & 71 & 0 & 12 & 426 & 0 & 18 & 885 & 6 & 512 & 3 & 0 \\
6 & 1 & 0 & 0 & 102 & 138 & 2009 & 781 & 54 & 846 & 2705 & 61 & 1497 & 2646 & 0 & 4909 & 121 & 77 \\
7 & 1 & 9 & 0 & 84 & 211 & 1980 & 1360 & 102 & 746 & 3486 & 450 & 502 & 2590 & 42 & 9352 & 84 & 494 \\
8 & 18 & 0 & 14 & 0 & 0 & 1 & 0 & 0 & 952 & 866 & 270 & 133 & 1513 & 0 & 500 & 1004 & 689 \\
9 & 0 & 0 & 0 & 0 & 0 & 0 & 0 & 0 & 0 & 0 & 0 & 0 & 147 & 0 & 0 & 0 & 0 \\
10 & 0 & 7 & 0 & 54 & 31 & 142 & 132 & 0 & 13 & 630 & 0 & 16 & 684 & 0 & 1243 & 0 & 0 \\
12 & 0 & 0 & 0 & 0 & 0 & 0 & 4 & 0 & 0 & 0 & 0 & 0 & 0 & 0 & 0 & 0 & 0 \\
13 & 81 & 0 & 112 & 24 & 41 & 344 & 30 & 37 & 0 & 19 & 0 & 0 & 2 & 0 & 8 & 0 & 9 \\
14 & 0 & 0 & 0 & 114 & 53 & 829 & 34 & 75 & 135 & 345 & 451 & 4 & 41 & 0 & 77 & 186 & 244 \\
15 & 1 & 0 & 0 & 13 & 0 & 23 & 16 & 0 & 4 & 80 & 0 & 11 & 132 & 53 & 127 & 2 & 0 \\
16 & 0 & 0 & 0 & 52 & 60 & 339 & 74 & 38 & 58 & 558 & 0 & 41 & 888 & 1 & 767 & 17 & 15 \\
17 & 56 & 0 & 2 & 47 & 18 & 293 & 13 & 7 & 53 & 292 & 77 & 35 & 251 & 0 & 167 & 157 & 175 \\
18 & 0 & 0 & 0 & 12 & 11 & 40 & 19 & 0 & 14 & 523 & 54 & 57 & 335 & 87 & 323 & 0 & 0 \\
19 & 0 & 1 & 0 & 24 & 4 & 57 & 39 & 0 & 4 & 185 & 0 & 12 & 1762 & 110 & 339 & 0 & 0 \\
20 & 0 & 2 & 1 & 72 & 30 & 179 & 53 & 24 & 23 & 609 & 0 & 45 & 640 & 0 & 753 & 53 & 34 \\
21 & 3 & 13 & 0 & 63 & 58 & 340 & 124 & 31 & 255 & 380 & 62 & 32 & 1004 & 9821 & 1401 & 133 & 102 \\
22 & 0 & 0 & 0 & 826 & 215 & 2004 & 299 & 212 & 909 & 2523 & 0 & 147 & 1486 & 0 & 786 & 36 & 123 \\
23 & 31 & 0 & 21 & 123 & 118 & 752 & 626 & 0 & 546 & 1721 & 68 & 282 & 14921 & 622 & 1981 & 50 & 29 \\
24 & 0 & 0 & 0 & 49 & 60 & 313 & 212 & 0 & 159 & 647 & 336 & 102 & 662 & 150 & 508 & 33 & 41 \\
25 & 3 & 5 & 8 & 442 & 39 & 354 & 94 & 20 & 219 & 490 & 253 & 82 & 1103 & 258 & 1251 & 47 & 98 \\
26 & 1 & 11 & 0 & 45 & 22 & 332 & 100 & 30 & 57 & 212 & 41 & 44 & 230 & 125 & 461 & 82 & 102 \\
27 & 0 & 0 & 0 & 11 & 12 & 64 & 33 & 0 & 19 & 221 & 67 & 84 & 336 & 61 & 302 & 0 & 0 \\
28 & 13 & 0 & 0 & 7 & 7 & 105 & 28 & 0 & 123 & 266 & 206 & 98 & 403 & 63 & 262 & 96 & 88 \\
29 & 0 & 0 & 0 & 5 & 3 & 31 & 13 & 0 & 15 & 193 & 0 & 0 & 112 & 33 & 126 & 4 & 9 \\
30 & 9 & 4 & 16 & 24 & 8 & 51 & 13 & 0 & 0 & 0 & 12 & 0 & 2 & 313 & 87 & 0 & 0 \\
\hline
\multirow{2}{4.7em}{\textbf{Total events}} & 399 & 103 & 349 & 2354 & 1190 & 10941 & 4215 & 630 & 5410 & 17468 & 2798 & 3242 & 32807 & 11745 & 26639 & 2367 & 3473\\
&&&&&&&&&&&&&&&&&\\
\hline
\multirow{3}{4.7em}{\textbf{Median events count}} & 0 & 0 & 0 & 34.5 & 20.5 & 160.5 & 35.5 & 0 & 55 & 318.5 & 47.5 & 33.5 & 369.5 & 3.5 & 331 & 34.5 & 31.5\\
&&&&&&&&&&&&&&&&&\\
\hline
    \end{tabular}
   % \end{adjustbox} % End of resizebox
   }
    \caption{Number of \gls{ble} Events for All Volunteers at Each Location.}
    \label{tab:bleevents}
\end{sidewaystable}

Moreover, the captured data also provided information at a finer level of granularity, for instance, Figure \ref{fig:journey_rssi} shows the aggregated \gls{rssi} record of one of the individual participating pedestrians for what appears to be a 30-$minute$ morning walk. The figure shows that the detected advertisements for this volunteer lead to a sequence of \gls{rss} values that reflect the journey of the said pedestrian. Bearing in mind that \gls{rssi} between -70 $dB$s and -80 $dB$s can be associated with a passing pedestrian at a distance of 10 to 20 $m$ from an Observer \citep{Alanbouri2019}, it was possible to infer the approximate trajectory of the pedestrian from the locations of the Observers.
\\

Figure \ref{fig:journey_rssi} depicts the inferred path of that pedestrian as an example. The numbers 1 through 6 on the figure are further represented on a map in Figure \ref{fig:journey}. From a comparison of the \gls{rssi} record and the campus map, the volunteer entered the campus through one of the main campus gates, and performed a circuit of one part of the campus over a period of 30 $minutes$ before exiting through the same gate. Another impactful estimation identified about a pedestrian's behaviour from this data was their pace of travel. Considering Figure \ref{fig:journey_rssi}, approximate time of proximity of the pedestrian to the Observers was identified. The approximate times when the pedestrian was at each of those locations were obtained from timestamps and were correlated against the distance between these locations. The distances between these locations were obtained using open mapping software. the probable pace of the pedestrian was then assessed with this information. Table \ref{tab:res/gg2} provides the distances between the locations traversed by the pedestrian, and based on the trajectory which passed through points 1 2 3 4 5 6 2 1, the volunteer travelled 1475 $m$ in approximately 30 $minutes$, resulting in a mean walking pace of 0.8 $m$ per $second$, indicating brisk walking pace.\\

\begin{figure}[htbp]
\centerline{\includegraphics[width=90mm]{Figures/Ch4/Picture2 (1).png}}
\caption{Scattered RSS of Observed Advertisement by Different Observers from a Single Volunteer Pedestrian Representing a Journey: Example 1.}
\label{fig:journey_rssi}
\end{figure}

\begin{figure}[htbp]
\centerline{\includegraphics[width=90mm]{Figures/Ch4/Untitled-1 (2).png}}
\caption{Journey Estimation of a Selected Pedestrian: Example 1.}
\label{fig:journey}
\end{figure}

\begin{table}[htbp]

\begin{center}
\begin{tabular}{ 
    >{\columncolor{MintCream}}c 
    >{\columncolor{MintCream}}c
}
\hline
\rowcolor{MidnightGreen}
\textbf{\textit{\textcolor{white}{Origin}} \textcolor{white}{to} \textit{\textcolor{white}{Destination}}} & \textcolor{white}{\textbf{Distance}}\\
\hline
Location 1 to Location 2 & 275 m \\
\hline
Location 2 to Location 3 & 175 m \\
\hline
Location 3 to Location 4 & 150 m \\
\hline
Location 4 to Location 5 & 150 m \\
\hline
Location 5 to Location 6 & 400 m \\
\hline
Location 6 to Location 2 & 50 m \\
\hline\hline
\end{tabular}
\caption{Distances in Metres Between Locations 1 to 6 from Figure \ref{fig:journey_rssi}.}
\label{tab:res/gg2}
\end{center}
\end{table}

% Another example is presented in figures \ref{fig:journey_rssi2} and \ref{fig:journey2} present another example where a journey of an individual pedestrian is estimated.

% \begin{figure}[htbp]
% \centerline{\includegraphics[width=90mm]{Figures/Ch4/journey2.png}}
% \caption{Scattered RSS of Observed advertisement by different Observers from a single volunteer pedestrian representing a journey: Example 2.}
% \label{fig:journey_rssi2}
% \end{figure}

% \begin{figure}[htbp]
% \centerline{\includegraphics[width=90mm]{Figures/Ch4/scatter2.png}}
% \caption{Journey estimation of a selected pedestrian: Example 2.}
% \label{fig:journey2}
% \end{figure}

This study suggests the usefulness of \gls{ble} to understand the behaviour of pedestrians and subsequently, estimate the utilisation of spaces. A mere counting of observed advertisements emanating from beacons carried by volunteer pedestrians over an extended period of time has been seen to be sufficient in identifying aggregated insights of space utilisation. This is favourable from the privacy preservation point of view since the presented method neither requires personal identification nor personal data. Through the \gls{rssi} record for an individual volunteer, it is possible to infer details of their journey and estimate the walking pace of the pedestrian. Such information is useful to identify frequently used routes and the type of pedestrian using the route, whether casual or purposeful. A casual or leisurely walker may walk at a slower pace and may even make stops along the journey. These crucial details are, as presented, easy to identify through this approach.\\

The presented experimental data also poses a challenge. For instance, a higher advertisement count is observed from volunteer number 23 at the Pedestrian Pathway, volunteer 21 at Kirwan St., and volunteer 7 at GG by the church. This presents a likelihood that these volunteers may either reside close to this space or work in the vicinity. Identification of such information limits the privacy preservation aspect of the technology. Therefore, future studies should aim to explore methods to mitigate such concerns.\\

\vspace{10pt}

    \begin{tcolorbox}[
        colframe=white, % Border color
        colback=Ivory, % Background color
        coltitle=white, % Title text color
        title=Section Summary, % Title text
        fonttitle=\bfseries, % Title font style
        sharp corners, % Sharp corners for the main box
        enhanced,
        attach boxed title to top left={yshift=-2mm, xshift=2mm}, % Positioning the title
        boxed title style={colback=DarkSlateGray, colframe=SlateBlue, rounded corners=west, boxrule=0pt, top=2mm, bottom=2mm, right=2mm}, % Title box style
        width=\linewidth, % Width of the box
        boxrule=0pt, % No border for the main box
        drop shadow, % Shadow effect
        rounded corners, % Rounded corners for the main box
    ]
         This study is a pilot use of \gls{ble} as a holistic pedestrian behaviour monitoring system. Through this study, it is possible to identify the choice of activity, for example, leisurely walking or commuting to work, through the estimation of the pace of the pedestrian. This behaviour corresponds to the strategic level behaviour. The choice of route, and entry and exit points are also identifiable in this experiment, which corresponds to the tactical level behaviour. Finally, the local interaction of the pedestrian is also identifiable through the observation of an extended period of \gls{ble} advertisements at a given location. There is a likelihood of further categorisation of the local interaction with the spaces or interactions with other pedestrians if other pedestrians are also participating in the experiment and happen to take a pause at the same time, at the same place, and for the same duration as with another participating pedestrian. However, this is not studied in this experiment. Regardless, this attribute corresponds to the operational-level behaviour. Thus, this study encompasses all the pedestrian behaviour types using \gls{ble}. It is noteworthy that no sophisticated algorithm was used in the analysis, which provides this approach with an edge over other monitoring techniques.
\\

    The study presented in this section is published in a paper titled, Capturing the Behaviour of Volunteer Pedestrians in a Newly-developed University Campus using a Distributed Array of Bluetooth Low Energy devices \citep{Alanbouri2023}, where my contribution as a second author was the identification of part of the contributions, the analysis of the portion of the data reported here, and significant input on co-authoring the paper.
    
    \end{tcolorbox}

